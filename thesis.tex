
%\clearpage

\chapternonum{Acknowledgements}

\loremipsum

\chapternonum{Résumé}

\loremipsum

\dominitoc
\tableofcontents

%\emptypage
\newpage

\setcounter{page}{1}
\pagenumbering{arabic}

\chapternonum{Introduction}

\todo{check le temps global de la these present/passe ?}

\loremipsum

%
% There are worlds out there where the sky is burning, and the sea’s asleep, and the rivers dream;
% people made of smoke and cities made of song. Somewhere there’s danger, somewhere there’s injustice,
% and somewhere else the tea’s getting cold. Come on, Ace. We’ve got work to do.”
%
% It’s like when you’re a kid. The first time they tell you that the world’s turning and you just can’t
% quite believe it ’cause everything looks like it’s standing still… I can feel it: the turn of the Earth.
% The ground beneath our feet is spinning at 1,000 miles an hour and the entire planet is hurtling around
% the sun at 67,000 miles an hour, and I can feel it. We’re falling through space, you and me, clinging to
% the skin of this tiny little world, and if we let go… That’s who I am.
%
% The universe is big. It’s vast and complicated and ridiculous. And sometimes, very rarely, impossible
% things just happen and we call them miracles.
%
% So, you find a breach, probe it, the sphere comes through, 600 feet above London, BAM! It leaves a hole
% in the fabric of reality. And that hole, you think: “Should we leave it alone, should we back off, should
% we play it safe?” NAH, you think: “Let’s make it BIGGER!”
%
% People assume that time is a strict progression of cause to effect, but actually from a non-linear,
% non-subjective viewpoint, it’s more like a big ball of wibbly-wobbly, timey-wimey stuff.
%
% Tracked you down with this. This is my timey-wimey detector. It goes ding when there’s stuff.
% Also, it can boil an egg at 30 paces, whether you want it to or not, actually, so I’ve learned to stay away from hens. It’s not pretty when they blow.
%

%==============================================================
\setcounter{mtc}{3}
\chapter{The Standard Model of particle physics and beyond}
\minitoc
\newpage
%==============================================================

    \todo{fix swap c <-> s}

    \section{Fields and symmetries}

        \subsection{Quantum field theory}

    Before building the Standard Model, we shall first introduce the fundamental objects
    that are used to build a particle physics theory, namely quantum fields. The concept
    of fields corresponds to a degree of freedom at each point of space-time. Originally
    used to describe electrodynamics, this concept was later found so suit well the
    description of many-particle systems with relativistic interactions, something that
    classical mechanics was not able to achieve.

    Quantum field theory apply the idea of quantum mechanics to fields, by treating the
    field $\phi$ as an operator subject to commutation relations analogous to those of
    quantum mechanics. The field can then be expressed as a Fourrier sum of quanta
    creation and annihilation operators. In such a theory, particles are the quanta of the
    field and can be seen as excitations or ripples on this field much like a plane wave.
    Another important element is the concept of virtual particles, which does not have
    any classical correspondance and can be seen as disturbance in the field. Virtual
    particles are not observable per se, but play a role when computing the physical
    observables via perturbation theory.

    The behavior of fields can be described using the powerful and concise Lagrangian
    formalism which introduces a quantity called the Lagrangian density (referred later
    as simply the Lagrangian),
    \eq{lagrangianForm}
    {
        \mathcal{L}(\phi,\partial_\mu \phi)
        =
        T(\phi,\partial_\mu \phi) - V(\phi,\partial_\mu \phi)
    }
    where the terms $T$ and $V$ describes respectively the kinetic and potential of the
    field $\phi$. The least action principle states that one can obtain the equation of
    motion of a system by requiring that the action, defined as
    \eq{actionDefinition}
    {
        \mathcal{S}
        =
        \int_V \mathcal{L}(\phi,\partial_\mu \phi) d^4x
    }
    is stationnary with respect to an infinitesimal variation $\phi \rightarrow \phi +
    \delta\phi$. This principle yields the Euler-Lagrange equation,
    \eq{eulerLagrange}
    {
        \frac{\partial \mathcal{L}}{\partial \phi}
        -
        \partial_\mu
        \left(
            \frac{\partial \mathcal{L}}{\partial (\partial_\mu \phi)}
        \right)
        =
        0
    }
    which can be solved to obtain the equations of motion. This is why the Lagrangian is
    a core element as it summarize all the dynamic of the fields.

    Free fields with spin zero, also called scalar fields are usually noted $\phi$ and
    their dynamic is described by Klein-Gordon's lagrangian :
    \eq{kleinGordonLagrangian}
    {
        \mathcal{L}_\text{free scalar} = \partial_\mu \phi \partial^\mu\phi - m^2 \phi^\dagger \phi.
    }
    Fields with spin one-half, also called fermionic fields are usually noted $\psi$ and
    are ruled by Dirac's lagrangian
    \eq{diracLagrangian}
    {
        \mathcal{L}_\text{free fermion} = \bar{\psi} (i \gamma^\mu \partial_\mu - m) \psi
        \,\,\,\,\,\,\,
        \text{ with }
        \,\,\,\,\,\,\,
        \bar{\psi} = \psi^\dagger \gamma^0
    }
    where $\gamma^\mu$ are the Dirac matrics. Finally, free fields with spin one are called
    vector fields, usually denoted $A_\mu$, and are described by Maxwell's lagrangian :
    \eq{maxwellLagrangian}
    {
        \mathcal{L}_\text{free vector} = -\frac{1}{4} F_{\mu\nu} F^{\mu\nu}
        \,\,\,\,\,\,\,
        \text{ with }
        \,\,\,\,\,\,\,
        F_{\mu\nu}
        =
        \partial_\mu A_\nu - \partial_\nu A_\mu
    }
    It is also possible to describe fields with spin three half or two, but these are beyond
    the scope of this document.

    The structure of the lagrangian $\mathcal{L}$ can be easily analyzed knowing that
    a term with the form $m^2 \phi^\dagger \phi$ corresponds to a mass $m$ for the field $\phi$,
    and a product $k \cdot \phi_1 \phi_2 \phi_3$ corresponds to an interaction
    between the fields $\phi_{i=1,2,3}$ with strength $k$. It is convenient to
    represent such an interaction using Feynman diagrams as they provide a graphical and
    intuitive understanding of what is going on. The Feynman diagram corresponding the
    previous term is sketched on \reffig{fig:feynmanDiagramExample}. In a sense, the
    goal of a particle physicist can be seen as using particles as a mean to understand
    which fields exists, what are their properties and how they couple with each other.

    \insertFigure{feynmanDiagramExample}{0.3}
                 {Example of vertex corresponding to a Lagrangian term
                 $k \cdot \phi_1 \phi_2 \phi_3$.}

        \subsection{Noether's theorem and gauge symmetries}

    A big part of the theoretical work of particle physics is related to the studies of
    the symmetries of the Lagrangian. One of the most remarkable result in physics is
    called the Noether theorem, stating that to every transformation of the space-time
    coordinate and fields that let the Lagrangian invariant, there are two quantities
    called current and charge which are conserved. This is a very powerful theorem as it
    creates a direct link between symmetries and laws of nature. Using this theorem on
    space-time symmetries, one can deduce for instance the conservation of energy-momentum
    from invariance of the Lagrangian according to space-time translations.

    However, a particular interest goes into studying internal symmetries of the field,
    called gauge symmetries. Gauge transformations are transformations built from a gauge
    group and acting on the fields.They take the form :
    \eq{gaugeTransformation}
    {
        \psi(x^\mu)
        \rightarrow
        U(x^\mu) \psi(x^\mu)
    }
    where $U$ is an element of the group. If $U$ does not depend on $x^\mu$, the gauge is
    said to be global, while if it does depend on $x^\mu$, the gauge is said to be local.

    One of the most striking example of the application of gauge symmetry is the case of
    quantum electrodynamics using the group $U(1)$. Starting from the Dirac lagrangian
    describing massive, non-interacting electrons,
    \eq{freeElectronLagrangian}
    {
        \mathcal{L}
        =
        \bar{\psi_e} (i \dslash - m) \psi_e
        \,\,\,\,\,\,\,
        \text{ with }
        \,\,\,\,\,\,\,
        \dslash \definedAs \gamma^\mu \partial_\mu
    }
    and considering a $U(1)$ transformation,
    \eq{U1gauge}
    {
        \psi_e(x^\mu)
        \rightarrow
        e^{i \,\cdot\, q_e \,\cdot\, \theta(x^\mu)} \psi_e(x^\mu),
    }
    one easily finds that $\mathcal{L}$ is invariant under global transformation (i.e.
    considering $\partial_\mu \theta = 0$) but not under local transformation.

    The transformation from \refequation{eq:U1gauge} can be seen as a change in the phase of the field
    $\psi_e$. A global invariance corresponds to the fact that one can choose to offset
    the phase of the field in the same way accross all the universe without changing the
    laws of physics. However, from the principle of locality, we know that what happens
    somewhere in the universe doesn't immediately affects a distant place. In a similar
    manner, the motivation to require local gauge invariance is that we should be able to
    vary continuously the offset we put on the phase of the field accross space-time
    without changing the laws of physics.

    Local gauge invariance can be obtained by introducing a vector field $A_\mu$ which
    transforms according to
    \eq{covariantDerivative}
    {
        A_\mu
        \rightarrow
        A_\mu + \frac{1}{g} \partial_\mu \theta,
    }
    where $g$ is an arbitrary constant we may call coupling constant, and replacing the
    derivative $\dslash$ in the Lagrangian with a covariant derivative $\Dslash$,
    \eq{Dslash}
    {
        \Dslash
        \definedAs
        \dslash - i \cdot q_e g \cdot \gamma^\mu A_\mu
    }
    The lagrangian becomes, after expansion,
    \eq{QEDLagrangian}
    {
        \mathcal{L}
        =
        \underbrace{i \bar{\psi_e} \gamma^\mu \partial_\mu \psi_e}_{\psi_e \text{ kinetic}}
        -
        \underbrace{\frac{1}{4} F_{\mu\nu} F^{\mu\nu}}_{A_\mu \text{ kinetic}}
        -
        \underbrace{m \bar{\psi_e} \psi_e}_{\psi_e \text{ mass}}
        -
        \underbrace{i q_e \bar{\psi_e} \gamma^\mu A_\mu \psi_e}_{\psi_e \leftrightarrow A_\mu \text{ interaction}}
        ,
    }
    where the kinetic term of $A_\mu$ was added with
    \eq{Fmunu}
    {
        F_{\mu\nu} = \frac{i}{g} [D_\mu, D_\nu] = \partial_\mu A_\nu - \partial_\nu A_\mu
    }
    There are now two kinematic terms, a mass term and an interaction term with strength $q_e g$
    between $\psi_e$, the electron field, and $A_\mu$, identified as the photon field.
    Therefore, imposing a local gauge symmetry has led to introduce interaction between
    the two fields. This important result was generalized to non-abelian groups $SU(n)$ by
    Yang and Mills. For these groups, the corresponding transformations are of the form
    \eq{SUNgauge}
    {
        \psi(x^\mu)
        \rightarrow
        e^{i \,\cdot\, q_\psi \,\cdot\, t_k \theta^k(x^\mu)} \psi(x^\mu),
    }
    where $q_\psi$ is the charge of $\psi$ under this transformation, $\theta^k$ are
    arbitrary values and the $t_k$ are the $n^2-1$ generators of $SU(n)$ satisfying the
    Lie algebra commutation relations. To get an invariance with respect to this
    transformation, one is forced to introduce $n^2 - 1$ vector gauge bosons $A^k_\mu$ and the
    covariant derivative becomes
    \eq{CovariantDerivativeSUn}
    {
        \Dslash
        =
        \dslash - i \cdot q_\psi g \cdot \gamma^\mu t_k A^k_\mu
    }
    An important feature of non-abelian groups is that additional terms appears because
    the matrices $t_k$ do not commute and leads interaction terms between the gauge
    bosons $A^k_\mu$.

    \section{The Standard Model of particle physics}
    %====================================================================================

    The Standard Model reflects our current understanding of particle physics. It is
    a quantum field theory constructed with the following ingredients :
    \begin{itemize}
        \item the fermionic fields and their properties ;
        \item the gauge symmetries corresponding to interactions ;
        \item one scalar field leading to spontanneous symmetry breaking, called the
              Higgs field.
    \end{itemize}

    The Standard Model includes two interactions : electroweak and strong. The electroweak
    sector is described by the gauge group $U(1)_Y \times SU(2)_L$ where $Y$ and $L$
    stands for weak hypercharge and weak isospin. A notable property is how it affects
    differently left-handed fermions from right-handed ones. This interaction is
    spontaneously broken by the Higgs field, leading to the known electromagnetism.
    The gauge group describing the strong interaction is $SU(3)_C$, where $C$ stands for
    color. This is an unbroken symmetry with $3^2-1 = 8$ associated gauge bosons. The
    symmetries of the Standard Model can be summarized under this form :
    \eq{symmetriesStandardModel}
    {
        \underbrace{SU(3)_C}_{\text{strong}}
        \,\,\,\,
        \times
        \,\,\,\,
        \underbrace{SU(2)_L \times U(1)_Y}_{\text{electroweak}}
        \,\,\,\,
        \xrightarrow{\text{Higgs field}}
        \,\,\,\,
        \underbrace{SU(3)_C}_{\text{strong}}
        \,\,\,\,
        \times
        %\,\,\,\,
        \underbrace{U(1)_Q}_{\text{electromagnetism}}
    }

    The fermionic fields of the Standard Model are categorized according to their
    properties : quarks are fields that carry both electroweak charges and colors, whereas
    leptons carry only electroweak charges. They are grouped into three families called
    flavour, or generations, sharing the same charges as shown on \reftab{tab:StandardModelFields}.

    Fermions are initially massless and acquire a mass via
    the electroweak symmetry breaking. While it is common to write directly the left-handed
    fermions with lowercase letters, let us write them for the moment with uppercase letter
    to emphasize that these flavour states are not necessarily the mass states. In the same
    vein, it is common to explicitly write the fields $Q_L$ and $\Lambda_L$ as $SU(2)$ doublets
    involving the right-handed counterparts of the left-handed fermions $U_R$, $D_R$ and $E_R$.
    However, we decide here to not introduce any psychological bias in the writing to show
    how these $SU(2)$ components can naturally be identified after the electroweak symmetry breaking.

    \begin{table}
        \centering
        \begin{tabular}{|ccc||ccc|}
            \hline
            1$^\text{st}$ gen. & 2$^\text{nd}$ gen. & 3$^\text{rd}$ gen.  & $Y$  & $L$ & $C$\\
            \hline
            \hline
            $Q^1_L$            & $Q^2_L$            & $Q^3_L$             &  1/3 & 1   & 1  \\
            $U^1_R$            & $U^2_R$            & $U^3_R$             &  4/3 & 0   & 1  \\
            $D^1_R$            & $D^2_R$            & $D^3_R$             & -2/3 & 0   & 1  \\
            \hline
            \hline
            $\Lambda^1_L$      & $\Lambda^2_L$      & $\Lambda^3_L$       &  -1  & 1   & 0  \\
            $E^1_R$            & $E^2_R$            & $E^3_R$             &  -2  & 0   & 0  \\
            \hline
        \end{tabular}
        \caption{Fundamental fermionic fields of the Standard Model and associated charges
        with respect to the fundamental interactions. The convention adopted here is to
        have $L = 1$ and $Y = -1$ for the left-handed lepton fields.}
        \label{tab:StandardModelFields}
    \end{table}

    \subsection{The electroweak sector}
    %====================================================================================

    One of the big challenge of particle physics around the 50's was to understand why
    the weak interaction was short-range whereas the electromagnetic interaction is
    infinite-range. To be able to describe an interaction with range $\orderOf{d}$, one
    would need a massive mediator with a mass $m \sim d^{-1}$. Attempting to describe the
    weak interaction with an $SU(2)_L$ gauge symmetry alone proved unsuccessful as mass terms
    for the gauge bosons are breaking the symmetry.

    Nevertheless, a remarkable property of certain physical system is how their symmetry can
    be spontanneously broken. To understand this, a straightforward experience is to
    place a pen perpendicular
    to a table. This system exhibits an invariance by rotation around the axis of the pen.
    However, it is in an unstable configuration and as soon as the pen is released,
    micro-fluctuations makes it fall in one particular direction. Despite the fact that the
    set of all possible outcomes are symmetric by rotation, the fact that only one of this
    outcome can be realized at a time breaks the cylindrical symmetry of the system.

    The keystone of the Standard Model is the succesful application of this idea to the
    the electroweak symmetry $U(1)_Y \times SU(2)_L$ to obtain massive gauge bosons and
    massive fermions, via a spontanneous breaking of this symmetry first introduced by
    Robert Brout, François Englert and Peter Higgs.

    \subsubsection{Spontanneous breaking of the electroweak interaction}

    Let's introduce a complex scalar $SU(2)$ doublet $\phi$ with $L_\phi = 1$ and $Y_\phi
    = 1$,
    \eq{higgsDoublet}
    {
        \phi
        =
        \begin{pmatrix} \phi_1 \\ \phi_2 \end{pmatrix}
        =
        \frac{1}{\sqrt{2}}
        \begin{pmatrix}
          h_1 \cdot e^{i\theta_1} \\
          h_2 \cdot e^{i\theta_2}
        \end{pmatrix}.
    }
    The Lagrangian of $\phi$ can be written under the form
    \eq{higgsLagrangian}
    {
        \mathcal{L}
        =
        D_\mu \phi D^\mu \phi - V(\phi),
        \,\,\,\,\,\,\,\,
        \text{with}
        \,\,\,\,\,\,\,\,
        V(\phi) = \mu^2 \left| \phi \right| + \lambda \left| \phi \right|^2.
    }
    For $\mu^2 < 0$ and $\lambda > 0$, one gets a degenerated minima for $V(\phi)$ satisfying
    \eq{higgsSolutions}
    {
        \left| \phi \right|
        =
        \sqrt{\phi^\dagger \phi}
        =
        \frac{v}{\sqrt{2}}
        ,
        \,\,\,\,\,\,\,\,
        \text{with}
        \,\,\,\,\,\,\,\,
        v \definedAs \sqrt{\frac{-\mu^2}{\lambda}}.
    }

    Because the symmetry is local, we may perform different isospin rotations for different
    values of $x^\mu$ and define the unitary jauge such that $h_1 = \theta_1 = \theta_2
    = 0$ and $h_2(x^\mu) = v + h(x^\mu)$ with $h = 0$ in vacuum. $v$ therefore corresponds
    to the vacuum expectation value of the Higgs field, and $h$ to a longitudinal excitation
    of the field, as summarized on \reffig{fig:theory/higgsPotential}.
    Note that even though this choice is arbitrary, one may work with $h_1$ instead of
    $h_2$ and would still obtain a consistent picture at the end. In our case, $\phi$
    takes the form
    \eq{higgsDoubletUnitaryJauge}
    {
        \phi
        =
        \begin{pmatrix} \phi_1 \\ \phi_2 \end{pmatrix}
        =
        \frac{1}{\sqrt{2}} \begin{pmatrix} 0 \\ v + h \end{pmatrix}.
    }

    \insertFigure{theory/higgsPotential}{0.4}
                 {Representation of the Higgs potential with $\mu^2 < 0$ and $\lambda > 0$.
                 The factor $1/\sqrt{2}$ is ommited for clarity.}

    To understand how this field impacts the physics of $SU(2)_L \times U(1)_Y$ let us
    look at the covariant derivative. We may note $W^{i=1,2,3}$ and $B$ the gauge bosons
    and $g_W$ and $g_B$ the coupling constants associated to the electroweak sector.
    \eqalign{CovariantDerivativeSU2U1}
    {
        D_\mu \phi
        & =
        \partial_\mu \phi - i g_B Y_\phi B_\mu \phi - i g_W L_\phi t_a W^a_\mu \phi \nonumber\\
        & =
        \partial_\mu \phi - i
        \begin{pmatrix}
            g_W  W_\mu^3 + g_B  B_\mu    &   g_W (W_\mu^1 - i  W_\mu^2) \\
            g_W (W_\mu^1 + i   W_\mu^2)  & - g_W W_\mu^3 + g_B  B_\mu   \\
        \end{pmatrix}
        \phi\\
    }
    This covariant derivative can be rewritten under the form
    \eqalign{CovariantDerivativeSU2U1NewForm}
    {
        D_\mu \phi
        & =
        \partial_\mu \phi - i
        \begin{pmatrix}
            \frac{g_W^2 - g_B^2}{g_{WB}} Z_\mu + \frac{2 g_W g_B}{g_{WB}} A_\mu & g_W W_\mu^-       \\
            g_W W_\mu^+                                                         & - g_{WB} Z_\mu    \\
        \end{pmatrix} \phi.
    }
    where we introduced :
    \eqalign{DefinitionAZW}
    {
        g_{WB}     & = \sqrt{g^2_W + g^2_B} \nonumber\\
        W^\pm      & = W^1 \pm i W^2        \nonumber\\
        \begin{pmatrix}
            Z \\ A
        \end{pmatrix}
        & =
        \frac{1}{g_{WB}}
        \begin{pmatrix}
            g_W & -g_B \\
            g_B & g_W
        \end{pmatrix}
        \begin{pmatrix}
            W^3 \\ B
        \end{pmatrix}
    }

    Expanding the kinetic term of $\phi$, one find the following terms appearing :
    \eqalign{HiggsFieldKinetic}
    {
    D_\mu \phi D^\mu \phi & = \underbrace{\frac{1}{2}\partial_\mu \phi \partial^\mu \phi}_{h \text{ kinetic}}
                            + \underbrace{(\frac{v^2 g_W^2}{2}) W_\mu W^\mu}_{\text{massive } W^\pm}
                            + \underbrace{(\frac{v^2 g_{WB}^2}{2}) Z_\mu Z^\mu}_{\text{massive } Z}
                            + \underbrace{(0) A_\mu A^\mu}_{\text{massless } A_\mu}
                            \nonumber\\
                            & + \underbrace{(v g_W) h W_\mu W^\mu}_{hWW \text{ interaction}}
                            + \underbrace{(v g_{WB}) h Z_\mu Z^\mu}_{hZZ \text{ interaction}}
                            + \underbrace{(\frac{g_W^2}{2}) hh W_\mu W^\mu}_{hhWW \text{ interaction}}
                            + \underbrace{(\frac{g_{WB}^2}{2}) hh Z_\mu Z^\mu}_{hhZZ \text{ interaction}}
    }
    And the potential term yields, up to a constant,
    \eq{HiggsFieldPotential}
    {
        V(\phi) = \underbrace{(\mu^2) h h}_{h \text{ mass}}
                + \underbrace{(\lambda v) h h h}_{hhh \text{ interaction}}
                + \underbrace{(\frac{\lambda}{4}) h h h h}_{hhhh \text{ interaction}}.
    }

    Introducing the $SU(2)$ doublet $\phi$ with the right properties therefore lead to two
    things. First, the prediction of a scalar, observable boson $h$ which is called the
    Higgs boson and is a radial excitation of the Higgs field. Second, new terms in the
    Lagrangian that break the invariance under $SU(2)_L \times U(1)_Y$, namely the mass
    terms of $W$ and $Z$ bosons which are the mass eigenstates of the $W^i$ and $B$ bosons.
    The phenomenon can be summarized that the breaking of the symmetry leads to four massless
    Goldstone bosons, one for each degree of freedom of the Higgs field, and three of them
    are absorbed by the $W^i$ and $B$ bosons which get massive. The last degree of freedom
    leads to the Higgs field $h$.

    We find that the Lagrangian is now invariant under a symmetry $U(1)_Q$ with
    $Q \definedAs \frac{1}{2}(Y+L^3)$ which is identified as the electromagnetic interaction, and
    whose gauge boson is the massless $A_\mu$ which is identified as the photon. With
    respect to electromagnetism, right-handed quarks $U_R$ and $D_R$ get charges equal to
    $2/3$ and $-1/3$, and right-handed leptons $E_R$ get charges to -1.
    The component of $\Lambda_L$ under $SU(2)_L$, $\Lambda_L = (\lambda_{+L}, \lambda_{-L})$,
    have electromagnetic charges equal to $0$ and $-1$. Since $\lambda_{-L}$
    now transforms the same way than $E_R$ under $SU(3)_C \times U(1)_Y$, we are
    tempted to simply call it $E_L$. The same observation goes for the left-handed
    quarks $Q_L$.

    \subsubsection{Fermions masses and mass eigenstates}

    Since we introduced a new field $\phi$, we may introduce also new terms in the
    Lagrangian corresponding to coupling between the scalar field $\phi$ and the fermion
    fields, so-called Yukawa coupling. Such terms can exist provided that they respect the
    symmetry of the Lagrangian. For instance, terms of the form $\phi$-quark-lepton can
    \emph{not} be introduced as they would break the $SU(3)_C$ invariance. However, we can
    introduce terms of the form $E^i_R\phi^\dagger\Lambda^j_L$. Notice how $i$ is not
    necessarily equal to $j$, meaning that there can be mixing between two different generations.
    All the terms for leptons can be written with the form :
    \eqalign{YukawaCoupling}
    {
        \mathcal{L}_{\text{Yukawa}}
        & =
        - \sum_{i,j} (y^{ij} \bar{E}^i_R \phi^\dagger \Lambda^j_L + \text{h.c.})
        \,\,\,\,\,\,
        \text{with } \Lambda_L = \begin{pmatrix} N_L \\ E_L \end{pmatrix}
        \,\,\,\,\,\,
        \nonumber\\
        & =
        - \frac{v+h}{\sqrt{2}}\sum_{i,j} (y^{ij} \bar{E}^i_R E^j_L + \text{h.c.})
    }

    The matrix $y^{ij}$ is not necessarily diagonal. However we can redefine the flavor
    eigenstates $\Lambda_R$ and $E^i_R$ such that $y^{ij}$ is diagonal. Let's label these
    eigenstates $(e,\mu,\tau)$. We simply obtain :
    \eqalign{YukawaCoupling2}
    {
        \mathcal{L}_{\text{Yukawa}}
        & = (\frac{y^e v}{\sqrt{2}}) \cdot \bar{e}_R e_L + (\frac{y^\mu v}{\sqrt{2}}) \cdot \bar{\mu}_R \mu_L + (\frac{y^\tau v}{\sqrt{2}}) \cdot \bar{\tau}_R \tau_L\nonumber\\
        & + (\frac{y^e}{\sqrt{2}}) \cdot h \bar{e}_R e_L + (\frac{y^\mu}{\sqrt{2}}) \cdot h \bar{\mu}_R \mu_L + (\frac{y^\tau}{\sqrt{2}}) \cdot h \bar{\tau}_R \tau_L\nonumber\\
        & + \text{h.c.}
    }
    The first line shows how the vacuum expectation value $v$ leads to mixing between
    the apriori unrelated $e_L$ and $e_R$ fields. We may simply summarize the situation
    by saying that the field $e = (e_R, e_L)$ obtains a mass term $\frac{y^e v}{\sqrt{2}}
    \bar{e} e$. The remaining component of the $\Lambda^i_L$ are relabelled
    $(\nu_{e,L},\nu_{\mu,L},\nu_{\tau,L})$ and remain massless as they don't have any
    $\nu_R$ counterpart to mix with. The reader should also notice how these mass terms
    for fermions are different from the those of the gauge bosons. The masses of gauge bosons
    are directly determined by the electroweak coupling constants and Higgs field expectation
    value while the fermion masses is related to a somewhat arbitrary constant $y$.

    The situation for quarks is analogous, with the exception that once that both components
    of $Q_L$ have a right-handed fermion to mix with. A complication appears, which is that
    it's not possible to simultaneously diagonalize the Yukawa matrices of the up-type
    quarks and down-type quarks. We therefore only diagonalize the up-type matrix and
    label the new flavor and mass eigenstates with $(u,s,t)$. For the down-type quarks,
    relabeled with $(d,c,b)$, there are off-diagonal elements remaining which mix the
    different flavor eigenstate. This corresponds to the so-called CKM matrix.

    \reftab{tab:StandardModelFieldsAfterElectroweakBreaking} summarize the observable
    fermions after breaking the electroweak symmetry and their associated charges.

    \begin{table}
        \centering
        \begin{tabular}{|ccc||cc|}
            \hline
            1$^\text{st}$ gen. & 2$^\text{nd}$ gen. & 3$^\text{rd}$ gen.  & $Q$  & $C$\\
            \hline
            \hline
            $u$                & $s$                & $t$                 &  2/3 & 1  \\
            $d$                & $c$                & $b$                 & -1/3 & 1  \\
            \hline
            \hline
            $\nu_{e,R}$        & $\nu_{\mu,R}$      & $\nu_{\tau,R}$      &  0   & 0  \\
            $e$                & $\mu$              & $\tau$              & -1   & 0  \\
            \hline
        \end{tabular}
        \caption{Observable fermionic fields after breaking of the electroweak symmetry
        by the Higgs field with their associated charges under $SU(3)_C \times U(1)_Q$.}
        \label{tab:StandardModelFieldsAfterElectroweakBreaking}
    \end{table}

    \subsection{The strong interaction \label{sec:strongInteraction}}
    %====================================================================================

    The group $SU(3)$ corresponds to the strong interaction and affect quarks. The
    associated gauge bosons are called the gluons and the charge is called color, which
    is at the origin of the name quantum chromodynamic (QCD) which designate the
    study of this interaction. As $SU(3)$ is a non-abelian group, the gluons are also
    carrying color and interact with each other.
    The name \emph{strong} comes from the value of the coupling constant, being larger
    than the weak interaction by a factor $\orderOf{100}$. The Lagrangian writes :

    \eq{QCDLagrangian}
    {
        \mathcal{L}_\text{QCD} = \bar{\psi} (i \Dslash - m) \psi - \frac{1}{4} G^a_{\mu\nu}G_a^{\mu\nu}
    }
    with the gluonic field tensors equal to
    \eq{QCDFieldTensors}
    {
        G^a_{\mu\nu} = \partial_\mu A_\nu^a - \partial_\nu A_\mu^a + g f_{abc} A^b_\mu A^c_\nu
    }
    \todo{where is the CP phase in this...}

    $SU(3)$ is unbroken and therefore its associated gauge bosons, the gluons, are massless.
    Because of this, one could conclude that the range of the strong interaction is infinite
    just like electromagnetism is. However, when computing the evolution of the strong coupling
    as function of the energy scale, it is found to tend to zero at high energies
    and tend to infinity at low energies. This leads to the two following key features of the strong
    interaction.

    At high energies, i.e. short distances, two colored particles are not much
    affected by each other presence. This is referred to as assymptotic freedom. In comparison,
    the same situation in electromagnetism with two particles close to each other feel important
    attraction or repulsion and are therefore not free.

    At low energies, i.e. long distances, the intensity of the interaction grows. Therefore
    to split a hadron into individual quarks, someone will need to give energy to the
    system up to the situation where this energy will be converted into new coloured
    particles that will form new hadrons. This phenomena is called confinment and explains
    why it is not possible to observe free quarks or gluon in nature. Interestingly,
    confinment is a property which is still not completely demonstrated from a mathematical
    point of view.

    In the context of high energy physics, confinment translates into the production of
    jets of hadrons when quarks or gluons are produced in collisions. Jet physics is
    therefore a crucial aspect when reconstructing an event in a particle collider.

    During the previous section, we have seen how the particle masses arises from the
    introduction of the Higgs field breaking the symmetry. However it is remarkable to notice
    that the mass of hadrons, which ultimately composes everyday matter, comes essentially
    from the kinetic and binding energy holding the quarks and not from the quark masses
    themselvesi \todo{relate this to chiral symmetry breaking}.

    \subsection{The success of the Standard Model \label{sec:standardModelSuccess}}
    %====================================================================================

    The Standard Model includes 19 free parameters : ten fermion and one scalar mass, three
    coupling parameters, four quark mixing matrix parameters, the Higgs vaccum expectation
    value and the strong CP-violating phase.

    The physical observables such as the cross-section associated to a given process
    can be predicted following a pertubative development. For instance, the observation
    of the processus $e^+ e^- \rightarrow \mu^+ \mu^-$ in a theory containing only an $U(1)_Q$
    interaction is given by the superimposition of several processes as represented on
    \reffig{fig:perturbativeDevelopment}. A given observable can be computed at leading-order
    (LO) by considering only the first diagram in the development, or next-to-leading-order (NLO)
    and so on with increasing accuracy. This makes virtually any observable sensitive
    to the rest (or at least) a part of the other sectors and deviations can indicate
    for instance that new particles to the loops.

    \begin{figure}
        \centering
        \eqalign{pertubativeDevelopmentEquation}
        {
            \vcenter{\hbox{\includegraphics[width=0.25\textwidth]{feynmanDiagrams/output/ElElToMuMu}}}
            =
            \vcenter{\hbox{\includegraphics[width=0.20\textwidth]{feynmanDiagrams/output/ElElToMuMu_tree}}}
            +
            \vcenter{\hbox{\includegraphics[width=0.20\textwidth]{feynmanDiagrams/output/ElElToMuMu_oneLoop1}}}
            \nonumber
            \\
            +
            \vcenter{\hbox{\includegraphics[width=0.20\textwidth]{feynmanDiagrams/output/ElElToMuMu_oneLoop2}}}
            +
            \vcenter{\hbox{\includegraphics[width=0.20\textwidth]{feynmanDiagrams/output/ElElToMuMu_oneLoop3}}}
            +
            \vcenter{\hbox{\includegraphics[width=0.20\textwidth]{feynmanDiagrams/output/ElElToMuMu_oneLoop4}}}
            +
            ...
            \nonumber
        }
        \caption{Illustration of development of the process $e^+ e^- \rightarrow \mu^+ \mu^-$
        in a model with only electrons and muons coupling to a photon. \label{fig:perturbativeDevelopment}}
    \end{figure}

    The parameters and consistency of the Standard
    Model has been extensively measured and tested in several experiments, in particular
    at the LEP experiments, the Belle and BaBar experiment, and at the Tevatron. In 2012,
    the CMS and ATLAS experiments at the LHC discovered the last remaining particle, the
    Higgs Boson with a mass $m_H \sim 125\GeV$. The full mass spectrum of the particles
    of the Standard Model is summarized on \reffig{fig:theory/fermionMasses} and
    shows a clear hierarchy between the different fermion generations.

    \reffig{fig:theory/standardModelFit} shows the result of the global fit of
    observables of the Standard Model. In such a fit, one observable at a time is set free
    while others are taken from experiment. The value of the free observable is then
    predicted from the model, and compared to its experimentally measured value. The
    difference is finally compared to the experimental uncertainty. This allows to have a
    global test of the consistency of the model, as any significant deviation between the
    predicted and measured value would indicated a flaw. The results indicate that the theory
    is consistent, as no deviation larger than 3$\sigma$ is found.

    \insertFigure{theory/fermionMasses}{0.6}
                 {Masses of the fermions and bosons of the Standard Model. The photon
                 $\gamma$, gluon $g$ and neutrinos $\nu_\ell$ are not represented as they
                 are massless in the context of the Standard Model.}

    \insertFigure{theory/standardModelFit}{0.43}
                  {Global fit of observable of the Standard Model}

    \section{Shortcomings of the Standard Model}
    %====================================================================================

    Despite its success, the Standard Model contains theoretical open questions and
    experimental facts unexplained. For these reasons, it is thought to be still incomplete
    or to be an effective theory valid only at low energy hiding a higher
    degree of symmetry at high energy. In this section, we focus especially on the hiearchy
    problem and dark matter as they will be the most relevant one in the context of this
    document, and discuss briefly other open questions.

        \subsection{The hierachy problem}

    The hierarchy problem is related to the prediction of the physical Higgs boson mass.
    Before stepping into this problem, let's first look at the case of the mass of fermions
    and gauge bosons.

    To compute a physical, observable parameter in a quantum field theory, one needs to
    integrate over all possible quantum corrections related to this parameter. In the case
    of the cross-section of a process, this means considering all loops and diagrams
    with the same initial and final state. In the case of the mass of a particle, one is
    interested to diagrams that contribute to the propagator. An example of diagram contributing
    to the propagator of the electron is given in \reffig{fig:feynmanDiagrams/output/electronPropagator}. We may
    express the observable mass $m_e$ of the electron as the sum of the bare mass $m^0_{e}$,
    that is to say the actual parameter inside the lagrangian, and $\Delta m_e$ representing
    the contributions from the quantum corrections :

    \eq{electronMassCorrection}
    {
        m^2_e \, = \, {m_e^0}^2 + \Delta m_e^2
    }

    \insertFigure{feynmanDiagrams/output/electronPropagator}{0.4}
                 {Example of contribution to the electron propagator.}

    For effective theories, the computation of quantum corrections is done with respect to
    an energy cutoff $\Lambda$, representing the scale at which new physics is expected
    to play a significant role. In the case of \reffig{fig:feynmanDiagrams/output/electronPropagator}, this mean
    that we shall integrate over all momentum inside the loop up to the scale $\Lambda$.
    For instance, if one considers electrodynamics as an effective theory of electroweak,
    at one-loop level :

    \eq{electronMassCorrection2}
    {
        \Delta m_e \, \simeq \, \frac{\alpha}{4\pi} \, m_e^0 \, \text{ln}\left(\frac{\Lambda}{m_e^0}\right)
    }

    With $\Lambda = \Lambda_{\text{electroweak}} = \orderOf{100\GeV}$, we find that
    $\frac{\Delta m_e}{m_e} \sim 20\%$. It is a relatively small correction, in the sense
    that it is not surprising to have a mass for the electron that is $\orderOf{1\GeV} \ll \Lambda_{\text{electroweak}}$.
    The most important feature in \refequation{eq:electronMassCorrection2} is the fact that $\Delta m_e$
    is proportional to $m_e^0$ and not $\Lambda$ as it means the mass will not diverge
    when the scale grows.

    There is a remarkable and more general property behind this term which is that a
    gauge invariant lagrangian cannot generate corrections that are breaking the symmetry. In
    the case of electrodynamics with a massive electron, the mass term is already breaking
    the chiral symmetry $U(1)_L \times U(1)_R$, but because this is a so-called soft-breaking
    term, the property should still hold in the limit where $m_e^0 \rightarrow 0$.
    Therefore the correction cannot be proportionnal to $\Lambda$ but only to $m_e$, by
    dimensional analysis. It is common to refer to this by saying that chiral symmetry
    protects the mass of fermions from diverging. The same fact is observed for gauge
    bosons masses in the case of the electroweak symmetry breaking by the Higgs field :
    here, the gauge bosons are protected by the gauge invariance preventing the mass terms.

    Looking at the mass of the Higgs boson, it is found that the Standard Model does not include
    any mechanism that prevent the mass of a scalar boson from diverging. The actual
    computation for the correction from a fermion loop gives :

    \eq{higgsMassCorrectionFromFermion}
    {
        \Delta m_h^2 \, \propto \, m_f^2 \Lambda^2,
    }

    which diverge with $\Lambda$. If one is confident that the Standard Model is valid
    up to the Planck scale, where gravity will obviously play a significant role, then
    the observable Higgs mass is

    \eq{higgsMassCorrectionTotal}
    {
        m_h^2 \, \simeq \, {m_h^0}^2 + \kappa \cdot m^2_\text{Planck}
    }

    where $\kappa$ is a function of the Standard Model coupling and masses. The
    three quantities $m_h^0$, $\kappa$, and $m_\text{Planck}$ are \emph{a priori} unrelated
    to each other from the point of view of the theory. Unless the parameters of the
    Standard Model conspire with each other and are \emph{fine-tuned} at twenty decimal
    places, there is no reason to expect that $m_h \ll m_\text{Planck}$. However we know
    that $m_h \sim \orderOf{10^2\GeV} \ll m_\text{Planck} \sim \orderOf{10^{19}\GeV}$ which
    is not natural. This is referred to as the hierarchy problem, as from this considerations
    there is no reason to expect such a large hierarchy between the electroweak scale and
    the Planck scale.

        \subsection{Dark matter}

    Dark matter is another puzzle of current cosmology and fundamental physics in general.
    The first main observation leading to the dark matter hypothesis came from the measurement
    of rotation curves of galaxies. These curves showed the evolution of the radial velocity
    of stars inside the galaxy as function of their distance to the galactic center. One
    can predict such curves by infering the galaxy's mass repartition from models and
    spectrometry, and applying Newton's law of gravitation. The observed curves are
    well-described for the central region of the galaxy but then instead of decreasing
    with distance, stayed approximately constant.

    This observation suggested that there is a halo of invisible matter around
    galaxies, interacting gravitationnaly but not electromagnetically with the rest of the
    matter, hence the denomination \emph{dark} matter. Further experimental measurements
    provided indirect evidences for the existence of such dark matter. One of the main
    ones are using the technique of gravitational lensing that allows the observer to
    infer the mass of a galaxy or a cluster from the bending of the light emitted by
    an object further in the background. Nowadays, dark matter is part of the standard
    model of cosmology as there are many astrophysical and cosmological phenomenom which
    cannot be described without it, and has been found to represent $\orderOf{80\%}$ of
    the matter content of the universe.

    However the nature of dark matter is still unknown and no clear direct evidence has
    been found for it. It is now commonly admitted that there is no suitable candidate
    for it in the Standard Model. Dark matter candidates must be gravitationally interacting,
    not be short-lived and must not be baryonic. Last but not least, it must also be
    cold, that is to say that it must have low kinetic energy.

    If one assume that dark matter is made of a single particle $X$, a simple model gives that
    the relic density $\Omega_X$ is proportionnal to $m_X^2 / g^4_X$ where $m_X$ is the
    mass of the dark matter particle and $g_X$ is the coupling constant in the process
    $XX \leftrightarrow q\bar{q}$. It is remarkable to note that taking $\Omega_X = \orderOf{0.1}$
    from cosmological observations and $g_X \sim 0.6$ from the weak interaction, then
    we expect $m_X = \orderOf{100\GeV}$. This encourage physicist to look for weakly
    interacting massive particles (or WIMPs) which are predicted by many models beyond
    the Standard Model.

        \subsection{Other open questions and criticism}

    Other open questions or unexplained facts add fuel to the belief that the Standard Model
    is not the final theory.

    \begin{itemize}
        \item \textbf{Dark energy} - The standard model of cosmology, so-called $\Lambda$CDM model,
            contains a non-null cosmological constant called $\Lambda$. The cosmological
            constant is initially an additional authorized term in Einstein's equation
            which relates space-time curvature to energy-momentum. Such term is necessary
            to explain the observed accelerated expansion of the universe from the study
            of the redshift of supernovas. The cosmological constant can be interpreted
            as vacuum energy, often referred to as dark energy. Despite the fact that this
            is in principle in agreement with quantum field theory which predicts vacuum
            energy from vacuum quantum fluctuations, there is a complete mismatch, of $\orderOf{10^{120}}$,
            between the small measured value of $\Lambda$ and its prediction
            by quantum field theory. This is often called the vacuum catastrophe or the worst
            prediction in the history of physics. A big challenge of current physics therefore
            consists in understanding the nature of dark energy and why it is so small
            compared to prediction.
        \item \textbf{Matter-antimatter asymmetry} - It is a well established fact that
            the baryonic content of the Universe is made of matter. Because matter and antimatter
            should have been produced in equal proportion during the big bang, there must
            be sources of asymmetry that led to a state dominated by matter. Sakharoved
            established a set of conditions that would
            induce such an asymmetry. Among them is that the symmetries C and CP, i.e. inversion
            of charge and inversion of both charge and parity, is violated. The Standard Model
            with massless neutrinos contains two sources of CP-violation originating from
            the CKM matrix and the CP-violating phase in the QCD sector. So far however,
            these sources are not important enough to explain the magnitude of the baryon
            asymmetry.
        \item \textbf{Neutrino masses} - It has been measured by experiments that neutrinos
            can oscillate from one flavor to another. This indicate that neutrinos should
            be massive and have different flavor states than mass states much like down-type
            quarks and the CKM matrix. The problem of neutrino being massive is itself not
            so much a problem as it is possible to introduce right-handed neutrinos in the
            field content of the Standard Model. However the exact mechanism depends on
            wether neutrinos are Dirac or Majorana fermions. The main source of interrogation
            is to understand why neutrinos have such a small mass compared to the other
            fermions.
    \end{itemize}

    \begin{itemize}
        \item \textbf{Strong CP problem} - As introduced before, there is a term allowed in
            the QCD lagrangian that introduces CP-violation via a phase $\theta$. The value
            of this parameter has been so far found to be very close to zero despite
            lack of theoretical arguments for it not to be of the order of 1. This is a situation
            similar to the hierarchy problem where we can expect that there is a mechanism
            at stake protecting this phase to be too different from zero.
        \item \textbf{Quantum gravity description} - Gravitation is currently not described
            by the Standard Model. This underlying problem, which is to know how to unify general
            relativity with quantum field theory, is one of the biggest problem of modern
            physics. If it exists, the gauge boson associated to the gravitation would
            need to have a spin equal to 2. It is however not known if and how such
            a theory would be renormalizable as it leads to uncancellable divergences.
        \item \textbf{Forces unification} - Several times in the history of physics,
            phenomenons have been understood to have a common structure and
            been unified. The simple name electromagnetism has in its ethymology a clear
            reference to electrodynamic and magnetism. It is therefore natural for physicist
            to look for possibilities of unification, in our case regarding the electroweak
            and strong interaction, and ultimately with gravitation. \todo{mention coupling constant convergence}

        \item \textbf{Free parameters and arbitrary field content} -
            Finally, there are conceptual interrogations about the fact that the quarks have
            such different masses, in particular regarding the top quark as it is the
            heaviest known particle. The same kind of conceptual interrogations goes about
            the number of fermion families : is there any fundamental reason to have three
            families instead of one, two, or an infinity of them ? More generally, there
            are numerous fact about the Standard Model parameters and field content that
            feel arbitrary and may hide an underlying, more fundamental pattern.
    \end{itemize}


    \section{Theories beyond the Standard Model}
    %====================================================================================

        \subsection{Zoology of Standard Model extensions}

        Since a few decades, extensions to the Standard Model have been proposed and
        studied with the aim to adress its shortcomings. One can attempt to categorize
        them as function of the fundamental ingredient they add to the theory, either
        additional symmetries, space-time dimensions or field content.

            \subsubsection{Additional or extended symmetries}
            % -> particles / gauge / space-time

        One of the possibility to extend the Standard Model is to look at symmetries.
        There are however different kind of symmetries, which do not impact the theory
        the same way, namely space-time symmetries or gauge symmetries. The only possible
        extension of space-time symmetries is supersymmetry which is discussed later
        in a dedicated section.

        It's possible to think about gauge groups which include the Standard Model group,
        $SU(3) \times SU(2) \times U(1)$ to create Grand Unification Theories (GUT) in which
        the coupling constants are unified. Such groups are also motivated because they
        naturally introduce a symmetry between leptons and quarks as one can think of
        when looking at the pattern of the Standard Model.

        The most simple group one can find is $SU(5)$ which would be a broken symmetry at
        low energy just as the electroweak symmetry is. Rotations in the $SU(5)$ space
        can mix the quark and lepton fields which led to a prediction of proton decay
        incompatible with experimental measurements. Grand-unificiation with $SU(5)$ is
        now mostly abandonned but remains an idea studied for larger groups such as $SO(10)$.

            \subsubsection{Additional dimensions}

        The second possibility is to add extra space-time dimensions. While everyday's life
        is only composed of 3+1 (space and time) dimensions, theoretical mechanism have been
        discovered that could explain why additional dimensions are noticeable at our scale.
        The extra dimension is usually considered to be a space-like dimension, as an
        additional time-like dimension would break causality. Extra dimensions theories
        are of particular interest for unifying interaction with gravitation and answering
        the hierarchy problem.

        Large extra dimensions (ADD) theories are based on the idea that the Standard Model
        interactions take place only in the four dimensions and gravity would be
        allowed to travel in the extra dimensions. At large distances, gravity would
        be diluted while at low distances (i.e. high energies) the Standard Model interaction
        and gravitation would have coupling constant that are of the same order. A phenomenological
        consequence is that microscopic black holes could be produced in high-energy collisions
        provided that the collision probe distances lower than the Schwarzschild radius.

        The Randall-Sundrum (RS) scenarios consider a five dimensions universe made of two
        four dimensions branes separated by an extremely warped fifth dimension. One of
        the brane corresponds to the Standard Model physics while the other brane would be
        the realm of gravity. The distance separating the two brane is directly linked to
        the difference of energy scale between the Standard Model sector and the gravitation
        sector. A phenomelogicial prediction of these models is the production of gravitaton
        excitations at the TeV scale.

        Finally, universal extra dimensions (UED) theories are based on the idea that
        the extra dimensions are compactified such that they are not noticeable at macroscopic
        scales. An useful analogy to understand the compactification of a dimension is
        consider that, for instance, a guitar string seen from very far looks like a one-dimensional
        object whereas if seen from very close exhibits a second periodic dimension around
        the string axis because of its thickness. The additional dimensions are therefore
        parametrized as function of a radius. Phenomenologically, one would observe a discrete
        mass spectra for the Standard Model particles corresponding to excitations along
        the compactified dimensions.

            \subsubsection{Additional or modified field content}

        A third possibility is to introduce new ad-hoc fields of modify existing ones
        to adress the shortcomings of the Standard Model.

        One first elegant idea following this principle is to explain the lightness of the
        Higgs by considering it is a composite object. This idea is inspired from QCD where
        light scalars are encountered and do not exhibit a naturalness problem. Pions,
        for example, can be seen as Goldstone bosons of the spontanneous breaking of chiral
        symmetry, and therefore are massless. However in practice, chiral symmetry is not exact as
        the underlying quarks do have mass, thus pions are only pseudo-Goldstone boson and get a mass,
        but they are still protected by the approximate chiral symmetry from being heavy.
        Applying the same idea to the Higgs would therefore answer the hiearchy problem.
        This also means that there would be a new interaction whom the Higgs would
        be just the lightest hadron-like state and other state could be seen at higher
        energy. Another indication of compositeness would also come from different couplings
        to gauge bosons.

        Other hypotheses of new fields include axions, an idea first introduced to solve
        the strong CP problem and eventually led to a dark matter candidate. Axions would
        have masses lower than $\orderOf{1~\text{eV}}$ while still satisfying criteria to
        be cold dark matter candidates.

        Focusing on the search for dark matter candidates leads to proposing that dark matter
        is not made of just one, but several kind of particles regrouped in the concept
        of an hidden sector. There are several possibilities to describe the link between
        the Standard Model and this hidden sector, for instance by assuming that the Higgs
        relate the two sectors, or by ntroducing a new
        $U(1)_H$ gauge group whose gauge boson would be a so-called dark photon.

            \subsubsection{The anthropic principle}

        Finally, one of the alternative to theories beyond the Standard Model is to
        simply accept the Standard Model as it is. This is strongly related to Einstein's
        interrogation regarding if Nature had any choice at teh creation of the universe.
        It can be argued that most of the shortcomings of the Standard Model related to
        gravitation and cosmology might simply be unsolvable at scales accessible by
        particle physics experiments. The most important remaining issue are related to the
        fine-tunning in the hierarchy problem and the arbitrary of the other parameters of
        the theory.

        The anthropic principle states that when studying the universe, it
        should be taken into account that conscious life is possible in it. It might be
        that our universe is simply one realization of universe among many others in what
        is called a multiverse. What appears as to be fine-tunning would in fact be
        equivalent to natural selection at the scale of the mutiverse where the observed
        universes can only be the one allowing conscious life.

        This view is however controversial as it yields very little if no testable
        prediction except understanding how likely life would be if the parameters of
        the Universe would have been different \todo{0807.3697v1}.

        \subsection{Supersymmetry}

        \subsubsection{General idea and motivations}

        Here, we propose to have a dedicated look at the idea of Supersymmetry. Supersymmetry,
        abbreviated SUSY, emerged around the 70's and designates a symmetry between bosons
        and fermions. Mathematically, we can write such a transformation :
        \eqalign{SUSYtransformation}
        {
            Q \rvert \text{F} \rangle \rightarrow \rvert \text{B} \rangle \nonumber
            \\
            Q \rvert \text{B} \rangle \rightarrow \rvert \text{F} \rangle
        }
        where the subscript $f$ and $b$ refer to fermion and boson. Here, $Q$ is a \emph{global}
        symmetry, i.e. it does not depend of space-time, and is therefore not a gauge
        symmetry. As it relates two states with different spins, it must itself have a
        spin 1/2. This means \todo{but why...} that SUSY is a space-time symmetry
        in the sense that it is a non-trivial extension of the Poincaré algebra. This is
        an important consideration : it has been demonstrated by Colemand and Mandula \todo{ref} that
        the only possible symmetry of Nature are the Poincaré symmetry and the gauge
        symmetries. Haag, Lopuszanski and Sohnius later relaxed this theorem by
        noticing that another type of symmetry is possible and precisely corresponds to
        supersymmetry. \todo{Haag, Lopuszanski, Sohnius, 1975}

        A remarkable property of SUSY is that trying to gauge it, meaning making the
        transformation space-time dependent, yields a description of gravitation called
        supergravity or, abbreviated, SUGRA. Supersymmetry also provides a link with
        string theory which is a candidate of theory of everything.

        To each fermion is therefore associated a boson with same charges, to each boson
        is associated a fermion. The particule associated to another by SUSY is called
        superpartner, and can be put together in a superfield. To ensure the consistency
        of the theory\todo{check this}, superpartners of vector bosons have spin 1/2 and superpartners
        of fermions have spin 0.  SUSY alone predicts that a particle and its superpartner
        should have the same mass. However this cannot be the case in pratice as no
        such thing has been observed by experiments, meaning that supersymmetry must be
        a broken symmetry.

        There are several mechanism that are candidate to spontanneously break SUSY. Some
        are based on a gravity mediated supersymmetry breaking, others involve a gauge
        mediated mechanism, and another possibility is to have an anomaly mediated breaking.
        Since the exact mechanism is unknown at the moment, one solution is to simply add
        a $\mathcal{L}_\text{soft}$ component to the lagrangian that includes all allowed
        terms and corresponding new parameters.

        One of the main motivation for supersymmetry is to provide a mechanism that
        protects the mass of scalar bosons, thus answering the hierarchy problem. In the
        case of the Higgs boson, one can show
        that if the superpartners have equal masses, then the quadratic divergences
        introduced in \refequation{eq:higgsMassCorrectionFromFermion} cancel each other and
        it gets natural for the Higgs to have a mass much lower than the Planck scale.
        If SUSY is only softly broken, as for chiral and gauge symmetry, then the
        boson-fermion symmetry still prevents to have a quadratic divergence. Instead,
        we are left with a mild logarithmic divergence :
        \eq{higgsMassCorrectionInSUSY}
        {
            \Delta m_h^2 \, \propto \, (m_f^2 - m_b^2) \, \text{ln}\left(\frac{\Lambda}{m_b}\right).
        }
        But this alone does not solves everything. If one consider the most massive
        fermion, that is the top quark $t$, then the mass ratio with its superpartner the
        stop $\tilde{t}$, $r = m_{\tilde{t}} / m_t$ cannot be too large. For instance, if
        $r = 100$ then we are left with corrections $\Delta m_h = \orderOf{100 \cdot m_t}$
        which is still large considering that we seek $\orderOf{m_h} = \orderOf{m_t}$.
        Therefore, we would need to reintroduce a certain level of fine-tunning to the
        parameter of the model for the corrections to cancel, and we want to avoid that
        to keep the theory natural. Hence, the naturalness of SUSY is strongly related to
        the mass of the superpartners of heavy fermions such as the top and bottom quarks.

        \subsubsection{The MSSM}

        The most studied realization of supersymmetry is the minimal supersymmetric standard
        model, or MSSM, corresponding to adding the minimum number of fields to the Standard
        Model for it to become supersymmetric. \todo{ref to The Minimal Supersymmetric
        Standard Model / Jorge C. Romao, and  hep-ph/9612229 SUSY and Such / Dawson}
        Since the MSSM breaks SUSY with ad-hoc terms, it is to be seen as an effective
        theory up to $\Lambda_{\text{SUSY breaking}}$.

        At each fermion and gauge boson is therefore associated a superpartner that we may
        call sfermion and gaugino. It should be
        stressed once again that left-handed and right-handed fermions are different
        fermions. Hence, there are for example two selectrons, one associated to
        each chirality of the electron. These are usually noted $\tilde{e}_R$ and $\tilde{e}_L$
        to associate them easily with their fermion partner, even though they are spinless.
        Regarding the Higgs sector, it is necessary to add not only a superpartner called
        higgsinos, but a second Higgs field to ensure the concistency of the theory.

        As shown in the case of the Standard Model, the mass eigenstates are not
        necessarily the flavor eigenstates. This is still true in the MSSM, especially
        because of the SUSY breaking terms, and is an important consideration for the
        phenomenology. In particular, masses eigenstates are formed by the combinations of
        the Higgsinos and electroweak gauginos and are called charginos and neutralinos.
        For the fermions, mixing terms are authorized in $\mathcal{L}_soft$ between the
        left and right-handed superpartners and are especially important for the
        phenomenology of the third generation. The masses eigenstates can be determined
        by diagonalizing the mass matrices. The resulting particles spectrum is given on
        \reftab{tab:MSSMsuperpartners}.

        \begin{table}
            \centering
            \begin{tabular}{ccc}
                \textbf{Flavor eigenstates} & & \textbf{Mass eigenstates} \\
                                            & & \\
            \begin{tabular}{|ccc|}
                \hline
                1$^\text{st}$ gen. & 2$^\text{nd}$ gen. & 3$^\text{rd}$ gen. \\
                \hline
                \hline
                $\tilde{u}_L$        &   $\tilde{s}_L$      & $\tilde{t}_L$ \\
                $\tilde{u}_R$        &   $\tilde{s}_R$      & $\tilde{t}_R$ \\
                $\tilde{d}_L$        &   $\tilde{c}_L$      & $\tilde{b}_L$ \\
                $\tilde{d}_R$        &   $\tilde{c}_R$      & $\tilde{b}_R$ \\
                \hline
                \hline
                $\tilde{\nu}_e$      &   $\tilde{\nu}_\mu$  & $\tilde{\nu}_\tau$ \\
                $\tilde{e}_L$        &   $\tilde{\mu}_L$    & $\tilde{\tau}_L$   \\
                $\tilde{e}_R$        &   $\tilde{\mu}_R$    & $\tilde{\tau}_R$   \\
                \hline
            \end{tabular}
            &
            &
             \begin{tabular}{|ccc|}
                \hline
                1$^\text{st}$ gen. & 2$^\text{nd}$ gen. & 3$^\text{rd}$ gen. \\
                \hline
                \hline
                "       &  " & $\tilde{t}_1$ \\
                "       &  " & $\tilde{t}_2$ \\
                "       &  " & $\tilde{b}_1$ \\
                "       &  " & $\tilde{b}_2$ \\
                \hline
                \hline
                "       &  " & "                  \\
                "       &  " & $\tilde{\tau}_1$   \\
                "       &  " & $\tilde{\tau}_2$   \\
                \hline
            \end{tabular}
            \\
            & $\rightarrow$ & \\
            \begin{tabular}{|ccc|}
                \hline
                Higgs   & Electroweak         & Strong          \\
                sector  & sector              & sector          \\
                \hline
                \hline
                $\begin{matrix} \tilde{h}^0_u, \tilde{h}^+_u \\ \tilde{h}^0_d, \tilde{h}^-_d \end{matrix}$
                &
                $\begin{matrix} \tilde{B} \\ \tilde{W}_{1,2,3}\end{matrix}$
                &
                $\begin{matrix} \tilde{g}_{1..8} \end{matrix}$ \\
                \hline
            \end{tabular}
            & &
            \begin{tabular}{|cc|}
                \hline
                Charginos and   & Strong          \\
                neutralinos     & sector          \\
                \hline
                \hline
                $\begin{matrix} \tilde{\chi}^\pm_{1,2} \\ \tilde{\chi}^0_{1,2,3,4} \end{matrix}$
                &
                "
                \\
                \hline
            \end{tabular}
            \end{tabular}
            \caption{Superpartners of the fermions and bosons of the standard model,
            categorized according to flavor eigenstates (on the left) and masses eigenstates
            (on the right). A symbol " indicates that the mass eigenstate is identical to
            flavor eigenstate. \label{tab:MSSMsuperpartners}}
        \end{table}

        By design, the MSSM conservs a quantum number called R-parity defined as \todo{define J and B}:
        \eq{RparityDefinition}
        {
            R = (-1)^{2J + 3B + L}.
        }
        $R$ is 1 for the particles of the standard model and -1 for their superpartners.
        As it is conserved, this means that supersymmetric particles can only be
        pair-produced and only decay to an odd number of supersummetric particles. In
        particular, this means that the lightest supersymmetric particle is stable as
        its decay would violate the $R$-parity. The important consequence is that it provides
        a good candidate for dark matter, provided that this particle is neutral.

        \subsubsection{Phenomenology of the chargino, neutralino and stop sector}

        \todo{explicit link between the parameters and mixing matrix + detail on
        coupling stop-top-EW}

        In the $(\tilde{B}, \tilde{W}^3, \tilde{h}^0_1, \tilde{h}^0_2)$ basis,
        \eq{neutralinoMixingMatrix}
        {
            M_{\tilde{\chi}^0}
            =
            \begin{pmatrix}
                \tilde{m}_{\tilde{B}} & 0                     & - m_Z \,\text{cos}\, \beta \,\text{sin}\, \theta_W &   m_Z \,\text{sin}\, \beta \,\text{sin}\, \theta_W
                \\
                0                     & \tilde{m}_{\tilde{W}} &   m_Z \,\text{cos}\, \beta \,\text{cos}\, \theta_W & - m_Z \,\text{sin}\, \beta \,\text{cos}\, \theta_W
                \\
                - m_Z \,\text{cos}\, \beta \,\text{sin}\, \theta_W  &   m_Z \,\text{cos}\, \beta \,\text{sin}\, \theta_W & 0 & \mu
                \\
                  m_Z \,\text{sin}\, \beta \,\text{sin}\, \theta_W  & - m_Z \,\text{sin}\, \beta \,\text{cos}\, \theta_W & \mu & 0
            \end{pmatrix}
        }

        In the $(\tilde{W}^\pm, \tilde{h}^\pm)$ basis
        \eq{charginoMixingMatrix}
        {
            M_{\tilde{\chi}^\pm}
            =
            \begin{pmatrix}
                \tilde{m}_{\tilde{W}}
                &
                \sqrt{2} m_W \,\text{cos}\, \beta
                \\
                \sqrt{2} m_W \,\text{sin}\, \beta
                &
                \mu
            \end{pmatrix}
        }

        In the $(\tilde{t}_L, \tilde{t}_R)$ basis,
        \eq{stopMixingMatrix}
        {
            M_{\tilde{t}}
            =
            \begin{pmatrix}
                \tilde{m}^2_{\tilde{t}_L} + m^2_t + m^2_Z(\frac{1}{2} - \frac{2}{3} \,\text{sin}^2 \theta_W) \,\text{cos}\, 2\beta
                &
                m_t ( X_{\tilde{t}} + \mu \,\text{cot}\, \beta)
                \\
                m_t ( X_{\tilde{t}} + \mu \,\text{cot}\, \beta)
                &
                \tilde{m}^2_{\tilde{t}_R} + m^2_t + \frac{2}{3} m^2_Z \,\text{sin}^2 \theta_W \,\text{cos}\, 2 \beta
            \end{pmatrix}
        }
        with :
        \begin{itemize}
            \item $\tilde{m}_p$ the mass terms for $p$ coming from the soft-breaking SUSY terms
            \item $m_p$ the mass term for $p$ coming from the electroweak symmetry breaking
            \item $X_{\tilde{t}}$ characterizing the trilinear coupling between the top partners and the Higgs field
            \item $\text{cos}\,\theta_W \definedAs m_W / m_Z$ defines the weak mixing angle
            \item $\mu$ and $\text{tan}\, \beta \definedAs v_2 / v_1$ relates the Higgs sector
        \end{itemize}




%==============================================================
\setcounter{mtc}{4}
\chapter{The Compact Muon Solenoid experiment at the LHC}
\minitoc
\newpage
%==============================================================

    \todo{references !!}

    \todo{explicitly explain LHC incident / + CMS run I at 7/8 TeV, run II will be at 13
    TeV + talk about the Higgs discovery at some point}

    \todo{explicit link between track curvature and impact parameter}

    \section{The Large Hadron Collider}

    \subsection{Scientific context and challenges}

    There are several approaches one can take toward new physics. First, using already
    available measurement of observables and exploiting the fact described in \refsection{sec:standardModelSuccess}
    that observables are impacted through loops by the entire Nature's lagrangian,
    it's possible to infer information or limits regarding yet-unknown physics. Historically,
    this has been proved successful numerous times, for instance to predict the
    mass of the top quark from the $W$ boson mass. Similarly, a smoking-gun for some
    supersymmetric model is the enhancement of the branching ratio BR($B_s \rightarrow \mu^+\mu^-$).
    However, despite the fact this approach provides valuable information, it can
    be criticized as only providing indirect evidences and not a clear window to new phenomena.
    Moreover, the statistics and detector resolution required to increase the accuracy
    of the prediction grows exponentially.

    The most attractive and popular approach is the direct production and observation of
    new phenomena. One often forgotten fact is that new physics might not appear at high
    energies but at low energies instead. This is motivated for intance by by axions
    theories and some hidden sector theories which predict particles with sub-eV masses
    or rare interactions. Nevertheless it is more common to assume that new physics will
    show up at high energies. To probe higher energies, particle colliders have been
    built with increasing center of mass energies ($\sqrt{s}$) during the last centuries.
    Two famous ones are the Large Electron-Positron (LEP) collider and the Tevatron. The
    LEP was an $e^+e^-$ collider which operated at a maximum of $\sqrt{s} \approx 200\GeV$
    at CERN in the 90's and led to the discovery and study of the $Z$ and $W$ bosons. The
    Tevatron was a $p\bar{p}$ collider which operated at a maximum $\sqrt{s} \approx 2\TeV$
    at Fermilab until 2011. Nowadays, the only high energy collider in operation is the
    Large Hadron Collider (LHC) at CERN.

    \todo{maybe mention somehow that pp are better at probing new physics in the strong
    sector whereas lepton collider are better at probing new physics in electroweak sector}

    The LHC is a circular hadron collider built at CERN, near Geneva. One of its physics
    goal is the production high-energy proton-proton ($pp$) collisions to explore physics around
    $1\sim10 \TeV$ to test the validity of the Standard Model and possibly discover new
    physics. It is also capable of producing heavy ions (PbPb) collisions or proton-ion
    collision ($p$Pb) to study nuclear physics at high energy, in particular the properties
    of quark-gluon plasma. The rest of this document will focus on $pp$ collisions.

    The LHC collider has been built using the same tunnel as the LEP. The beam energy at
    the LEP was largely limited by the energy loss from synchrotron radiation. This loss
    is proportional to $E^4 / (m^{4} r^{2})$ with $E$ and $m$ the energy and mass of the
    particle, and $r$ the radius of the tunnel. As $m_p$ is much larger than $m_e$, it is
    easier to maintain the energy of a rotating proton beam than an electron/positron one.
    Consequently, using protons is preferred if the goal is to reach the highest possible
    energy, justifying this choice for the LHC. We will see in section
    \refsection{sec:physicsFromCollisionsAtTheLHC}, how proton collisions are however challenging
    from a reconstruction and analysis point of view.

    \subsection{Beam injection, control and acceleration}

    The creation and first acceleration of the proton beam work as follow. Protons are initially
    produced by heating low-pressure $H_2$ gas which produces $H^+ = p$ and $e^-$ and
    separating the protons using high voltages. The protons are fed to an acceleration
    chain made of consecutive accelerators which use radiofrequencies to create an acceleration
    gradient. At the end of the chain, in the Super Proton Synchrotron (SPS), the
    beam acquired an energy of $450\GeV$ before being injected in the LHC ring. The LHC
    ring is a tunnel of 27 kilometers built between $50$ and $200$ m underground. The
    project was originally designed to accelerate the proton beam up to $7\TeV$ per beam,
    corresponding to a center of mass energy $\sqrt{s} = 14\TeV$. Four experiments are
    installed along the ring, at each collision point. Two of them, ATLAS and CMS, are
    general-purpose experiments while ALICE focuses on heavy-ion collisions and LHCb studies
    $b$-hadrons properties. \reffig{fig:LHClayout} presents an overview and the general
    layout of the machine.

    \insertTwoFigures{LHClayout}
                     {LHC/LHCring}
                     {LHC/LHClayout}
                     {0.6}
                     {On the top, sketch of the underground installation showing the SPS
                     and the injection tunnels as well as the four experiments. On the bottom,
                     layout of the LHC ring. The installation is divided in eight octants,
                     with some of them dedicated to collisions but also injection, acceleration (RF)
                     and beam dump.}

    One of the biggest challenge of the LHC was the construction and operation of magnets
    powerful enough to bend the trajectory of the protons at such high energies. To do
    so, superconducting magnets made of a niobium-titanium alloy are used and are able to
    produce a magnetic field of up to $8.33$ T. The operating temperature of such magnets
    is only a few Kelvins : in the context of the LHC, they are required to be cooled using
    a complex system of superfluid helium-4 at 1.9K. It is remarkable to note that this
    temperature is actually cooler than the cosmic microwave background at $\approx2.7$ K.

    Operating superconducting magnets require a good control and understanding of quenches.
    A magnet quench occurs when a part of the superconducting material loose its superconducting
    property and starts to heat because of Joule effect. Because the critical temperature
    is usually low, this can create a chain reaction as the heat spread. Quenches are
    not such unusual events, and the LHC employs a sophisticated system to detect them and protect
    the magnetic system by automatically dumping the beam, mitigating the effect of the
    quench and extracting the energy stored inside the magnet.
    \todo{http://cerncourier.com/cws/article/cern/54383}. However, deffects in the protection
    system or more generally in the electrical and mechanical apparatus can have dramatic
    impact as the energy can easily damage the machine if released improperly. In 2008,
    a faulty electrical interconnection between two magnets led to an electrical arc that
    punctured the cryogenic fluid enclosure and vaporized a fraction of the fluid causing
    important damages to the installation.

    %\insertFigure{LHC/pipePhoto}{0.7}{\todo{caption} Credit : Maximilien Brice, CERN}
    \insertTwoFigures{LHCpipe}
                     {LHC/dipoleCrossSection2}
                     {LHC/pipePhoto2}
                     {0.7}
                     {On the top, cross-section of a cryodipole of the LHC
                     \todo{ref to LHC Machine / JINST 3 S08001 doi:10.1088/1748-0221/3/08/S08001},
                     showing the magnet coils around the two beam pipes, and the cooling system.
                     On the bottom, technician working on an interconnection between two LHC
                     sections (credit : Anna Pantelia, CERN).}

    The pipes of the LHC consist of about 1200 dipole sections represented on
    \reffig{fig:LHCpipe}, each $15$ m long, whose role are to bend the beam using dipole
    magnets. About 400 quadrupole magnets that are used to squeeze the
    beam. Higher multipolar magnets are also used to prevent and correct instabilities of
    the beam. The effective bending radius, created from the $17.6$ km of dipoles is
    $r \approx 2.8$ km, to be compared to the radius of the full machine being about $4.3$km.

    The beam is organized in bunches of protons separated by a time interval of 50 or
    25 ns depending of the machine configuration. Each bunch is made of about $10^{11}$
    protons. The LHC uses radiofrequency cavities to increase the energy of the beam by
    about 0.5 MeV per rotation, or $5.4\GeV$ per second. At $7\TeV$ per beam with 25 ns
    between bunches, the total energy of the beam is about 360 MJ and the energy stored
    in the magnet system is about 600 MJ, roughly corresponding to the energy of a
    lighting bolt. The beams are then made to collide at four different points along the
    ring, corresponding to the four main experiments.

        \subsection{Physics of $pp$ collisions \label{sec:physicsFromCollisionsAtTheLHC}}

    The number of $pp$ collisions happening per unit of time is characterized by the
    instanteneous luminosity $L$ which depends on the bunch crossing frequency, the
    number of bunches and proton per bunches and other beam and geometry parameters.
    The two general purpose experiments, ATLAS and CMS, aim for a peak luminosity
    $ L = 10^{34} \text{cm}^{-2} \text{s}^{-1} = 10 \text{nb}^{-1} / \text{s}$. For a
    given process, one can compute the number of expected events.

    $$ n_\text{event} = L \times \sigma_\text{process} $$

    where $\sigma_\text{process}$ is the cross-section of the process and $n$ a number
    of event per unit of time. Later, we may refer to the integrated luminosity over all
    data-taking which relates to $N$, the total number of events produced.

    \reffig{fig:LHC/ppCrossSections} shows the evolution of the cross section of different
    processes such as $W$ or $Z$ production. The total inelastic cross section is about
    60 mb. At peak luminosity and with a bunch crossing every 25 ns, this corresponds
    to 15 inelastic interactions per bunch crossing on average. This is what is behind
    the notion of pile-up : one event, or collision, contains $\orderOf{10-30}$
    interactions with generally only one is of interest, such as the production of a $W$
    or $Z$ boson.

    \insertFigure{LHC/ppCrossSections}{0.5}{Evolution of the cross sections of typical
    processes for $p\bar{p}$/$pp$ collisions as function of $\sqrt{s}$. Two vertical
    lines corresponds to the center of mass energy of the Tevatron ($\sqrt{s} \approx 2\TeV$)
    and the LHC ($\sqrt{s} = 14\TeV$) \todo{ref to  arXiv:0812.2341}}

    This is a crucial point in the context of $pp$ collisions, as inelastic hadron collisions
    naturally produce jets arising from the parton showering and hadronization of gluons
    and quarks. Therefore, a lot of particles polute the environment of each event, affecting the
    number of impact in the tracking detector and energy measurement in calorimeters.
    It is thus needed to develop methods to mitigate the effect of pile-up.

    Furthermore, when considering one process of interest, it must be kept in mind that
    protons are composite objects made of partons, i.e. quarks and gluons, among which three
    valence quarks. At a given moment, a parton only carries a fraction of the total
    proton momentum. This is described by the parton distribution functions (PDF), characterizing
    the probably to find a parton with a fraction of impulsion $x$, depending of its
    nature. This implies that the center of mass energy of the two incoming partons is in
    fact always lower than the beam energy. Each incoming parton may also radiate partons
    right before the interaction, a processus called initial state radiation, leading to
    additional jets in the event. Finally, the remaining partons may interact with each
    other in what is called the underlying event, and produce high-energy jets, almost
    collinear to the beam axis.  $pp$ collisions
    are therefore complex processes, as summarized in \reffig{fig:LHC/ppCollision}, due
    to the composite nature of the proton and the fact that the parton are colored objects.

    \insertFigure{partonDistributionFunctionsExample}{0.7}{Example of PDFs at different
    for two energies $Q$ provided by the MSTW group \todo{confirm "group" ?}. At lower energies,
    one clearly distinguish the large fraction of momentum $x$ carried by $u$ and $d$ valence
    quarks while other species are only present in the parton sea and less probable as
    their mass increase. As $Q^2$ grows, the probability to probe sea partons increases.}

    \insertFigure{LHC/ppCollision}{0.65}{Feynman diagram showing the complexity of a
    single proton-proton collision. Two partons from the incoming protons partipate
    to the hard scattering leading to a $\gamma^*/Z$ resonnance which decays into two
    quarks. In both the initial and final states, the coloured particles may radiate gluons
    (or a gluon may split into a quark pair), leading to initial and final state radiations.}

    % Estimate of price/event ?

    \section{The Compact Muon Solenoid experiment}

        The CMS experiment is one of the four experiments installed at the collision
        points of the LHC. It is a general purpose experiment, though mostly dedicated
        to the study of $pp$ collisions, in particular to test the validity of the Standard
        Model, to understand the electroweak symmetry breaking, and to search for new physics. From this
        perspective, CMS is in scientific cooperation and competition with ATLAS, the
        other general purpose experiment. In the context of the search for new physics,
        having two experiments independently studying the same phenomenons is both
        stimulating but also crucial to have robust claims. It can indeed be
        argued that having two experiments both seeing a $3\sigma$ discrepancy is more
        convincing than one experiment seeing a $5\sigma$ discrepancy with no possibility
        of cross-checkign the result.

        In this section, the CMS detector will first be introduced generally before inspecting
        each subsystems. Then, the trigger system that aim to select which collisions to
        record will be presented. Finally, it will be explained how the physics objects
        can be reconstructed from the information recorded by the detector.

        \subsection{The CMS detector}

            \subsubsection{Physics motivations and detector overview}

        The CMS detector is an instrument that answer the fantastic challenges posed by
        the LHC machine while providing good measurement for physics purpose.

        The subsystems of the detector must indeed have a time response lower than 25 ns,
        corresponding to the design bunch crossing frequency, and to be synchronized with
        each other. The detector must also be able to disentangle between the
        $\orderOf{10-30}$ collisions occuring at each bunch-crossing by being able to
        reconstruct the collision vertices with good precision. Last but not least,
        the quantity of data per event is such that all events cannot be recorded. The
        detector must then decide in real time which events to keep, and which to reject
        from a fast analysis of each events.

        Among the many physics signature CMS aims to look at, the discovery and study of a
        light ($\orderOf{125 \GeV}$) Higgs boson in the channels $h\rightarrow\gamma\gamma$
        and $h\rightarrow ZZ \rightarrow 4\ell$ requires good measurements of the photon,
        electron and muon energy-momentum. Furthemore, from the perspective of final
        states involving dark matter candidates, one needs to obtain a good measurement
        of the energy escaping detection. This requires a good and an as hermetic as
        possible measurement of the repartition of the hadronic and electromagnetic energy
        in each event.

        \insertFigure{CMS/detectorOverview}{0.9}{Overview of the CMS detector.}

        The CMS detector is a hermetic, cylinder-shaped detector consisting of several
        complementary layers\footnote{while this structure is found in three of the four
        main experiments at the LHC, other structures exists in particle physics detector,
        for instance for the LHCb experiment or the AMS detector, composed of successive
        layers.} as represented on \reffig{fig:CMS/detectorOverview}.
        CMS design's is centered around a superconducting magnet delivering a field of 3.8~T
        using the same technology than the LHC, a nobium-titanium alloy. Its role is to
        bend the trajectory of charged particles, directly impacting the accuracy of
        their momentum measurement. Near the interaction point is found the tracker system, dedicated to
        the detection of impacts left by charged particles. It is composed of a pixel detector
        layers and microstrip layers. Around the tracker is the electromagnetic
        calorimeter, whose aim is to measure the energy of photons and electrons using
        scintillation crystals. The next layer, the hadronic calorimeter, aims to measure
        the energy carried by hadrons produced in the collisions. It is a scintillator made
        of brass and plastic sandwich. Finally, the muon system uses three different
        technologies to record the impact of muons. \reffig{fig:CMS/detectorPhoto}
        shows a photography of the actual detector during its assembly.

        \insertFigure{CMS/detectorPhoto}{0.75}{Photography of the CMS detector, showing some
        of the muon system, the magnet and the return yoke, and the HCAL barrel during
        assembly.}

        The coordinate system $(x,y,z)$ of CMS, represented on \reffig{fig:CMS/coordinateSystem},
        is defined such that $x$ is directed to the center of the LHC, $y$ points toward
        the sky and $z$ is colinear to the beam axis. As the detector exhibits a
        cylindrical symmetry, it is convenient to work with the azimuthal angle $\phi$
        between the momentum $\vec{p}$ of a particle and the $x$ axis in the transverse
        plane. The projection of $\vec{p}$ in the transverse plane is called $\vec{p}_T$.
        Additionnaly, the pseudo-rapidinity $\eta$ is defined by :

        $$ \eta \definedAs - \text{ln}(\text{tan}\frac{\theta}{2}) $$

        where $\theta$ is the polar angle. $\eta$ is equal to 0 for particles produced in
        the transverse plane, about 0.88 for $\theta = \pi/4$ and tends to $\infty$ in the
        limit where the particle is produced along the beam axis. More qualitatively, we
        refer to low $\eta$ values as the central or barrel region, while high $\eta$ values
        are referred to as forward or endcap region. In the plane ($\phi$,$\eta$), we
        define the distance between two directions as $\Delta R \definedAs \sqrt{\Delta
        \phi^2 + \Delta \eta^2}$.

        \insertFigure{CMS/coordinateSystem}{0.7}{Coordinate system of CMS represented in
        the longitudinal plane (on the left) and in the transverse plane (on the right).}

            \subsubsection{Tracker system}

        The tracker system is the piece of detector closest to the interact point. Its goal
        is to reconstruct the trajectory left by charged particles to estimate its diretion
        and momentum, and to identify not only the primary interaction vertices, but also
        secondary vertices from the hadronization of $b$-quarks. This requires a precision
        on the vertex to less than one milimeter.

        While achieving this level of accuracy, the instrument must be able to handle the conditions of
        the LHC. At nominal conditions, it can be estimated that about 1000 charged particles
        are produced for every bunch crossing each 25~ns. This roughly corresponds to a
        hit rate of 1~MHz/mm$^2$ at 4~cm of the interaction point and 3~kHZ/mm$^2$ at 115~cm.
        To not be overwhelmed (i.e. have a low occupancy), the detector must therefore
        have a high granularity. Moreover, the electronics should resist and be reliable with
        respect to the high radiation environment and the magnetic field. Last but not
        least, the overall quantity of material involved should be as small as possible to
        not alterate the trajectory of the particle. The choosen technology is a fully
        silicon-based detector as it is at the same time compact and accurate. However,
        for optimum performances, silicon must be cooled to $-10\sim15^oC$.


        \begin{figure}[h!]
            \centering
            \includegraphics[width=0.55\textwidth]{CMS/tracker}
            \includegraphics[width=0.44\textwidth]{CMS/tracker3D}
            \caption{Layout of the tracker system.
            On the left, red lines represent the pixel, blue lines represent double-sided
            strips module and black lines represent single-sided strips modules.}
            \label{fig:CMS/tracker}
        \end{figure}



        The tracker system is made of several sensors arranged in layers, in turn arranged
        in modules, as represented in \reffig{fig:CMS/tracker}. At the center, the pixel module
        uses silicon pixel sensors of $100 \times 150~\mu$m$^2$. The barrel of this module
        is composed of three 53 cm layers in the region $r < 10$ cm, supplemented by two
        disk layers placed at $\left|z\right| \approx$ 35 and 45~cm to cover the forward
        region. In total, the pixel detector contains 66 million pixels covering about 1~m$^2$.

        The rest of the tracker system is bassed on silicon microstrips with sizes ranging
        from $10~\text{cm} \times 80~\mu\text{m}^2$ to $25~\text{cm} \times 180~\mu\text{m}^2$.
        It is divided in two barrels, the tracker inner barrel (TIB) and outer barrel (TOB),
        supplemented by two endcap modules, the tracker inner disks (TID) and tracker endcaps
        (TEC). The strip tracker contains 9.3 million strips and cover almost 200 m$^2$ of
        surface. Overall, the system has a length of 5.6~m and a radius of 1.1~m and covers
        up to $\abseta \approx 2.5$.

            \subsubsection{Electromagnetic calorimeter}

        The electromagnetic calorimeter role is to measure the energy of incoming electrons
        and photon. The specifications of this subdetector are particularly oriented for
        the search of the $h \rightarrow \gamma \gamma$ channel. The resolution on the
        invariant mass of the diphoton system depends directly on the energy and angular
        resolution of the photons. This part of the detector is of course also crucial for
        all processes involving electrons, such as $Z$ and $W$.

        The technology adopted for the ECAL is lead tungstate crystals (PbWO$_4$). Lead
        tungstate is a very dense scintillation material, about 8.3~g/cm$^3$, with a short
        radiation length 0.89~cm, making it an appropriate choice for a compact calorimeter.
        On the other hand, the light output is relatively low, about 4.5 photoelectrons
        per MeV, and thus requires the use of photomultipliers to improve the accuracy.
        The calorimeter uses crystals with a size of $2.2\times2.2~\text{cm}^2$
        for the front face and 23~cm in length, as represented on \reffig{fig:CMS/ECALcrystal}.
        In terms of $\eta-\phi$, each crystal covers a region of approximately $0.0174
        \times 0.0174$.

        \insertFigure{CMS/ECALcrystal}{0.65}{PbWO$_4$ crystal used in the ECAL with the photodiode
        glued to the back of the crystal.}

        In the endcaps, to help discriminate between prompt photons and $\pi^0$ decaying
        to two directionnaly close photons, two preshower disks are placed in front of the
        ECAL. Their role is to initiate electromagnetic shower and provide a finer granularity
        with silicon strips 0.2~cm wide to be compared to the $\sim2\times2~cm^2$ faces
        of the crystals.

        \insertFigure{CMS/ECAL}{0.6}{Layout of the electromagnetic calorimeter. Each of the
        blue segments represent a PbWO$_4$ crystal.}

        The global layout of the ECAL is presented on \reffig{fig:CMS/ECAL}. The ECAL
        barrel (EB) extends to $\abseta = 1.479$, supplemented by the ECAL endcap (EC)
        up to $\abseta = 3.0$. The preshower disks cover the region $\abseta \in [1.653,2.6]$.
        Overall, the system is 7.8m long and lie within $1.2 < r < 1.8$~m.
        The optical properties of the crystal is a crucial parameter that depends on the
        temperature and radiations received. To obtain accurate measurements, a cooling
        system ensure that the crystal temperature is stable at $18\pm0.05^oC$ and the
        transparency is monitored in real time via a laser system.

        The resolution measured using electrons is parametrized using the formula

        \eq{ECALresolution}
        {
            \left( \frac{\sigma_E}{E} \right)^2
            =
            \left( \frac{S}{\sqrt{E}} \right)^2
            \oplus
            \left( \frac{N}{E} \right)^2
            \oplus
            C^2
        }

        with the different parameters

        \eqalign{ECALresolution2}
        {
            S & = 0.028~\text{GeV}^{1/2} \text{, the stochastic contribution,}\nonumber\\
            N & = 0.12~\text{GeV}\text{, the noise contribution,}\\
            C & = 0.003\text{, the constant contribution.}\nonumber
        }

            \subsubsection{Hadronic calorimeter}

        The hadronic calorimeter (HCAL) role is to measure the energy of incoming charged
        and neutral hadronic particles, which will be in turn a crucial information to
        reconstruct jets and missing transverse energy.

        The main technology used for the HCAL consist of layers of dense absorbers and
        scintillation tiles. The absorbers are made of steel and brass, with which
        the incoming hadrons interact to develop a shower. As the particules of the shower
        travel through the calorimeter, they encouter scintillator layers in which they
        emit lights. The light is gathered by fibers inside the tiles, which is then
        linked to readout electronics.

        \insertFigure{CMS/HCAL}{0.6}{Layout of the hadronic calorimeter.}

        The layout of the HCAL is presented on \reffig{fig:CMS/HCAL}. The HCAL barrel
        (HB) extends to $\abseta \approx 1.4$ while the HCAL endcap (HE) extends up to
        $\abseta \approx 3.0$. The segmentation term of $\eta,\phi$ is about $0.087 \times 0.087$
        in the barrel which is 25 times coarser than ECAL. A last layer of the HCAL called
        the HCAL outer (HO) uses the magnet coil as an absorption layer. HO is exclusively
        made of scintillation tiles and complements the measurement of the HB. Finally, to
        measure hadrons produce in the high-forward region, two calorimeters are located at
        $z \approx 11.2$~m to cover $\abseta$ up to 5.2. As they receive a high flux of
        particles coming from underlying events, the material for these calorimeters must
        therefore be more resistant to radiation and use hard quartz fiber instead of
        plastic ones.

        The resolution measured on pions is
        \eq{HCALresolution}
        {
            \left( \frac{\sigma_E}{E} \right)^2
            =
            \left( \frac{S}{\sqrt{E}} \right)^2
            \oplus
            C^2
        }

        with

        \eqalign{HCALresolution2}
        {
            S & = 0.084~\text{GeV}^{1/2} \text{, the stochastic contribution,}\\
            C & = 0.074\text{, the constant contribution.}\nonumber
        }

        \todo{note about neutral hadrons being likely to escape detection, though neutral fraction is only about 10\%}

            \subsubsection{Muon system}

        The muon system is composed of several subsystems placed in the return yoke of
        the superconducting magnet as represented in \reffig{fig:CMS/muon}. It aims to
        measure accurately the trajectory of muons.

        \insertFigure{CMS/muon}{0.6}{Layout of the muon system.}

        Three different gas-based technologies are used, whose choice was driven by the
        large surface to be covered and the differents radiation environment :
        \begin{itemize}
            \item \textbf{Drift tubes} (DT) - This technology uses streched wire within
                a gas volume. Charged particles travelling through the gas ionize atoms,
                leaving a cloud of electron along the track. These electrons then drift
                in the tube and are collected by the positively-charged wire. By knowing
                the point where the electron where collected along the wire, and the
                time between the hit and collection, it is possible to infer the position
                of the impact in the drift tube.
            \item \textbf{Cathode strips chambers} (CSC) - Cathode strips chambers are
                made of array of anode wires crossing cathode strips within a gas volume.
                A charged particle travelling through the gas ionize atoms and provokes an
                avalanche of electrons. The electrons and ions move to the anode and
                cathode respectively, producing a signal in both of them which leads to
                a bidimensionnal information.
            \item \textbf{Resistive plate chambers} (RPC) - Resistive plate chambers are
                made of two plates of high-resistivity platic acting as cathode and anode
                separated by a gas volume. After ionization of the gas causing an
                electron avalanche, the chages travel through the plastic plates to be
                collected by strips situated behind them.
        \end{itemize}

        The DT technology is used in the barrel, up to $z \approx 6.5~\text{m}$, covering
        pseudo-rapidities up to $z = 1.2$, while the CSC technology is used for the endcap,
        where the background rate is large and the magnetic field is also large and non-uniform,
        between $z \approx 6~\text{m}$ and $z \approx 10.5~\text{m}$ covering up to $\abseta = 2.4$.
        The RPC complements the previous technologies for $\abseta$ up to 1.6, providing a
        coarser position resolution but a faster response and better time resolution.

        When combining the information of the muon system with the tracker, one can
        obtain a relative resolution on the $p_T$ typically of $1\%$, and a relative
        resolution on $\eta$ and $\phi$ typically of the order of $10^{-5}$ and $10^{-4}$
        respectively.

        \subsection{Trigger system}

        The CMS detector must sustain a rythm of one bunch crossing each 50 or 25 ns, depending
        of the condition of the LHC machine. The raw output of the detector is therefore
        about $40~$MHz. It is unrealistic to consider storing and reconstructing all these events
        considering of the transfer link, memory and computing power it would require.
        Therefore, a selection must be applied online to reduce the rate to a more realistic
        one, set to about $400~$Hz.

        The system in charge of this task is called the trigger. The reduction of the rate
        is done by performing a minimalist reconstruction of objects in the event
        and applying thresholds to decide quickly whether the event might be relevant for
        physics analysis, typically because it contains high-$\pT$ objects. The design of
        the thresholds must be done carefully to have an adequate balance between reduction
        of the rate and trigger efficiency of the relevant physics processes. Optimization
        of the execution time of the software as well as the accuracy of the minimalist
        reconstruction is in this context crucial to efficiently use the available ressources.

        The trigger system of CMS is composed of two levels. The level 1 (L1) trigger first
        reduce the rate from $40~$MHz to $100~$kHz, followed by the high-level trigger (HLT)
        which further reduces the rate to $400~$Hz.

        \insertFigure{CMS/triggerL1}{0.8}{Architecture of the L1 trigger.}

        The L1 trigger is implemented is itself composed of different stages as
        represented on \reffig{fig:CMS/triggerL1}. It only takes information from
        the calorimeters and muon system as input. The first stages, the regional calorimeter
        and muon triggers, are at the closest of the subsystems, identifying significant
        energy deposits or muon impacts in each region of the detector independently.
        Regional triggers produce a list of muons, $e/\gamma$ candidates and local energy
        deposits. This information is fed to the global calorimeter and muon triggers which
        combine it and remove redundant information. The final global trigger implements
        the trigger menu, that is the set of requirements applied on the final list of
        objects. L1 triggers range from simple single objects with a $\pT$ threshold, to
        selection requiring the coincidence of several objects with topological conditions
        on them. Generally speaking, the total rate of a given signature decreases as the
        variety and number of objects it contains increases. It is therefore possible to
        put lower threshold on more complex selection compared to simpler ones. For instance,
        the $\pT$ thresholds are lower for a $e+\mu$ trigger compared to a dimuon trigger,
        itself having lower thresholds than a single muon trigger.

        If the events pass any of the selection in the L1 menu, it is handled to the HLT.
        The HLT selection consists of a much larger list of path compared to L1.
        It is performed on a computer farm able to perform a full reconstruction
        of the objects present in the event, including the information from the tracker.
        However to optimize the processing time for each event, the reconstruction is
        performed as much as possible in steps of increasing complexity. For instance,
        selection steps involving only calorimeters and muon detectors are performed before
        any tracking reconstruction is required, as it is a CPU intensive step. Once
        an event is selected, it is stored to be used in offline physics analysis.

        \subsection{Object and events reconstruction}

        After recording an event with the detector, the outputs of each subdetectors
        are to be analyzed to identify the nature and properties of particles produced in the
        event. Ultimately, one want to reduce step by step the complexity of an event to
        something physically understandable. Here, we briefly present some of the
        cornerstones of the CMS reconstruction. First, the vertexing and tracking step
        corresponds to the identification of tracks left by charged particles in the tracker
        and the collision vertices. This step is a crucial point for the second part, the
        particle flow algorithm which combines measurement from the tracker, muon systems
        and calorimeters to produce a collection of reconstructed particles. In a third step,
        jets of particles can be constructed using clustering algorithms to reach the original
        parton impulsion and direction, and a general impulsion imbalance can be computed
        to sign the production of particles that escapted detections.

            \subsubsection{Tracking and vertexing \label{sec:trackingAndVertexing}}

        The challenge to which the tracking step must responds is to find and reconstruct
        tracks left by the thousands particles produced in each bunch crossing, among the
        many hits in each layers. If designed badly, this step can lead to increased
        computing time due to the number of possible hits combination one has to try,
        or a high fake rate. The tracking algorithm must therefore be efficient in time,
        but also have a good reconstruction efficiency and a low fake rate.

        CMS tracking algorithm starts by constructing seeds from pair or triplets of hits
        in the pixel detector, compatible with the beamspot. Then, assuming that these
        hits are coming from a charged particle, the trajectory is extrapolated to the
        next layers. If a hit is found compatible with the extrapolation, it is used to
        update the track parameters estimation. Finally, once some criterias are reached,
        the algorithm stops and a final fit is performed to determine the track parameters.
        Once this first step is done, an iterative procedure is performed by removing the
        hits associated to reconstructed tracks, and repeating the track reconstruction
        from new seeds with looser criterias.
        The track reconstruction efficiency for muons is found to be very good as it is
        higher than 99\% up to $\abseta \sim 2.4$, whereas for pions the efficiency varies
        between 85 and 95\% depending on $\pT$ and $\eta$.

        Reconstructed tracks are used to identify collision vertices, or primary vertices
        (as opposed to secondary vertices which occur in the hadronization of $b$ quarks).
        Only good quality tracks are used in the process, which is defined using their
        compatibility with the beam spot, the number of hits and their fit quality. A
        metric is defined from the closest approach of the track to the $z$-axis and
        the associated uncertainty coming from the track measurement. Then, a deterministic
        annealing algorithm is used to cluster the tracks into vertices. Compared to the
        traditional jet algorithm that will be discussed later, this algorithm is a divisive
        clustering algorithm : at first, all tracks are put in a single group, which is then
        iteratively divided in smaller groups until a given condition is met. Once it is
        done, the tracks of each group are used to fit the position of the initial vertex.
        The resolution on the vertex position in $(x,y,z)$ depends on the number of tracks
        associated to the vertex, being about 0.1~mm when the vertex has only 5 tracks, and
        down to 10-20~$\mu$m when it has more than 40 tracks. \reffig{fig:CMS/vertexing}
        illustrates the vertexing capabilities in an event with 78 vertices reconstructed.
        Finally, the primary collision vertex is defined as the vertex with highest
        $\sum_\text{tracks} ({\pT}^2)$.

        \insertFigure{CMS/vertexing}{0.7}{Zoom on the beamspot region of an impressing event
        containing 78 reconstructed vertices.}

            \subsubsection{Particle flow algorithm}

        \todo{ref to PFT-09-001}

        \todo{illustration from https://cds.cern.ch/record/1627818/files/CR2013\_325.pdf }

        The particle flow algorithm's role is to efficiently combine the output of each
        subdetectors to reconstruct each single particles according to their nature. Thanks
        to a development before the data taking, it has quickly become the most commong
        method of reconstruction used to physics analysis such as cross-section measurements
        and searches for new physics. Particle flow is especially relevant in the context
        of jet clustering that will be presented in the next section : he tracker provides
        key information for low $\pT$ charged particles and their direction directly at the
        vertex, whereas the calorimeters are better suited for high $\pT$ regime and neutral
        particles.

        To combine the outputs of the subdetectors, links are created between elements
        to create blocks. Given the granularity of the detector, blocks are typically made
        of one, two or three elements. For instance, reconstructed charged-particle tracks
        are used to extrapolate the trajectory of the particle to the ECAL and HCAL layers.
        If energy deposits are found nearby the predicted impact using the $\Delta R$
        metric, the track is linked to the energy deposits. Similarly, tracks in the
        tracker can be linked to tracks in the muon system under the condition that the
        global fit of the two tracks is good enough, using a $\chi^2$ as metric.

        The reconstruction of electron is more challenging as they are significantly
        affected by Bremsstrahlung : the tracker acts for them as a preshower in which
        they are likely to radiate photons when travelling through the layers.
        In an attempt to collect and link those photons to the charged-particle track,
        tangents to a track at each layer are extrapolated to the ECAL and compatible
        energy deposits are linked to the track.

        After the link step, the identification step is performed from the least ambiguous
        object to the most ambiguous object. Each time an object is identified, it is
        removed from the algorithm before starting the next identification step.

        The least ambiguous objects, the muons, are identified from blocks made of compatible
        tracks in the tracker and the muon system. Then, electrons are considered from
        block made of tracks and energy deposits in the ECAL. The exact treatment for
        electron is more careful than for other tracks, as the fit of the track must take
        into account the successive loss of energy due to Bremsstrahlung in the tracker
        layers. Therefore, the tracks are refitted using a Gaussian-Sum filter (GSF)
        \todo{ref here}. The final identification involve criteria put on the tracking,
        calorimetric and compatibility variables.

        After the charged leptons identification, criterias are applied on the remaining
        tracks and those are considered to be charged hadron candidates. Their energy
        and momentum is computed either from the track alone assuming a charged pion
        hypothesis, or combining the track and calorimeter information which is relevant
        in particular at high $\pT$ or large $\eta$ where the tracker resolution is
        degraded. Finally, after substracting the energy deposits from charged particles
        in the ECAL and HCAL, the remaining significant energy deposits give rise to photon
        candidates and neutral hadron candidates.

        \todo{discuss performance}

            \subsubsection{Jet reconstruction}

        The concept of jets refer to collimated bunches of stable hadrons coming from partons
        (quarks or gluons) after they fragment and hadronize. Reconstructing jets can
        therefore yields information regarding quarks or gluons produced in the events.
        However, it can not be stressed enough that there is no direct equivalence between
        a jet and a parton : a jet is fundamentally an ambigous object as one parton
        shower can overlap with another, and their very definition is given by the choice
        of algorithm to reconstruct them. Moreover, a jet can be contaminated by other
        hadronic activity in the event, such as pile-up, which can degrade the energy
        and angular resolution. Jet physics is therefore a very rich topic and a field
        in itself, which goes from reconstruction to calibration and study of the jet
        substructure.

        Two main category of algorithm exist to build jets : cone-based algorithm and
        incremental clustering algorithm. CMS mainly use two incremental clustering
        algorithm. The starting point is a definition of the metric between objects in the
        event, which may be for instance the energy deposits in the calorimeters or the
        particle flow collection. The main algorithm is called anti-$k_T$ and uses a metric
        defined by
        \eq{jetClusteringMetric}
        {
            d_{ij} \definedAs \text{min}(p^{2p}_{T,i},p^{2p}_{T,j}) \frac{\Delta R^2}{r^2}
        }
        where $p = -1$ for anti-$k_T$, $\pT$ is the value of the transverse momentum of
        the ojects, and $r$ is a size parameter, analogous to the size of a cone and
        typically equal to 0.5 for CMS during Run I. The algorithm looks for the smallest
        $d_{ij}$ in the event and the corresponding objects are merged together into a
        jet. The algorithm stops when all the $\Delta R$ are higher than $r$. In the
        simple case with one high-$\pT$ object and several softer objects around, the
        hard one will agregate all the softer ones and the result is a cone-like jet. When
        there are two or more high-$\pT$ objects in the event, they either get merged into
        a single jet ($\Delta R < r$) with a cone-like shape around their barycenter, or
        lead to two separated jets ($\Delta R > r$) where the dominant object will first
        aggregate the softer objects. An example of jet reconstructed with this algorithm
        is shown in \reffig{fig:CMS/jetExample}.

        The second algorithm, called Cambridge/Aachen algorithm, uses the metric from
        \refequation{eq:jetClusteringMetric} but with $p = 0$, i.e. not relying on the $\pT$ of
        the objects but only on their angular proximity. The shape of the jets obtained with
        this algorithm are less likely to be cone-like. Nevertheless, it has been found to
        yield better results for substructure-oriented studies of jets.

        \reffig{fig:jetEfficiencies} presents a comparison of the performances obtained
        for particle-flow jets and calorimeter-based jets. For a given initial parton, the
        probability to succesfully reconstruct a jet in $\Delta R < 0.1$ reach 90\% for
        a parton of $30\GeV$ for particle-flow jets compared to $80\GeV$ for
        calorimeter-based jets. In terms of $\pT$ resolution, the reconstructed $\pT$ is
        for the particle-flow jets roughly within $90\% \pm 15\%$ of the generated $\pT$
        when in the range $40-60\GeV$, whereas it is within $50\% \pm 15\%$ for
        calorimeted-based jets. This bias in the reconstructed transverse momentum, called
        jet energy scale, is typically due to thresholds and inefficiencies in the calorimeter
        and can be measured and corrected after the jet reconstruction. This correction is
        typically parametrized as function of $\pT$ and $\eta$.

        \insertFigure{CMS/jetExample}{0.6}{Representation of a $115\GeV$ jet reconstructed in CMS
        with the anti-$k_t$ algorithm, showing the reconstructed tracks left by charge
        particles in plain blue lines, and the trajectory of neutral particles in dotted
        lines. The electromagnetic and hadronic deposits in calorimeters are represented
        in lego plots in yellowish and teal respectively. One should note how the curved
        trajectory of some charged particles make it so that they actually leave the
        pseudo-cone when flying out of the vertex.
        \todo{ref ? http://www.quantumdiaries.org/2011/06/01/anatomy-of-a-jet-in-cms/}}

        \insertTwoFigures{jetEfficiencies}
        {CMS/jetMatchingEfficiency}
        {CMS/jetResolution}{0.49}
        {Comparison of some performances for the jet reconstruction using particle flow as
        input compared to a calorimeter-based approach. On the left, jet matching efficiency
        as function of the generated $\pT$ in the central region $\abseta < 1.5$, effectively
        corresponding to the probability of succesfully reconstructing a jet within
        $\Delta R < 0.1$ of the generated parton. On the right, relative difference between
        the reconstructed and generated $\pT$ for initial partons with $40 < \pT < 60\GeV$.}

        \todo{composition of jet: https://cds.cern.ch/record/1627818/files/CR2013\_325.pdf
        at 100 GeV, 65\% charged hadr, 25\% phton and 10\% neutral hadron }

            \subsubsection{Missing transverse energy}

        So far it has been shown how to reconstruct charged leptons and photons and get
        information regarding quarks and gluons produced in the event. The remaining
        type of stable particles one expects from the Standard Model is neutrinos, which
        have very low rate of interaction with matter and therefore escape detection. Some
        theories beyond the Standard Model also include stable particles escaping detection
        such as neutralinos in R-parity conserving SUSY.

        It is possible to infer information about the production
        of such particles via the energy-momentum conservation in the transverse plane\footnote{The
        longitudinal plane can however not be used, since the two incoming partons carry
        different momentums.}. A significant imbalance may reveal the momentum carried by
        the sum of invisible particles. We define the missing transverse energy as :
        \eq{ETmissDefinition}
        {
            \MET \definedAs - \sum_{\text{objects}} \vec{\pT}.
        }
        In this formula, one must define what are the objects to consider and this choice
        leads to several possible reconstruction, such as calorimeter-based, track-based
        or PF-based $\MET$. Several effects may lead to articial $\MET$ in addition to the
        actual contribution of invisible particles, which can be classified in two categories :
        intrinsec condition and resolution of the detector, for instance the jet energy
        resolution ; and misreconstruction of the events or mismeasurement in the
        operation of the detector, for instance in the operating of the laser correction
        of the ECAL. One of the challenge of $\MET$ reconstruction is therefore to identify
        and understand the different sources of unphysical missing energy as well as bias and
        smearing, and either correct them to improve the resolution or veto the corresponding
        events
        .
        \reffig{fig:METspectrum}, on the left, shows how sources of unphysical $\MET$
        lead to an unexpected large tail in the reconstructed distribution for the data.
        After identifiying several causes and filtering out events likely to have unphysical
        $\MET$, the agreement between data and simulation is restored.

        \begin{figure}[h!]
            \centering
            \includegraphics[width=0.53\textwidth]{CMS/MET/cleaning}
            \includegraphics[width=0.45\textwidth]{CMS/MET/spectrumInZmumu}
            \caption{On the left, comparison of data with simulation of the
            particle-flow based $\MET$ distribution, before and after cleaning of the events
            likely to contain unphysical $\MET$. On the right, distribution of $\MET$ in
            a $Z \rightarrow \mu\mu$ enriched sample.}
            \label{fig:METspectrum}
        \end{figure}

        Events with no genuine $\MET$ provides a mean to study the resolution.
        $Z \rightarrow \mu\mu$ events fit well this purpose as muons are well-identified objects
        with a good resolution and a constrain can be put on the invariant mass of the
        dimuon system to obtain a sample with good purity. \reffig{fig:METspectrum},
        on the right, shows the distribution of the reconstructed particle-flow based
        $\MET$ in such events. The maximum of the distribution is around $10-20\GeV$ and
        drops by three orders of magnitude up to $70-80\GeV$.

        The resolution can be studied as function of $q_T$, the recoil of the $Z \rightarrow \mu\mu$
        system due to initial state radiations, and also as function of the pile-up in the
        event, as presented on \reffig{fig:METresolution}. At low $q_T$, the ISR is
        essentially too soft to be measured accurately, and therefore $\vec{u}_{||}
        \definedAs \vec{\MET} - (\vec{\mu}_1 + \vec{\mu}_2)$ is quite different from $\vec{q}_T$.
        At $q_T \sim 40\GeV$, however, the measurement of the ISR gets better and $u_{||}$
        is within $\sim 3\%$ of $q_T$. The resolution of $\vec{u}_{||}$ strongly depends
        of the PU since more hadronic activity tends to increase jet energy mismeasurement.

        \begin{figure}[h!]
            \centering
            \includegraphics[width=0.58\textwidth]{CMS/MET/resolutionVsQt}
            \includegraphics[width=0.41\textwidth]{CMS/MET/resolutionVsPU}
            \caption{On the left, measurement of the ratio $-u_{||}/q_T$ as function of $q_T$
            if $Z\rightarrow\mu\mu$, $Z\rightarrow e e$ and $\gamma$+jets events using
            the particle-flow based $\MET$. On the right, measurement of the resolution of
            $u_{||}$ as function of the number of reconstructed vertices, for
            $Z\rightarrow\mu\mu$ events.}
            \label{fig:METresolution}
        \end{figure}


        \section{Collisions and detector simulations}

            After registering the events from the detector, and reconstructing the particles
            produced, one needs a mean to compare the experiment with theories. Given
            the complexity of a single bunch crossing, including the hard scattering,
            underlying event and pile-up, and the complexity of the detector, the distribution
            of observables can not in the general case be predicted analytically. One must
            instead rely on Monte-Carlo generations and an accurate detector simulation.
            Such simulation can also help design the detector in the first place, for
            instance by studying the performances on a given process, all the way
            from the hard scattering to the objects after reconstruction.

            The full chain of simulation can be implemented as follow. First, the model
            is implemented and the feynman rules associated to the Lagrangian are computed.
            Then, for a given process, a Monte-Carlo generator computes all possible
            feynman diagrams up to a given order, and randomly generates hard-scatterings.
            To these hard-scatterings are added the underlying event, initial and
            final state radiations, and the quarks are hadronized. Finally the particles
            are propagated to the detector and its response is simulated.

            \subsection{Monte-Carlo generations of the hard scattering}

            \todo{ref to arXiv:hep-ph/0403045v2}

            To get a better idea of how Monte-Carlo event generation of the hard scattering
            is performed, it is useful to take a simple example such as the diagram
            $u\bar{u} \rightarrow Z \rightarrow d\bar{d}$ at tree level. The corresponding
            infinitesimal cross section is
            \eq{crossSectionEventGeneration}
            {
                d \sigma(u\bar{u} \rightarrow Z \rightarrow d\bar{d})
                =
                \frac{1}{2\hat{s}}
                \left|
                    \mathcal{M}(u\bar{u} \rightarrow Z \rightarrow d\bar{d})
                \right|^2
                \frac{d\text{cos}(\theta)d\phi}{32\pi^2}
            }
            where $(\theta,\phi)$ are the angles of the $Z$ decay, $\mathcal{M}$
            is the matrix element and $\hat{s}$ is the center of mass energy. The matrix
            element is, for a given diagram, a number computable from the feynman rules.
            To generate events according to this simple process, one may draw random
            values of $cos(\theta)$ and $\phi$, uniformly distributed in $[-1, 1]$ and
            $[0,2\pi]$ respectively. From the mass constrains and the direction of the
            decay, it is then possible to compute the quadri-momentum of the products.

            The value of $d\sigma$ is directly related to the probability of this event
            occuring and is different for each value of $cos(\theta)$ and $\phi$. To obtain
            a set of events with a $\cos(\theta)$, $\phi$ distribution corresponding to
            what is to be observed in Nature, it is possible to unweight these events using
            the hit-and-miss technique.

            With these simple steps, we presented the basic procedure to build a Monte-Carlo
            event generator. However in real life, and in particular in the case of the
            LHC, the initial state is made of protons. Therefore, in order to obtain a
            description of actual collisions, one must consider instead the process
            $pp \rightarrow Z \rightarrow d\bar{d}$ and integrate over the parton distribution
            function.

            Furthermore, a complete description of such a process must also include
            diagrams with additional partons produced in the event, coming for instance
            from initial or final state radiation, e.g. $pp \rightarrow Z \rightarrow
            d\bar{d}$, $d\rightarrow dg$, but also from more complex diagrams. Generating
            diagrams with additional partons in the final state however induces an exponentially growing
            number of diagrams to consider, and one must find adequate ways to implement
            this from a purely software point of view.

            Finally, this description focused on tree-level diagram, but contributions from
            higher order diagrams (NLO, NNLO, ...) are to be taken into account to obtain
            an accurate description of the processus.

            \subsection{Hadronization and hadron decays}

            Partons are affected by QCD and cannot exist in free state, as discussed in
            \refsection{sec:strongInteraction}. If partons move away from each other
            with a sufficient energy, this energy will be used to create new colored
            particles and bind into hadrons forming a jet of particle collimated in the
            direction of the initial partons. This processus is referred to as hadronization,
            and is a crucial point of event generation as it determines the structure
            of generated jets.

            The number of problematics related to hadronization makes it a quite rich and
            active topic. The interface between the generation of the hard scattering
            and the parton shower is not trivial as no factorization technique to rigourosly
            factorize these two problematics. For instance, a gluon emission during
            the parton showering (so-called soft emission) may correspond to an additional
            parton during the generation of the hard process (so-called hard emission),
            ultimately causing double counting and biasing the event set. To solve this
            problem, a factorization prescription, called a matching scheme, must be define
            to remove this double counting .

            There are different techniques to simulate hadronization. The
            technique used in the software \textsc{Pythia} is based on the Lund string
            model. In this model, two partons moving away from each other are linked
            via a gluonic string being stretched, its potential energy growing at
            the expense of its kinetic energy. When its potential energy becomes of the
            order of the mass of two quarks, this string is likely to break and pull
            out a pair of quarks out of vaccuum. This process goes on until
            hadrons are formed, which may in turn decay according to known branching
            fractions. So far, hadronization techniques contains parameters that must
            be tuned according to experiments.

            \subsection{Detector simulation}

            Now knowing the final state after hadronization and decay of the hadrons, we
            want to put our events in the context of a bunch crossing at the LHC. In
            particular, we need to take into account the pile-up. To do so, a number of
            PU collision is drawn according to an arbitrary distribution similar to the
            one expected in the data. Pile-up collisions are added to the event from a
            set of pre-generated and hadronized events.

            Then, we want to simulate the response of the entire CMS subdetectors. To do
            this, a detailed description of the detector is implemented in the software
            \textsc{Geant4}. This description can be tuned to study, for instance,
            misalignment condition. The interaction between the particles emerging from
            the bunch crossing and the material of the detector is simulated using physical
            models \todo{maybe be more specific here}, eventually generating a signal in
            the readout electronics. The trigger is simulated as well, though only to know
            which bits are fired rather than filtering out events at this stage. The
            simulation takes up to a few seconds per event.

            Mainly for time constrains, it is in some case good enough to perform a
            fast simulation by parametrizing the reponse of the detector instead of
            simulating the whole interaction of particles with the detector material.

%==============================================================
\setcounter{mtc}{5}
\chapter{$b$-tagging techniques and validation in CMS}
\minitoc
\newpage
%==============================================================

    $b$ quarks are found in the final state of a large variety of physics processes
    involving top quarks, $Z$ or Higgs bosons, and beyond the Standard Model signatures.
    The hadronization of $b$ quarks produces $B$ hadrons, i.e. hadrons with a structure
    involving a $b$ quark. $B$ hadrons have remarkable properties compared to other hadrons,
    making it possible to identify jets originating from the hadronization of $b$ quarks.
    Such identification is done using $b$-tagging algorithms, and are an important tool of
    a large percentage of analysis in the CMS collaboration to select or reject processes.

    In \refsection{sec:bTagAlgorithms}, we will discuss the $B$ hadrons and $b$ jet properties,
    which are the starting point to construct discriminating observables between $b$ and lights
    and $c$ jets. It will then be presented how these variables are used to build the $b$ jets
    tagging algorithms.
    In \refsection{sec:bTagValidation}, we shall discuss the $b$-tagging
    validation activity within the CMS collaboration, in particular focusing on some major v
    alidations to prepare the Run II of the LHC.

    \section{Topology of $b$ jets and tagging algorithms \label{sec:bTagAlgorithms}}

    \subsection{Properties of $B$ hadrons and topology of $b$ jets}

    The hadronization of $b$ quarks produce $B$ hadrons, which are bound states of
    a $b$ quark and one or two other quarks. The study of $B$ hadrons decays is
    a field in its own, and of particular importance as they are related to the CKM matrix
    and CP violation. $B$ hadrons have particular properties compared to other hadrons
    found in light jets, i.e. arising from the hadronization of $u$, $d$, $s$ quarks and
    gluons, making it possible to identify $b$ jets. $C$ hadrons, which are bound states
    of a $c$ quark, also share to a lesser extends some of these properties, which will
    naturally make $c$ jets more difficult to distringuish from $b$ jets.
    \begin{itemize}
        \item \textbf{Large mass} - $B$ hadrons have masses ranging from $5$ to $10 \GeV$, as
              they contain a $b$ quark. This is a large mass compared to hadrons from
              light jets, typically less than $0.1 \GeV$.
        \item \textbf{Long life time and decay length} - $B$ hadrons decay via the weak
              interaction, with a strength proportionnal to the CKM matrix elements $V_{cb}$ and $V_{ub}$
              which are of the order of $10^{-2}$ and $10^{-3}$ respectively. Because of
              the smallness of this parameter, the life time of $B$ hadrons is about
              $10^{-12}~\text{s}$, corresponding to a decay length of a few tenth of millimeters
              ($10^{-4}~\text{mm}$). In comparison, light hadrons have life time of the order
              of $10^{-16}~\text{s}$.
        \item \textbf{High charged multiplicity decay} - $B$ hadron decays typically
              contain 5 charged particles on average, whereas other light hadrons decay
              usually decay to 1 or 3 charged particles.
        \item \textbf{Leptonic decay} - As the decay of $B$ hadrons involve a virtual $W$,
              they are likely to directly decay leptonically, with a branching ratio of
              11\% per lepton family. This fraction goes up to about 20\% if one
              includes the full decay cascade since $C$ hadrons may also decay leptonically.
    \end{itemize}

    In the context of a $b$ jet, several particles including the $B$ hadron emerge from
    the primary vertex where the hard scattering and hadronization occured. Due to its
    life time, the $B$ hadron decays a few tenth of millimeters away from the primary
    vertex, producing a secondary vertex, as represented on \reffig{fig:bTagging/jetTopology}.
    In the cases where $B$ hadrons decay to a $C$ hadron, a tertiary vertex might
    be produced due to the life time of $C$ hadrons.
    \insertFigure{bTagging/jetTopology}{0.75}
                 {Topology of a jet originating from the hadronization of a $b$ quark : the
                 jet contains a neutral $B$ hadron with a decay length around $0.5~\text{mm}$,
                 producing a secondary vertex with a high multiplicity of charged particles
                 emerging from it.}

    Provided a good enough tracking resolution, such a topology can therefore be identified
    by looking for displaced tracks and the fact that several of them are compatible with
    originating from a common secondary vertex. Other properties may be used, such as their
    distribution inside the jet, or the presence of a charged lepton.

    \subsection{$b$-jet discriminating quantities and objects}

    For a given jet, the tracks considered for $b$-tagging studies must have
    a good quality (normalised $\chi^2 < 5$ as well as sufficient number of hits in the
    pixel and in the tracker as a whole) and $\pT > 1\GeV$. The track must be within
    $\Delta R < 0.3$ of the jet axis (except for the TCHP algorithm which uses $\Delta R < 0.5$).
    The points $J$ on the track and $Q$ on the jet axis are defined as the points of closest
    approach between the track and jet axis. In a similar way, the point $P$ is defined
    as the point of closest approach on the track with respect to the primary vertex $V$.
    The situation is summarized on \reffig{fig:bTagging/impactParameter}. To reject
    contributions from pile-up and other jets that would be interpreted as displaced
    tracks, the distance to jet axis, $JQ$, is required to be lower than $700~{\mu\text{m}}$
    and the so-called decay length of the track, $VQ$, is required to be lower than
    $5~\text{cm}$.

    \insertFigure{bTagging/impactParameter}{0.75}
                 {Representation of two tracks inside a $b$ jet, one coming a prompt charged
                 hadron and the other from a charged hadron originating from the decay of
                 the $B$ hadron. The prompt track has been offseted from the primary vertex
                 to represent resolution effect. One can define along the track the moment
                 of closest approach to the primary vertex, $P$, and the the closest approach
                 to the jet axis, $Q$. The impact parameter corresponds to the distance $VP$
                 and is given the sign of the scalar product $\vec{VP} \cdot \vec{j}$.}

    \subsubsection{Impact parameter}

    An important quantity to characterize the displacement of tracks is the impact parameter.
    As represented on \reffig{fig:bTagging/impactParameter}, the impact parameter is defined
    as the distance of closest approach between the track and the vertex, $VP$. Additionnaly,
    it is given the sign of the scalar product $\vec{VP} \cdot \vec{j}$ where $\vec{j}$ is
    the jet direction.

    Because of their displacement, tracks from the decay of $B$ hadrons are expected to
    have a large, positive impact parameter. Tracks in light parton jets are instead expected
    to originate directly from the primary vertex and therefore to have non-zero value
    only due to resolution effects. To integrate the knowledge of the tracking resolution,
    the uncertainty on the impact parameter, $\sigma_\text{IP}$ can be computed for each
    track, and one can define the impact parameter significance, $S_\text{IP} \definedAs
    \text{IP}/\sigma_\text{IP}$.

    \reffig{fig:bTagging/lowLevelVariables} presents on the top row the distribution of the
    impact parameter (IP) and impact parameter significance for tracks in $b$ jets, $c$
    jets and light jets. In particular, the distributions falls rapidly for tracks from
    lights jets after $S_\text{IP} = 2$. The distribution for tracks from $b$ jets shows
    a clear large tail in large positive values, though some tracks have low or negative
    $S_\text{IP}$ because they are prompt tracks.

    \subsubsection{Secondary vertex}

    One can attempt to reconstruct a secondary vertex corresponding to the common origin
    of several tracks of the jet. This can be done using adaptive vertex fitting techniques,
    much like for the identification of primary vertices described in \refsection{sec:trackingAndVertexing}
    but using parameters relevant to this context and in particular to be robust against
    outliers. Several quantities related to secondary vertex candidates can then be computed,
    such as the distance from the primary vertex (or flight distance), the flight direction
    and the vertex mass.

    $B$ hadrons are expected to have a flight distance of a few tenth of millimeters, a
    flight direction within $\Delta R < 0.5$ of the jet direction and a mass of a few $\GeV$.
    To reduce contamination from vertices of long-lived mesons and particles interactions with
    the detector material, secondary vertices with a flight distance higher than $2.5~\text{cm}$
    or a mass compatible with the mass of $K_0$ or higher than $6.5\GeV$ are rejected.

    \reffig{fig:bTagging/lowLevelVariables} presents on the middle row the distribution of
    the vertex mass and the flight distance significance for the different jet categories,
    showing a clear distinct shape for the distribution of $b$ jets compared to light and
    $c$ jets.

    \subsubsection{Soft leptons}

    Due to the leptonic decay of $B$ hadrons, a significant fraction of $b$ jets are
    expected to contain a soft charged electron or muon. Despite the intrinsic limitation
    due to the low branching ratio, this specificity remains useful to complement other
    techniques and to enrich a sample of events in $b$ jets as light jets have low probability
    to contain a lepton.

    One useful variable, relating the soft lepton to the jet, is $p_T^\text{rel.}$ defined
    as the projection of the lepton momentum to the plane perpendicular to the jet axis
    $\vec{j}$. In this definition, the jet direction is computed by also including the lepton.
    $p_T^\text{rel.}$ is expected to have larger values in $b$ jets compared to light jets.

    Additionnaly, the impact parameter significance of the lepton can be considered to
    characterize its deplacement with respect to the primary vertex. In that particular
    case, the significance can benefit from the excellent resolution on muon tracking
    provided by the detector.

    \reffig{fig:bTagging/lowLevelVariables} presents on the bottom row the distribution of
    the $\Delta R$ and $p_T^\text{rel.}$ for leptons inside the different categories of
    jets, showing the discriminating power that they provide.

    \begin{figure}[th!]
        \centering
        \begin{minipage}{\textwidth}
        \includegraphics[width=0.49\textwidth]{bTagging/8TeV/TTbar_PU_lowLevelVariables/IPTag_ip_3D_GLOBAL_all}
        \includegraphics[width=0.49\textwidth]{bTagging/8TeV/TTbar_PU_lowLevelVariables/IPTag_ips_3D_GLOBAL_all}\\
        \includegraphics[width=0.49\textwidth]{bTagging/8TeV/TTbar_PU_lowLevelVariables/CSVTag_vertexMass_GLOBAL_all}
        \includegraphics[width=0.49\textwidth]{bTagging/8TeV/TTbar_PU_lowLevelVariables/CSVTag_flightDistance3dSig_GLOBAL_all}\\
        \hspace*{0.4cm}
        \includegraphics[width=0.49\textwidth]{bTagging/8TeV/muonDeltaR}
        \includegraphics[width=0.49\textwidth]{bTagging/8TeV/muonPtRel}
        \end{minipage}
        \caption{Distribution of observables estimated from a $t\bar{t}$ sample and for
        different jet categories. On the top row, the impact parameter and its significance
        for tracks inside the jets. On the middle row, the vertex mass and the flight
        distance of the vertex. On the bottom row, $\Delta R$ and $p_T^\text{rel.}$ for
        electrons and muons inside jets.}
        \label{fig:bTagging/lowLevelVariables}
    \end{figure}

    \subsection{$b$-tagging algorithms}

        \subsubsection{Performances}

    Before introducing the algorithms, we shall briefly introduce the notion of performance
    of an algorithm. The algorithm usually associate to each a real value, the discriminator,
    to each jet, and the final step consists in setting a threshold such that a jet
    is said $b$-tagged if its discriminator is higher than the threshold.

    Typically, one wants to study what is the tagging effiency associated as function of
    the threshold value, that is to say the probability that, given a $b$ jet, it is
    succesfully $b$-tagged. Similarly, it is relevant to look at the fake rate, i.e. the
    probability that given a non-$b$ jet, it gets incorrectly $b$-tagged. Because the $c$
    jets are harder to differentiate from $b$ jets, the fake rate of these jets is studied
    independently from light jets.

    The performances are then usually presented by showing the fake rate as function
    of the $b$ jet efficiency. To benchmark the performances and allow easier comparison
    of the performances, one can define three operating points on this curve. The
    loose, medium and tight working points are therefore defined such that the fake rate
    for light jets is respectively 10, 1 and 0.1\%.

    Finally, the performances are likely to depend on the environment of the events in which
    the jets are, in particular if there are other hadronic activity from the hard-scattering
    or from pile-up. In this section, the performances are computed from a $t\bar{t}$ with
    pile-up included.

        \subsubsection{Track counting algorithm}

    The track counting algorithm is based on the impact parameter of tracks. Tracks are
    sorted according to the value of the impact parameter significance. Because of this
    ranking, the $S_\text{IP}$ of the first track is biased and is likely to be high even
    for light jets. The second track offers a better compromise between background (light
    jets) rejection and signal ($b$ jets) efficiency. A first algorithm is defined using
    the $S_\text{IP}$ associated to this track and is called track counting high efficiency
    (TCHE). Using the third track $S_\text{IP}$ as discriminator leads to a drop in the
    signal efficiency but on the other hand to a very low level of background. This
    version of the algorithm is therefore called track counting high purity (TCHP).

    The first row of \reffig{fig:bTagging/perfTCandJPandSSV} presents the discriminant and
    performances for the different jet categories for the high efficiency version of this
    algorithm. The $b$-jet efficiency are around 77\%, 60\% and 18\% for the loose, medium
    and tight working points respectively.

        \subsubsection{Jet probability}

    A more elaborate technique consist in using the impact parameter significance of
    several tracks in the jet. This can be done by computing a likelihood that all tracks
    associated to the jet originate from the primary vertex. The starting point for this
    is to know the distribution of $S_\text{IP}$ for prompt tracks and to compute, for
    each given track with impact parameter significance $X$, the probability $P_\text{track} =
    P(S_\text{IP} > X)$. To protect the algorithm from single, poorly reconstructed tracks,
    a lower bound is put on $P_\text{track}$ at 0.005. The probability $P_\text{jet}$ is
    then computed with the likelihood estimator, using the $N$ tracks of the jet,

    \eq{JPlikelihood}
    {
        P_\text{jet} = \Pi \cdot \sum_{i=0}^{N-1} \frac{(-\text{ln } \Pi)^i}{i!} \text{,\hspace*{0.6cm} with \hspace*{0.6cm}} \Pi = \prod_{i=1}^{N} \text{max}(P_{\text{track }i},0.005)
    }

    The jet probability (JP) discriminator is finally defined as $- \text{ln} P_\text{jet}$
    to have a convenient range to work with. An alternative version of this algorithm is
    called jet $B$ probability (JBP) and give more weight to tracks with high $S_\text{IP}$.

    The middle row of \reffig{fig:bTagging/perfTCandJPandSSV} presents the discriminant and
    performances for the different jet categories for the jet probability algorithm. The
    $b$-jet efficiency are around 81\%, 55\% and 39\% for the loose, medium and tight
    working points respectively.

        \subsubsection{Simple secondary vertex}

    The simple secondary vertex (SSV) algorithm is based on secondary vertex reconstruction,
    and in particular uses the largest flight distance significance among all secondary
    vertices as a discriminant. As for the track counting algorithm, two versions are defined
    to obtain either high efficiency (SSVHE) using secondary vertices with at least two
    tracks, while the high purity version (SSVHP) requires at least three tracks.
    SSVHE version, this efficiency is around 65\% for $b$-jets, and around 20\% for $c$ jets.

    The last row of \reffig{fig:bTagging/perfTCandJPandSSV} presents the discriminant and
    performances for the different jet categories for the high efficiency of this algorithm.
    The maximum efficiency achievable is limited by the intrinsic efficiency of
    actually finding a secondary vertex which satisfies the constrains. For this version,
    it is around 62\% for $b$-jets and 20\% for $c$ jets, while fake vertices rate (i.e.
    probability to find a vertex in light jets) is around 2\%. The $b$-jet efficiencies
    for the medium and tight working points are around 58\% and 20\% respectively.

        \subsubsection{Soft leptons algorithms}

    One can target the $b$ jets containing soft leptons with dedicated algorithms. However,
    as no observable exhibits a strong discriminating power, a multivariate analysis
    has to be performed. Several discriminating observables are fed into a neural network
    which is trained to differentiate signal jets ($b$ jets) from background jets (light jets).
    These observables include $p_T^\text{rel.}$, the $\Delta R$ between the lepton and
    jet axis, the relative lepton momentum, the impact parameter significance of the lepton
    and the lepton quality. Two algorithms are defined, targetting either jets with a
    muon (soft muon tagger, SMT) or electron (soft electron tagger, SET). While muons are
    well-identified object because they leave a distinct signal in the muon system,
    electrons are more challenging due to the hadronic environment and dedicated in-jet
    electron identification must be defined. The maximum efficiency achievable for each
    lepton flavor is intrinsically limited by the leptonic branching ratio of $B$ hadrons,
    which is about 20\% per flavor.

    In practice, the maximum $b$-jet efficiency for the the soft electron tagger is around
    14\% compared to around 17\% for the soft muon tagger. The top row of
    \reffig{fig:bTagging/perfSMTandCSV} presents the discriminant and performances of the
    SMT algorithm. The $b$-jet efficiencies at medium and tight working points are around
    10\% and 5\% respectively.

        \subsubsection{Combined secondary vertex}

    Finally, one wants to combine the techniques previously described (except the
    soft lepton taggers which are recent development) as they complement each other. To
    do so, first, jets are divided into categories according to whether or not a secondary
    vertex has been reconstructed. An intermediate categories is designed for jets
    with no vertex fit but with still two tracks with $S_\text{IP} > 2$, which are used to
    define a pseudo-vertex in 15\% of the cases where no vertex is found for $b$ jets.
    With the goal of improving the rejection of $c$ jets, for jets with a vertex or pseudo-vertex,
    the tracks are ordered in according to their impact parameter significance, and the
    $S_\text{IP}$ of the first track to raises the invariant mass of the vertex above
    the charm threshold ($1.5 \GeV$) act as a good discriminating variable.

    Likelihood are computed, based on the secondary vertex information, the energy
    and rapidity distribution of the tracks at the secondary vertex compared to the remaining
    tracks in the jet, as well as the $S_\text{IP}$ of each track. Two likelihood ratios
    are defined, to discriminate between light jets and $b$ jets, and $c$ jets and $b$ jets.
    These likelihood are then combined by weighting with a factor 0.75 and 0.25 respectively.

    \reffig{fig:bTagging/perfSMTandCSV} presents, on the bottom row, the discriminant and
    performances of the algorithm. At the loose working point, the $b$-jet efficiency is
    around 78\%, close to the track counting high efficiency algorithm. The $b$-jet efficiency
    at the medium working point is around 65\%, significantly better than track counting,
    simple secondary vertex and jet probability due to the completeness of the combined
    secondary vertex approach. The loose working point has a $b$-jet efficiency of 40\%,
    comparable to the jet probability algorithm.

    \begin{figure}[th!]
        \centering
        \begin{minipage}{\textwidth}
        \includegraphics[width=0.49\textwidth]{bTagging/8TeV/TTbar_PU/TCHE_discr_GLOBAL_all}
        \includegraphics[width=0.49\textwidth]{bTagging/8TeV/TTbar_PU/TCHE_performance_vs_B_GLOBAL_all}\\
        \includegraphics[width=0.49\textwidth]{bTagging/8TeV/TTbar_PU/JP_discr_GLOBAL_all}
        \includegraphics[width=0.49\textwidth]{bTagging/8TeV/TTbar_PU/JP_performance_vs_B_GLOBAL_all}\\
        \includegraphics[width=0.49\textwidth]{bTagging/8TeV/TTbar_PU/SSVHE_discr_GLOBAL_all}
        \includegraphics[width=0.49\textwidth]{bTagging/8TeV/TTbar_PU/SSVHE_performance_vs_B_GLOBAL_all}\\
        \end{minipage}
        \caption{Superimposed discriminant distributions for light, $c$ and $b$ jets (on
        the left) and corresponding performances in term of $b$-jet efficiency versus
        misidentification rate (on the right) for the track counting high efficiency
        (top row), jet probability (middle row) and simple secondary vertex high efficiency
        (bottom row) algorithms. These distributions and performances were estimated from
        a $t\bar{t}$ samples in 8 TeV conditions, including pile-up.}
        \label{fig:bTagging/perfTCandJPandSSV}
    \end{figure}

    \begin{figure}[th!]
        \centering
        \begin{minipage}{\textwidth}
        \includegraphics[width=0.49\textwidth]{bTagging/8TeV/TTbar_PU/SMT_discr_GLOBAL_all}
        \includegraphics[width=0.49\textwidth]{bTagging/8TeV/TTbar_PU/SMT_performance_vs_B_GLOBAL_all}\\
        \includegraphics[width=0.49\textwidth]{bTagging/8TeV/TTbar_PU/CSV_discr_GLOBAL_all}
        \includegraphics[width=0.49\textwidth]{bTagging/8TeV/TTbar_PU/CSV_performance_vs_B_GLOBAL_all}\\
        \end{minipage}
        \caption{Superimposed discriminant distributions for light, $c$ and $b$ jets (on
        the left) and corresponding performances in term of $b$-jet efficiency versus
        misidentification rate (on the right) for the soft muon tagger (top row), and
        combined secondary vertex high efficiency (bottom row) algorithms. These
        distributions and performances were estimated from a $t\bar{t}$ samples in 8 TeV
        conditions, including pile-up.}
        \label{fig:bTagging/perfSMTandCSV}
    \end{figure}

    \insertFigure{bTagging/vbfHbb_Candidate}{0.8}{Illustration of $b$-tagging in an event
    at $\sqrt{s} = 8\TeV$. The display shows the surrounding region of the primary vertex
    and the tracks reconstructed, in a $H \rightarrow b\bar{b}$ candidate event produced through
    vector boson fusion (VBF) characterized by the two forward jets. The green tracks are
    prompt tracks while the blue tracks have a high impact parameter significance. One
    clearly distinguish the secondary vertices in each of the central jets which have
    therefore a high $b$-tagging discriminant. The invariant mass of the two $b$ candidates
    is found to be close to $125 \GeV$. \todo{HIG-13-011} \todo{ref to this fig in the
    text}}

    \section{$b$-tagging validation in the CMS collaboration \label{sec:bTagValidation}}

    In this section, we focus on the contribution on $b$-tagging during this thesis.
    After presenting the context of the work and dicussing statistical aspects, some
    important validations are presented in particular, related to the preparation of the
    Run II of the LHC.

        \subsection{Context and method}

    The software of the CMS collaboration is constantly evolving. The development cycle
    includes rigorous and recurring checks by all the detector groups (DPG), physics
    object groups (POG) and analysis groups (PAG). The frequency of such a validation procecure
    of the software ensures an early indentification of bugs and to understand
    the evolution of distributions and performances. The type of changes in the CMS
    software includes, but are not limited to :
    \begin{itemize}
        \item \textbf{Versions of dependencies} such as \textsc{Geant4} and
              \textsc{Root} which may in turn impact the description of particle-matter
              interaction or statistical aspects.
        \item \textbf{Generators and simulation workflow}, for instance related to
              the tunning of the parton shower software, or the simulation of pile-up.
        \item \textbf{Detector geometry and description}, which is likely to change during
              long shutdowns as subdetectors may be upgraded.
        \item \textbf{Alignment and calibration conditions} can be tuned for instance to
              simulate early data-taking with misaligned tracker layers.
        \item \textbf{Reconstruction techniques}, which are likely to evolve accross time to include
              new methods, tune parameters or optimize performances.
    \end{itemize}

    In particular, $b$-tagging algorithms are highly dependent on tracking aspects,
    particle flow algorithms, as well as jets reconstruction, kinematic and pile-up conditions.

    The validation activity consists in comparing typical $b$-tagging variables and performances
    between a new version and an older one, the reference. The reference is usually taken to
    be the previous version to only consider the changes introduced in the new version.
    One can estimate the $b$-tagging performance for the two versions and check their
    compatibility.

    \subsection{Sample size and impact on performance estimation}

    The validation work is made on four different samples to allow to cross-check results
    and factorize the impact of pile-up or simulation type :
    \begin{itemize}
        \item a QCD sample with generated partons with $50 < \pT < 120\GeV$ and full simulation
        of the detector
        \item a $t\bar{t}$ sample with no pile-up and full simulation of the detector
        \item a $t\bar{t}$ sample with with pile-up and full simulation of the detector
        \item a $t\bar{t}$ sample with no pile-up and fast simulation of the detector
    \end{itemize}

    As the validation is a reccurent process and the production of samples takes CPU
    ressources and disk space, each sample contains only about 9000 events. This constrains
    impacts directly the magnitude of statistical fluctuations one will observe in
    variable distributions and in the performance estimation, in turn affecting the check of compatibility
    between the two versions.

    The compatibility of the two versions, the release of be validated (val.) and the
    reference (ref.) can be quantified in the following way. The efficiency for a given
    cut on the discriminant is computed in both versions and yields the corresponding
    values $\epsilon_\text{val.}$ and $\epsilon_\text{ref.}$ with relative uncertainty
    $\sigma^\text{rel.}_\epsilon$. The ratio $r \definedAs \epsilon_\text{val.}/\epsilon_\text{ref.}$
    is computed, and one wants to check if the ratio is significantly different from 1.
    The relative uncertainty on this ratio, $\sigma^\text{rel.}_r$, is $\sqrt{2} \cdot \sigma^\text{rel.}_\epsilon$,
    assuming that $\epsilon_\text{val.}$ and $\epsilon_\text{ref.}$ are uncorrelated.

    To get an idea of the sensitivity of the validation, let's investigate what are the
    conditions to be able to see effects of magnitude lower than 20\% on $\epsilon$ at 2
    sigma level. We want $2 \cdot \sigma^\text{rel.}_r < 0.2$, and therefore approximately
    $\sigma^\text{rel.}_\epsilon < 7\%$. Considering the efficiency estimation $\epsilon
    = k/n$ from observing $k$ objects passing a selection among $n$, one can define a
    critical efficiency such that the relative uncertainty on the efficiency estimation
    will be higher than 7\% :

    \eq{criticalEfficiency}
    {
        \sigma^\text{rel.}_\epsilon = \frac{\sigma_\epsilon}{\epsilon} = \sqrt{\frac{1}{N} \frac{1-\epsilon}{\epsilon}} > 0.07
        \hspace*{1cm}
        \Rightarrow
        \hspace*{1cm}
        \epsilon < \frac{1}{\frac{N}{200} + 1}
    }

    \reftab{tab:criticalEfficiencyBTag} presents the number of jets according to their flavor
    and for the QCD and $t\bar{t}$ samples, as well as the critical efficiency associated
    to these numbers. This efficiency is about 1.2\% for lights jets, meaning that effects
    lower than 20\% will be undistinguishable from statistical fluctuations below the medium
    working point.

    \begin{table}
    \centering
    \begin{tabular}{c|cc|cc}
        \hline
                         & \multicolumn{2}{c}{QCD sample} & \multicolumn{2}{c}{$t\bar{t}$ sample} \\
        \hline
           Jet category  & Number & Critical efficiency   & Number & Critical efficiency \\
        \hline
            $u,d,s,g$    & 17000  & 1.2\%                 & 19600  & 1.0\% \\
            $c$          & 1400   & 12\%                  & 4500   & 4\%   \\
            $b$          & 700    & 22\%                  & 13000  & 1.5\% \\
        \hline
    \end{tabular}
        \caption{Typical number of selected jets per flavor category for different sample types,
        and critical efficiency at which the relative uncertainty on the efficiency will
        get higher than 10\%. \label{tab:criticalEfficiencyBTag}}
    \end{table}

    To avoid this problem, some validations only intended to study the impact of reconstruction
    techniques and algorithm recycle the Monte-Carlo generation and detector simulation
    steps of previous validation samples, in order to remove the statistical fluctuations
    coming from these steps.

        \subsection{Validation of new jet size parameter for Run II}

        The preparation for the Run II of the LHC involve a change in the jet size parameter.
        Because the jets are expected to be narrower on average due to the energy increase,
        and to limit the pile-up contamination, this parameter is moved from $R = 0.5$ to $0.4$.

        The validation of this change is made using two set of samples using the same
        generated events and simulation to limit the effect of statistical fluctuations,
        but reconstruction is done with the two different jet size parameters. The sample
        used to produce the following figures is a $t\bar{t}$ sample at $\sqrt{s} = 13\TeV$
        with pile-up.

        \todo{more comments}

        \insertTwoFigures{bTagging/AK4vsAK5/TTbar_FullSim_PU_kinematic}
                         {bTagging/AK4vsAK5/TTbar_FullSim_PU/2_CSV_jetPt_GLOBAL_all}
                         {bTagging/AK4vsAK5/TTbar_FullSim_PU/4_CSV_jetEta_GLOBAL_all}
                         {0.49}
                         {Comparison of the $\pT$ (left) and $\eta$ (right) spectrum of
                         the selected jets in the context of the validation of the new
                         jet size parameter for Run II.}

        \insertFourFigures{bTagging/AK4vsAK5/TTbar_FullSim_PU_lowLevelVariables}
                          {bTagging/AK4vsAK5/TTbar_FullSim_PU_lowLevelVariables/CSVTag_trackDeltaR_GLOBAL_all}
                          {bTagging/AK4vsAK5/TTbar_FullSim_PU_lowLevelVariables/CSVTag_trackSumJetEtRatio_GLOBAL_all}
                          {bTagging/AK4vsAK5/TTbar_FullSim_PU_lowLevelVariables/IPTag_ips_3D_GLOBAL_all}
                          {bTagging/AK4vsAK5/TTbar_FullSim_PU_lowLevelVariables/CSVTag_flightDistance3dSig_GLOBAL_all}
                          {0.49}
                          {Comparison of the $\DeltaR(\text{track},\text{jet axis})$ (top left),
                          ratio of track sum to jet energy (top right), impact parameter
                          significance (bottom left) and flight distance significance (bottom
                          right) in the context of the validation of the new jet size
                          parameter for Run II.}

        \insertFourFigures{bTagging/AK4vsAK5/TTbar_FullSim_PU_performances}
                          {bTagging/AK4vsAK5/TTbar_FullSim_PU/TCHE_performance_vs_B_GLOBAL_all}
                          {bTagging/AK4vsAK5/TTbar_FullSim_PU/JP_performance_vs_B_GLOBAL_all}
                          {bTagging/AK4vsAK5/TTbar_FullSim_PU/SSVHE_performance_vs_B_GLOBAL_all}
                          {bTagging/AK4vsAK5/TTbar_FullSim_PU/CSV_performance_vs_B_GLOBAL_all}
                          {0.49}
                          {Comparison of the algorithm performances for the track counting high
                          efficiency (top left), jet probability (top right), simple
                          secondary vertex (bottom left) and combined secondary vertex
                          (bottom right) in the context of the validation of the new jet size
                          parameter for Run II.}

        \subsection{Validation of \textsc{Geant4} 10.0.patch1}

        A new version of \textsc{Geant4} was introduced in the CMS software v7.1.0 and
        had to be validated. This version contained several changes related to design
        changes and optimisations which are expected to not affect physics results,
        and a new, better model for the multiple scattering of leptons and hadrons.

        As the event generation and simulation of the detector had to be redone for
        this validation, to limit effects due to statistics, special high statistics
        samples were produced allowing a precise comparison of the distributions.

        Overall, little changes are expected. This expectation is confirmed by the
        good agreement between the two versions as illustrated in \reffig{fig:bTagging/newGeant4}
        showing the impact parameter significance and the performances of the CSV algorithm.

        \insertTwoFigures{bTagging/newGeant4}
                         {bTagging/newGeant4/TTbar_710pre4_newGeant_HS_vs_710pre4_HS_FullSim_lowLevelVariables/IPTag_ips_3D_GLOBAL_all}
                         {bTagging/newGeant4/TTbar_710pre4_newGeant_HS_vs_710pre4_HS_FullSim/CSV_performance_vs_B_GLOBAL_all}
                         {0.49}
                         {Comparison of the impact parameter significance (left) and
                         performances of the combined secondary vertex algorithm (right)
                         in the context of the validation of \textsc{Geant4} 10.0.patch1.}

        \subsection{Validation of new default PU rate for Run II validation samples}

        In CMS software v7.3.0, the default pile-up level was changed from 10 to 35. As
        tracking and $b$-tagging variables strongly depends on pile-up effects, this
        new condition is expected to impact significantly the performances.

        \reffig{fig:bTagging/newPU35/tracking} compares the efficiency and fake rate of track
        reconstruction, as function of $\eta$, between a pile-up of 10 (\emph{old}) and
        35 (\emph{new}). The new pile-up condition decreases the track reconstruction
        efficiency by about 2\% in the central region, and increases the fake rate from
        3\% to 8\%.

        \insertFigure{bTagging/newPU35/tracking}{0.95}{Comparison of the tracking
        performances as function of $\eta$ between the pile-up level of 10 (old) and 35 (new).
        The left plot shows the efficiency for real tracks while the right plot shows
        the fake and duplicate rate.}

        On the $b$-tagging side, this translates to more contamination from pile-up and
        fakes, likely to have large impact parameters or to create fake secondary vertices.
        \reffig{fig:bTagging/newPU35/lowLevelVariables} shows the comparison of the impact
        parameter significance and secondary vertex category between the two pile-up conditions.
        The impact parameter distribution for tracks in light jets gets a wider peak around
        0, from which we can expect an increase in fake rate for all algorithm based on
        this variable. The secondary vertex categorization shows also a 5\% absolute loss
        in $b$ jet vertexing efficiency, and an relative increase of around 35\% in light
        jets containing a secondary vertex. From this, we can expect that algorithms based
        on secondary vertex reconstruction will not only a higher fake rate from lights,
        but also a lower maximum efficiency for $b$ jets.

        \insertTwoFigures{bTagging/newPU35/lowLevelVariables}
                         {bTagging/newPU35/TTbar_730pre1NewPU_vs_730pre1_FullSim_PU25ns_lowLevelVariables/IPTag_ips_3D_GLOBAL_all}
                         {bTagging/newPU35/TTbar_730pre1NewPU_vs_730pre1_FullSim_PU25ns_lowLevelVariables/CSVTag_vertexCategory_GLOBAL_all}
                         {0.49}
                         {Comparison of the impact parameter significance (on the left)
                         and vertex category for the CSV algorithm (on the right) in the
                         context of the validation of the new default pile-up rate.}

        \reffig{fig:bTagging/newPU35/performances} presents the impact on the performance
        for the four main algorithms TCHE, JP, SSVHE and CSV. The overall impact is a higher
        fake rate from light jets by about 50\% and from $c$ jets by about 5-10\%. The
        maximum efficiency for $b$ jets for SSVHE is decreased by about 5\%, directly
        related to what is observed on the vertex category in \reffig{fig:bTagging/newPU35/lowLevelVariables}.

        \insertFourFigures{bTagging/newPU35/performances}
                          {bTagging/newPU35/TTbar_730pre1NewPU_vs_730pre1_FullSim_PU25ns/TCHE_performance_vs_B_GLOBAL_all}
                          {bTagging/newPU35/TTbar_730pre1NewPU_vs_730pre1_FullSim_PU25ns/JP_performance_vs_B_GLOBAL_all}
                          {bTagging/newPU35/TTbar_730pre1NewPU_vs_730pre1_FullSim_PU25ns/SSVHE_performance_vs_B_GLOBAL_all}
                          {bTagging/newPU35/TTbar_730pre1NewPU_vs_730pre1_FullSim_PU25ns/CSV_performance_vs_B_GLOBAL_all}
                          {0.49}
                          {Comparison of the impact parameter significance (on the left)
                          and vertex category for the CSV algorithm (on the right) in the
                          context of the validation of the new default pile-up rate.}

        \subsection{Validation of the premixing technique for pile-up simulation}

        In the CMS software v7 cycle, a new technique called premixing started being developped.
        This technique concerns the simulation of pile-up : the number of particles
        originating from pile-up is quite high and represents a large fraction of the CPU time
        during the simulation of the detector (also called digitization).

        In order to optimize
        the time of the simulation, which in turn may allow to produce for instance more events
        for the Monte-Carlo samples, strategy can be developped. In this case, the idea
        is to factorize the digitization step of the pile-up from the hard-scattering.
        By doing this, pile-up events can be digitized once for all, and overlayed
        any hard-scattering event after that, drastically reducing the simulation time
        per event.

        In order for this to work, one must however be careful regarding the way that the
        pile-up and hard-scattering are digitized and combined together. For instance,
        thresholds in the subdetectors, for instance a given calorimetric tower, may
        impact the digitization such that the hard-scattering alone, or the pile-up alone,
        may not produce a significant signal, but the combination of the two may. To
        work around this problem, thresholds are initially set to zero when the digitization
        is done, and applied only after mixing of the pile-up with the hard-scattering.

        Overall, one wants the premixing technique to not impact the physics at the reconstruction
        level. $b$-tagging, as a high-level objects, is a good probe for this to check
        the impact primarily on tracking and the hadronic calorimeter. While there were
        several validations related to this work, this document focuses on a pathological
        validation done in v7.4.0.

        As shown on \reffig{fig:bTagging/premixing/lowLevelVariables}, the version of the
        premixing used led to tracks having an artificially higher number of hits in the
        pixels on average, and therefore a better track quality, which manifests as a lower
        tail for the normalized $\chi^2$ distribution, and overall a better resolution.
        This in turns impact the number of tracks at the vertex for light jets, and the
        impact parameter significance.

        \insertFourFigures{bTagging/premixing/lowLevelVariables}
                          {bTagging/premixing/TTbar_740pre6premix_vs_740pre6_FullSim_PU50ns_lowLevelVariables/IPTag_tkNPixelHits_3D_GLOBAL_all}
                          {bTagging/premixing/TTbar_740pre6premix_vs_740pre6_FullSim_PU50ns_lowLevelVariables/IPTag_tkNChiSqr_3D_GLOBAL_all}
                          {bTagging/premixing/TTbar_740pre6premix_vs_740pre6_FullSim_PU50ns_lowLevelVariables/CSVTag_vertexNTracks_GLOBAL_all}
                          {bTagging/premixing/TTbar_740pre6premix_vs_740pre6_FullSim_PU50ns_lowLevelVariables/IPTag_ips_3D_GLOBAL_all}
                          {0.49}
                          {comparison of the number of pixel hits (top left), track $\chi^2$
                          (top right), number of tracks at vertex (bottom left) and
                          impact parameter significance (bottom right) in the context
                          of the premixing validation.}

        This impact the performances of both secondary vertex-based algorithms and impact
        parameter-based algorithms significantly, as shown on
        \reffig{fig:bTagging/premixing/performances} : TCHE gets up to 35\% lower fake
        rate from lights jets while SSVHE gets about 20\% lower fake rate from $c$ jets.

        \insertTwoFigures{bTagging/premixing/performances}
                         {bTagging/premixing/TTbar_740pre6premix_vs_740pre6_FullSim_PU50ns/TCHE_performance_vs_B_GLOBAL_all}
                         {bTagging/premixing/TTbar_740pre6premix_vs_740pre6_FullSim_PU50ns/SSVHE_performance_vs_B_GLOBAL_all}
                         {0.49}
                         {Comparison of the algorithm performance of track counting high
                         efficiency (left) and simple secondary vertex high efficiency (right).}

        \subsection{Studies of high pile-up scenario for Phase 2}

        This validation relates to one of the early investigation of $b$-tagging at high
        pile-up ($\orderOf(150)$ interactions per bunch crossing). Such conditions are
        foreseen for the operation of the LHC in the Phase I, after the second long shut
        down (LS2), in the data-taking period between 2019 and 2022, and for the following
        Phase II and high-luminosity LHC (HL-LHC). Such prospective work is therefore
        crucial to obtain a first idea of the performance of the detector in these conditions,
        anticipate possible problems and guide the orientation of the development for object
        reconstruction.

        The reference used for this validation is a $t\bar{t}$ sample generated with $\sqrt{s}
        = 14\TeV$. The detector is simulated with the expected geometry for Phase I with
        a pile-up of 50, and the conditions of the technical design report (TDR). The
        comparison is made with a sample with a pile-up of 140.

        \reffig{fig:bTagging/highPU/kinematic} presents the comparison of the kinematic
        distributions of the jet. The first observation is that the increase of pile-up
        significantly bias the $\pT$ spectrum of the jets towards higher values, while the
        generated energy of the underlying partons are expected to have the same spectrum
        between the two releases. It is therefore interpreted as polution from the pile-up
        and lack of strategy used in the reconstruction (at this point) to efficiently
        correct this effect. The higher pile-up rate also create a significantly higher
        number of jets in the central region in $\eta$, contributing to the gluon jet
        category.

        \insertTwoFigures{bTagging/highPU/kinematic}
                         {bTagging/highPU/pT}
                         {bTagging/highPU/eta}
                         {0.49}
                         {Comparison of the $\pT$ and $\eta$ spectrum of the selected jets
                         in the context of the studies of high pile-up scenario for Phase 2.}

        \reffig{fig:bTagging/highPU/kinematic} shows track level information, namely the
        distribution of the number of selected tracks per jets, and the distribution of the
        impact parameter significance for these tracks. In the high pile-up scenario, the
        number of selected tracks is significantly lower for all tracks category. In particular
        for the $b$-jet category, the distribution peaks around 7 or 8 selected tracks per
        jet, compared to around 10 in the TDR conditions. From this observation, one
        can expect a significant decrease in the $b$-jet efficiency. The distribution of
        the impact parameter significance also exhibits a worserning of the discriminating
        power. First, the distribution for light jets exhibits a much wider peak and
        increase of the fraction with large positive values. Secondly, the ratio of the
        distribution for the $b$-jets shows a trend in the positive-tail such that the
        distribution in high pile-up conditions is less in favor of $b$-jets discrimination.

        \insertTwoFigures{bTagging/highPU/lowLevel}
                         {bTagging/highPU/nSelTracks}
                         {bTagging/highPU/ips}
                         {0.49}
                         {Comparison of the number of selected tracks per jet (left) and
                         impact parameter significance (right) in the context of the
                         studies of high pile-up scenario for Phase 2.}

        \reffig{fig:bTagging/highPU/perf} shows the comparison of the performances curves
        obtained for the two scenarios for the four main algorithms : track counting high
        efficiency, simple secondary vertex, jet probability and combined secondary vertex.
        Track counting and jet probability, both relying essentially on the impact parameter
        significance distribution, show a dramatic increase of the fake rate by a factor
        5 to 10 in the high PU scenario. The maximum efficiency achievable for $b$ jets
        with these two algorithms, around $85\sim90$\% is interpreted as a significant
        fraction of the $b$ jets not having the requirements to actually compute the
        discriminator value, such as at least two selected tracks.

        The maximum $b$ jet efficiency for simple secondary vertex is also significantly
        affected and gets down to about 55\%, while the fake rate for is only increased by
        a factor between 2 and 5.

        The combined secondary vertex performances finally gives information to what can be
        expected from combining the techniques together. The fake rate for light jets
        is overall increased by a factor 5 to 10 while it is increased by a factor 2 for $c$ jets.
        The $b$ jets efficiency in the high pile-up scenario becomes 65\%, 42\% and about 20\%
        for the loose, medium and tight working points respectively.

        \insertFourFigures{bTagging/highPU/perf}
                          {bTagging/highPU/TCHEperf}
                          {bTagging/highPU/SSVHEperf}
                          {bTagging/highPU/JPperf}
                          {bTagging/highPU/CSVperf}
                          {0.49}
                          {Comparison of the performances for track counting high efficiency
                          (top left), simple secondary vertex high efficiency (top right),
                          jet probability (bottom left) and combined secondary vertex
                          (bottom right) in the context of the studies of high-pile-up
                          scenario for Phase 2.}


%==============================================================
\setcounter{mtc}{6}
\chapter{Search for stop pair production at $\sqrt{s} = 8\TeV$}
\minitoc
\newpage
%==============================================================

    This chapter focuses on an analysis performed within the CMS collaboration
    and searching for the production of stop pair using the data recorded during
    the Run I of the LHC at $\sqrt{s} = 8\TeV$. In \refsection{sec:analysis_contextAndPheno},
    we concentrate on the context and phenomelogy of the signature while
    \refsection{sec:analysis_objectAndEventSelection} to \refsection{sec:analysis_results} discuss
    the different aspects of the analysis itself and the results. In the last
    \refsection{sec:analysis_perspective}, some
    of the perspectives for this analysis are presented in particular to investigate the
    use of $W$-tagging, as well as the preparation of the Run II.

    \section{Context and phenomenology \label{sec:analysis_contextAndPheno}}
    %==============================================================

        \subsection{Theoretical context and assumptions}
        %==============================================================

        \todo{This theoretical subsection will be updated after being done writing
        the theoretical chapter to correctly sync things.}

        One of the best assets of supersymmetry is its ability to explain the low mass
        of the Higgs boson, since the boson-fermion symmetry introduces a mechanism to
        protect the mass of the Higgs . This is illustrated on
        \reffig{fig:higgsCorrections}, showing the one-loop corrections to $m_H^2$
        from a fermionic and bosonic field coupled to the Higgs via the Lagrangian terms
        $- \lambda_f H \bar{f} f$ and $\lambda_b \left| H \right|^2 \left| b \right|^2$.
        The leading order corrections writes :

        \insertTwoFigures{higgsCorrections}
        {feynmanDiagrams/output/higgsFermionCorrection}
        {feynmanDiagrams/output/higgsBosonCorrection}{0.4}
        {One-loop correction to the Higgs for a fermionic field $f$ and bosonic field $b$.}

        \begin{equation}
            \Delta \mass{h}^2 = - \frac{\left| \lambda_f \right|^2}{8\pi^2} \Lambda^2_{\text{UV}}
            \, \, \, \, \, \, \, \, \, \, \text{and} \, \, \, \, \, \, \, \, \, \,
            \Delta \mass{h}^2 =   \frac{\lambda_b}{16\pi^2} \Lambda^2_{\text{UV}}
        \end{equation}

        where $\Lambda^2_{\text{UV}}$ is the ultraviolet momentum cutoff which regulates
        the loop integral. In the most ideal case, one could have $\left| \lambda_f \right|^2
        = \lambda_b$ and associate two scalar to each fermion (one for each chirality),
        the corrections cancel each others. \todo{should clarify link between $\lambda$ and masses..}
        This however does not happens if the mass of the particle and its superpartner
        are not the same, and tunning has to be reintroduced to keep the Higgs mass low.
        This is especially important in the case of the top as the correction it introduces
        on $\mass{h}^2$ is one of the biggest.

        One can estimate the level of tunning needed as function of the superpartner of
        the top for the theory to keep providing a natural explanation to the hierarchy
        problem, and use it as a constrain. Despite the fact that such constrain is
        highly dependent on hypothesis made on the SUSY parameters, it is commonly admitted
        that stop quarks should have a mass below or around $1\TeV$ for SUSY to remain natural.
        This makes the search for stop and important channel for possible SUSY discovery.

        \todo{from arXiv:1212.6847, A light stop can be very helpful to obtain the right dark-matter relic density, which
        is typically too large for B-ino LSP or too small for higgsino or W-ino LSP in generic
        supersymmetric models. The process of coannihilation selects the preferred value of
        the mass difference between stop and neutralino.}

        \todo{http://arxiv.org/pdf/1211.2997v1.pdf, http://arxiv.org/pdf/1212.4856v2.pdf}

        A second appealing feature of supersymmetry is that it can provide a dark matter
        candidate. This happens in R-parity conserving models where the lightest supersymmetric
        particle (LSP) is a natural WIMP candidate if it is not a charged particle. In the
        context of the MSSM, the lightest neutralino $\lneutralino$, the gravitino $\tilde{G}$
        and the lightest sneutrino $\tilde{\nu}$ can be the LSP and therefore are dark matter
        candidate. The lightest neutralino is the one that is most often studied. Neutralinos
        $\tilde{\chi}$ are combination of the bino $\tilde{B}$, the two neutral higgsinos
        $\tilde{H}_{1,2}$, and the neutral wino $\tilde{W}_3$. One can compute the relic
        abudance of dark matter by solving the Boltzmann equation \cite{EllisDarkMatter}
        depending essentially on the LSP mass $m_{\text{LSP}}$ and the annihilation cross-section.
        The next-to-lightest supersymmetric particle (NLSP) mass is a crucial parameter as
        it strongly affect the annihilation cross-section and therefore the relic abudance.
        In particular, cosmological observations are favorizing cases where the LSP is
        degenerated with the lightest stop quark $\lstop$ or stau $\tilde{\tau_1}$.

        These two considerations lead to a logical interest in phenomenology where both a
        neutralino LSP and a light stop NLSP are relatively light ($\lesssim 1\TeV$) and
        accessible at the LHC.

        \subsection{Phenomenology and signature}
        %=======================================

        In this subsection, we introduce the processes and phenomenology of direct stop
        quark pair production that will be the target of the search described in
        \refsection{sec:analysis_objectAndEventSelection} to \refsection{sec:analysis_results}.

        As it is in practice impossible to scan the whole phase space of SUSY models, a
        pragmatic approach often consists in using simplified SUSY models where an
        effective lagrangian introduces a limited set of new physics features. This makes
        it possibles for experimental searches to produced generic results that can later
        be reinterpreted in specific realizations of SUSY \cite{LiemSMS, SmodelS}
        or other BSM theories. In our case, let's consider the existence of only three
        particles\footnote{The remaining SUSY particles can be assumed to be too heavy to
        have any significant impact on what is discussed here.} : the lightest stop quark $\lstop$, the lightest neutralino $\lneutralino$
        and the lightest chargino $\lchargino$. We recall that $\lneutralino$ and $\lchargino$
        are the lightest neutral and charged mass eigenstates formed by the linear
        combination of the gauginos and higgsinos. $\lneutralino$ is assumed to be the LSP
        and escapes detection. Our interest here is in the case where a stop pair
        $\lstop\lstop^*$ is produced during a $pp$ collision.

        First, we can assume that $\mass{\lchargino} \gg \mass{\lstop}$ and therefore the
        stop decays through $\lstop \rightarrow t \lneutralino$, as represented on
        \reffig{fig:stopDecayModes} on the left. This signal is referred to as \textsc{T2tt}
        in the simplified model nomenclature and depends on two free parameters
        $\mass{\lstop}$ and $\mass{\lneutralino}$.

        A second case is considered with $\mass{\lchargino} \in [\mass{\lneutralino},
        \mass{\lstop}]$ and the stop decays through $\lstop \rightarrow b \lchargino
        \rightarrow b W^{\pm} \lneutralino$. This signal is referred to as \textsc{T2bw}
        in the simplified model nomenclature. In addition to the free  parameters
        $\mass{\lstop}$ and $\mass{\lneutralino}$, $\mass{\lchargino}$ is set through a
        third parameter $x$ defined such that $\mass{\lchargino} = x \cdot \mass{\lstop}
        + (1 - x) \cdot \mass{\lneutralino}$. We study three cases $x = 0.75$, $0.50$
        and $0.25$, as represented on \reffig{fig:stopDecayModes} on the right.

        \insertFigure{stopDecayModes}{0.8}
                     {Representation of the mass hierarchy in the $\lstop \rightarrow t
                     \lneutralino$ decay mode (on the left) and $\lstop \rightarrow b
                     \lchargino \rightarrow b W^{\pm} \lneutralino $ decay mode (on the
                     right). For the later, the chargino mass $\mass{\lchargino}$ is
                     parametrized using $\mass{\lchargino} = x \cdot \mass{\lstop}
                     + (1 - x) \cdot \mass{\lneutralino}$ and three different values of $x$ will
                     be studied, $x = 0.75$, $0.50$ and $0.25$.}

        All of these four decay hypotheses are studied independently
        of each other\footnote{We do not consider mixed decays where one stop of the pair
        decays through $\lstop \rightarrow t \lneutralino$ and the other decays through
        $\lstop \rightarrow b \lchargino \rightarrow b W^{\pm} \lneutralino$.} and with a
        branching ratio equal to 1. It should also be noted that the polarization of the
        top quarks in the $\lstop \rightarrow t \lneutralino$ mode, and $\lchargino$ and
        $W$ bosons in the $\lstop \rightarrow b \lchargino \rightarrow b W^{\pm}
        \lneutralino$ mode, are dependent on the mixing matrices of the $\lstop$, $\lchargino$
        and $\lneutralino$. This can later significantly affect the distributions of
        variables and the acceptance of the signal, and will be discussed during the
        interpretation of the results in \refsection{sec:analysis_results}.

        \reffig{fig:stopFeynmanDiagrams} shows the two Feynman diagrams that are
        considered. It is relevant to target both those signals with the same analysis
        considering that, in \textsc{T2tt}, the top quark almost exclusively decays
        through $t \rightarrow b W^+$ and therefore lead to the same intermediate state
        $b \bar{b} W^+ W^- + \lneutralino \lneutralino$ as in \textsc{T2bw}. Because the
        $\lneutralino$ are assumed to be dark matter candidates and do not interact with
        the detector, the signature left by this new physic process is a final state much
        like Standard Model $t\bar{t}$ production but with extra missing transverse energy
        coming ($\MET$) from the two $\lneutralino$.

        \insertTwoFigures{stopFeynmanDiagrams}
                         {feynmanDiagrams/output/T2tt}{feynmanDiagrams/output/T2bw}{0.4}
                         {Feynman diagrams of stop pair production in $pp$ collisions for the
                         $\lstop \rightarrow t \lneutralino$ decay mode (on the left) and
                         $\lstop \rightarrow b \lchargino \rightarrow b W^{\pm} \lneutralino$ decay mode
                         (on the right). The lines of the supersymmetric particles are drawn in red.}

        The cross-section of direct $\lstop$ pair production, which depends only on $\mass{\lstop}$, is presented
        on \reffig{fig:stopPairXsec} for $8\TeV$ and $13\TeV$ $pp$ collisions as computed at next-leading-order
        by the software \textsc{Prospino} \refNeeded. At $8\TeV$, the cross-section range from $\orderOf{100\pb}$
        at $\mass{\lstop} = 150\GeV$ to $\orderOf{1\fb}$ at $\mass{\lstop} = 900 \GeV$.

        \insertFigure{stopPairXsec}{0.8}
        {Direct $\lstop$ pair production cross-section as function of $\mass{\lstop}$, computed at next-to-leading order
        for $8\TeV$ (in blue) and $13\TeV$ (in orange) proton-proton collisions. The bands represent uncertainty from PDF.}

        The search is performed across the $(\mass{\lstop},\mass{\lneutralino})$ plane
        represented on \reffig{fig:stopMassSpace}. Depending on the value of
        $\deltam \definedAs \mass{\lstop} - \mass{\lneutralino}$, different phenomenologies
        appears as discussed hereafter.

    \insertTwoFigures{stopMassSpace}
                     {sketchMassSpace/T2tt}{sketchMassSpace/T2bw}{0.75}
                     {Mass space of the stop pair production search for the $\lstop
                     \rightarrow t \lneutralino$ decay mode (top) and $\lstop
                     \rightarrow b \lchargino \rightarrow b W^{\pm} \lneutralino $ decay
                     mode when $x$ is close to 0 (bottom)}

        The $\lstop \rightarrow t \lneutralino$ decay mode is allowed starting from $\deltam
        > \mass{W} + \mass{b}$. In the region where $\mass{W} + \mass{b} < \deltam < \mass{t}$,
        the available energy in the center of mass is however not sufficient to have an
        on-shell top, and the top is therefore off-shell. This type of kinematic is also
        called a three-body decay of the stop. Around the $\deltam \sim \mass{top}$ limit,
        the kinematic is quite challenging as the signal looks almost identifical to
        the standard model $t\bar{t}$ production. Because of this, this region is called
        stealthy as such a signal is difficult to bring out. In the part of the phase space
        corresponding to $\deltam > \mass{top}$, the top is on-shell. The $\MET$ from $\lneutralino$,
        and the $\pT$ of the visible decay products are expected to grow as function of
        $\deltam$, as shown illustrated in \reffig{fig:phenoMET} on top left. The remaining
        part of the phase space, with $0 < \deltam < \mass{W} + \mass{b}$, does not allow
        the decay through a top but instead is expected to decay through $c+\lneutralino$
        (flavor-violating two-body decay) or $b\ell\nu_\ell+\lneutralino$ (four-body decay).
        This topology, called compressed spectra, won't be considered here as it requires
        a dedicated analysis.

        The $\lstop \rightarrow b \lchargino \rightarrow b W \lneutralino$ decay mode also
        has a kinematic transition around $\deltam \sim m_W/x$ corresponding to the limit
        for the $W$ in the decay of the chargino starts to be on-shell. Overall, in the
        case where $x$ is close to 1, the evolution of the kinematic across the plane
        is quite similar to $\lstop \rightarrow t \lneutralino$ in the off-shell top case,
        i.e. depends essentially of $\deltam$.
        However, as $x$ gets close to 0, the mass of the intermediate $\lchargino$ plays
        a much important role in the understanding of the kinematic. For instance, at a
        constant $\mass{\lstop} = 400\GeV$, $\mass{\lchargino}$ varies between $100$
        and $325\GeV$ for $x = 0.25$, compared to a variation between $300$ and $375\GeV$
        for the same situation at $x = 0.75$. The fact that $\mass{\lchargino}$ gets to
        lower values directly impacts the energy of the decay products, an in particular
        the resulting $\MET$ from $\lneutralino$. The effect is visible in the bottom right
        plot of \reffig{fig:phenoMET} where the mean $\MET$ is for example lower at
        $(\mass{\lstop},\mass{\lneutralino}) = (400,0) \GeV$ compared to $(\mass{\lstop},
        \mass{\lneutralino}) = (400,100) \GeV$.

        \insertFourFigures{phenoMET}
                     {pheno/T2tt}{pheno/T2bw075}
                     {pheno/T2bw050}{pheno/T2bw025}
                     {0.45}
                     {Evolution of the mean generated missing transverse energy from neutrinos
                     and neutralinos for the $\lstop \rightarrow t \lneutralino$
                     decay mode (top left) and $\lstop \rightarrow b \lchargino$ with
                     $x = 0.75$ (top right), $0.50$ (bottom left) and $0.25$ (bottom right).
                     A selection requiring at least one high-$\pT$ central
                     electron or muon and at least three high-$\pT$ central jets is applied.}

    After the decay of the tops or charginos, each $W$ can decay hadronically (i.e. into
    a pair of quarks, $W \rightarrow q\bar{q}$) or leptonically (i.e. into a charged
    lepton and a neutrino, $W \rightarrow \ell \nu_{\ell}$). It is common to refer to
    the channel of interest via the number of charged leptons in the final state : 0-lepton
    (or fully hadronic) channel, 1-lepton (semi-leptonic) channel or 2-lepton (di-leptonic)
    channel. In \refsection{sec:analysis_overview} to \refsection{sec:analysis_results}, we focus on
    a search in the 1-lepton channel. This channel has the advantage to be less sensitive
    to multijet background while still having a relatively large branching ratio. In the
    0-lepton, one can however profit from the fact that the main backgrounds are expected
    to contain no genuine $\MET$, and benefit from the ability to fully reconstruct the
    decay of the tops in the $\lstop \rightarrow t \lneutralino$ decay mode. Finally, the
    2-leptons channel, despite its relatively low branching ratio, tends to be competitive
    for the low $\deltam$ region because of the $\pT$ threshold of the dilepton triggers are
    generally lower than the hadronic or single lepton triggers.

    \section{Analysis strategy and overview \label{sec:analysis_overview}}
    %==================================

    While there has been several version of this analysis in the context of the CMS
    collaboration \cite{SUS-12-023-PAS, SUS-13-011-PUB, SUS-14-015-PAS}, this document
    will focus here on the $8\TeV$ legacy version of it \cite{SUS-14-015-PAS}. Furthermore,
    it will emphasize on some aspects which have been studied during the thesis.

    This short section aims to give the reader a general overview of the analysis strategy
    and its key parts to have them in mind before describing further each of them. First,
    we select events with one reconstructed electron or muon, four jets among which at
    least one is $b$-tagged, at least 80 GeV of missing transverse energy, and veto events
    with a second lepton.

    Then, the key variable $\MT$ is introduced as the first main discriminant between signal
    and background. It is defined as the transverse mass of the lepton+$\MET$ system. This
    variable is useful to supress backgrounds for which the only main source of $\MET$ is
    one neutrino $\nu$ coming from a leptonically-decaying $W$. It however less relevant
    to reduce processes with several sources of $\MET$, in particular the dileptonic $t\bar{t}$
    background with one lepton escaping selection, which then becomes a main background. This
    motivates the need of an efficient second lepton veto to reject this kind of events in the
    first place.

    To further increase the sensitivity, two parallel approaches are followed to define
    signal regions that target different parts of the $(\mass{\lstop},\mass{\lneutralino})$
    space. The first one is a cut-and-count approach, in which a counting experiment is performed
    after a minimal set of cuts. The second one is a multivariate approach, in which boosted
    decision trees (BDT) are trained on a set of variables and a counting experiment is performed
    after cutting on the BDT discriminant. This document will focus on the design of the cut-and-count
    approach and its optimization using a figure of merit.

    A background prediction is computed for each signal region using data-driven methods
    in control regions, i.e. enriched in a specific type of background. In particular, we
    will see how the Monte-Carlo description of the tail of the $\MT$ variable needs to be
    corrected, and which method is put in place to obtain a reliable prediction.

    It has also recently been noticed that signal contamination in the control regions of
    this analysis can significantly bias the data-driven aspects of the analysis. It will
    be explained how one can correct the background prediction to produce a rigorous
    interpretation of the counting experiments.

    Finally, the results and interpretations of the analysis are discussed, as well as
    some studies and possible benefits for the future of the analysis at $\sqrt{s}=13\TeV$
    using new techniques.

    \section{Monte-Carlo generation and datasets}
    %==================================

    The analysis is performed on the dataset of $pp$ collisions recorded by the CMS detector
    during the Run I of the LHC at $8\TeV$. The integrated luminosity is $\mathcal{L} =
    19.5\invfb$.

    Background samples are generated from Monte-Carlo simulations : the $t\bar{t}$ and
    single top processes are simulated using \textsc{Powheg} \cite{Powheg} while $W$+jets,
    Drell-Yan, diboson, triboson, $t\bar{t}W$ and $t\bar{t}Z$ simulations are performed
    using \textsc{MadGraph} \cite{Madgraph}. For all the background samples, the
    parton shower step is performed with \textsc{Pythia}6 \cite{Pythia} and the response of
    the detector is simulated through a \textsc{Geant4}-based model of the detector.

    Signal benchmark samples are generated according to a grid in term of $\mass{\lstop}$
    and $\mass{\lneutralino}$ with $25\GeV$ steps. The stop pair production is simulated
    using \textsc{MadGraph} with up to two additional partons generated in the hard scattering,
    and the decay of the stops and parton shower is done with \textsc{Pythia6}. The detector
    response is simulated using the CMS fast simulation package \refNeeded.

    Simulated events are weighted to match the integrated luminosity and pile-up distribution
    of the recorded collisions. Additionally, $t\bar{t}$ events are weighted as function of
    the $\pT$ of the generated top quark to correct for a known disagreement \refNeeded.
    Finally, we also reweight signal events to correct for a mismodeling of initial-state
    radiation observed in the recoil of $Z$+jets and $t\bar{t}$ events \cite{ISRmodelingDominick}.

    \section{Objects and events selection \label{sec:analysis_objectAndEventSelection}}
    %==============================================================

    This section focuses on the first aspect of the experimental search, namely the objects
    and events selection. The goal here is to, first, present the criteria used to define the
    objects in the context of this analysis, and secondly, the baseline event selection based
    on the signature being looked for.

        \subsection{Trigger}
        %==============================================================

    The data used in this analysis are recorded from three triggers \todo{TRG-12-001}, that requires the
    presence of a muon or an electron in the event.

    The single muon trigger requires an isolated muon candidate with $\pT > 24\GeV$
    with a relative isolation lower than 0.15 in a cone of $\Delta R = 0.3$. With the selection
    of this analysis, the efficiency of this trigger ranges from 78\% to 95\% depending
    on the $\pT$ and $\abseta$ of the muon. In addition to this single muon trigger, a
    muon+jets trigger is used, allowing a lower $\pT$ threshold for the muon. This trigger
    requires a muon candidate with $\pT > 17\GeV$ and $\abseta < 2.1$ and at least three
    jets with $\pT > 30\GeV$ and $\abseta < 3.0$. In the context of this analysis, the
    efficiency of this trigger is essentially dependent of the muon $\pT$ and $\eta$ and
    varies from 76\% to 97\% for a $\pT > 20\GeV$.

    The single electron trigger requires an electron candidate with $\pT > 27\GeV$. Because
    the reconstruction of electron is more challenging and subject to fakes than muons,
    criterias are applied on the shower shape, the matching between track and supercluster,
    and the ratio between hadronic and electromagnetic energy. In the context of this analysis,
    the efficiency of this trigger ranges from 86\% to 97\% depending on the $\pT$ of the
    electron.

    The analysis makes also use of dilepton triggers to later control the dileptonic
    $t\bar{t}$ component. For these, electrons candidates are identified using loose
    requirements on the isolation and information from tracker and calorimeters. The
    leading lepton must have $\pT > 17\GeV$ while the second lepton must have $\pT > 8\GeV$.

        \subsection{Leptons}
        %==============================================================

    After full reconstruction of the events, further criteria are applied on the lepton
    candidates. In the context of this analysis, we define two categories of leptons :
    the first one, called \emph{selected} leptons, corresponds to well-identified high-$\pT$
    isolated leptons ; the second one is a category meant to be used to veto events with
    a lost second lepton, and is constructed from the particle-flow candidates directly,
    with loose requirements.

            \subsubsection{Selected leptons}
            %==============================================================

        To enter the selected lepton category, muon candidates are requested to have
    $\pT > 20\GeV$ and $\abseta < 2.1$ as well as a good vertex compatibility, good fit
    quality for the track and a minimum number of hits in both the tracker and the muon
    subdetectors using the tight identification working point of the collaboration
    \refNeeded.
        Electron candidates are requested to have $\pT > 30\GeV$, to be in the barrel
    ($\abseta < 1.4442$) and have good vertex compatibility, low number of missing hits
    and low amount of radiation in the tracker using the medium identification working
    point of the collaboration
    \refNeeded.

    For both muons and electrons, a relative isolation is computed inside a cone of
    $\Delta R = 0.3$ around the lepton. The absolute isolation, $i_\text{abs}$, is computed
    using the particle-flow information by summing the $\pT$ of charged particles inside
    the cone, as well as neutral particles. An estimation of the neutral pile-up
    contribution is substracted from the neutral component, using an effective-area scheme
    for electrons and $\Delta \beta$ scheme for muons. The relative isolation, $i_{rel}
    \definedAs i_{abs} / p_T(\ell)$ is required
    to be lower than 0.15 while the absolute isolation is required to be lower than $5\GeV$.

        \subsubsection{Leptons for second lepton veto \label{sec:vetoLeptons}}
        %==============================================================

    As announced in \refsection{sec:analysis_overview}, the dileptonic $t\bar{t}$
    process becomes a major background of this analysis after cutting on the variable $\MT$.
    This motivated the development of efficient ways to reject this specific background,
    in particular via a veto targetting events with a 'lost' lepton.

    To characterize the problem, the dileptonic $t\bar{t}$ events passing a selection
    requiring exactly one selected lepton and at least three jets are classified according
    to the nature and kinematics of the lost lepton taken from the Monte-Carlo truth. Five
    categories are considered :
    \begin{itemize}
        \item ($e/\mu$) Electrons or muons with $\pT > 5\GeV$, $\abseta < 2.5$ ;
        \item ($\tau \rightarrow e/\mu$) Taus decaying to an electron or muon with $\pT > 5\GeV$, $\abseta < 2.5$ ;
        \item (1-prong $\tau$) Taus decaying to one charged hadron with $\pT > 10\GeV$, $\abseta < 2.5$ ;
        \item ($\geq$ 3-prong $\tau$) Taus decaying to three or more charged hadrons
              with total visible energy $> 20\GeV$, $\abseta < 2.5$ ;
        \item (Not in acceptance) Other cases fall in this category as their reconstruction
              is considered too challenging.
    \end{itemize}

    \reffig{fig:secondLeptonVeto/ttllComposition/initial} shows a diagram representing
    the contribution of each category to the dileptonic $t\bar{t}$ events. To address all
    the categories (appart from the lepton not in acceptance), two vetos are designed.

    \insertFigure{secondLeptonVeto/ttllComposition/initial}{0.5}
                 {Nature of the lost lepton in dileptonic $t\bar{t}$ after a selection
                 requiring exactly one high-$\pT$ selected lepton and at least three
                 high-$\pT$ jets}

    The first veto category targets the $e/\mu$, $\tau \rightarrow e/\mu$ and
    1-prong $\tau$ categories. We look for a reconstructed isolated particle in the event,
    discarding those that are within $\Delta R < 0.1$ of an already selected lepton. To
    remove fakes from pile-up activity, we require the track of the particle to be
    compatible with the primary vertex with $d_z < 0.05$ cm. A different treatment is then
    applied on the particle depending if it is flagged or not as electron or muon by the
    particle-flow algorithm. If the particle is flagged as an electron or muon candidate,
    it is required to have $\pT > 5\GeV$ and a relative isolation lower than 0.2.
    If the particle is not flagged as electron or muon candidate, the $\pT$ requirement
    is tighten to $10\GeV$, the relative isolation must be lower than 0.1, and its charge
    must be opposite to the one of an already selected lepton.
    \reffig{fig:isoTrackVeto_distributions} shows the distribution of $d_z$ and the
    relative isolation before cutting on these variables. Particle-flow candidates are categorized depending if
    they are flagged as $e/\mu$ candidate by the particle flow, and if they are matched to a
    generated lepton or not.

    \insertTwoFigures{isoTrackVeto_distributions}
                     {secondLeptonVeto/isoTrack/pdf/1DSuperimposed/singleLepton/passID/dz}
                     {secondLeptonVeto/isoTrack/pdf/1DSuperimposed/singleLepton/passID/relIso}
                     {0.45}
                     {Distribution of $d_z$ (on the left) and relative isolation (on the right)
                     for the particle-flow candidates. A cut on $\pT > 5\GeV$ for candidates
                     flagged as $e/\mu$ is applied while $\pT > 10\GeV$ is required for
                     the other candidates (non-identified).}

    The second veto category targets a $\tau$ lepton that decayed hadronically
    into one or more charged hadron. $\tau$ candidates are reconstructed using the
    hadron-plus-strips (HPS) algorithm \todo{explain principle} \refNeeded. To reject fake
    $\tau$'s, we use a discriminator based on a multivariate analysis of the parameter and
    topology of the jet. The medium working point is used, leading to a tagging efficiency
    around $70\sim80$\% for a fake rate of about 1\% as shown on \reffig{fig:tauVeto_distributions}
    on the left. The $\tau$ candidates must be separated from an already selected lepton *
    by $\Delta R > 0.4$ and to be oppositely charged. In addition, to reject fakes at low
    $\pT$ as shown on \reffig{fig:tauVeto_distributions} on the right, we require the $\tau$
    candidate to have $\pT > 20\GeV$.

    \insertTwoFigures{tauVeto_distributions}
                     {secondLeptonVeto/tau/taggingEfficiency}
                     {secondLeptonVeto/tau/pT}
                     {0.45}
                     {Tagging efficiency of the $\tau$ MVA discriminator as function of the
                     $\pT$ of the $\tau$ candidate (on the left) and $\pT$ spectrum (on the
                     right) of the fakes and candidates matched to generated leptons.}

    \reftab{tab:secondLeptonVetoPerformances} summarizes the performances of the
    second lepton vetos by showing the efficiency on the different categories estimated on
    $t\bar{t}$ events. It is important to keep an eye also to the category of events with no
    generated second lepton (i.e. semileptonic $t\bar{t}$), especially because we don't
    want to loose too much efficiency for signal events. This is what is shown in the first
    column 'no 2$^\text{nd}$ $\ell$', while the last column shows the global impact on all
    dileptonic $t\bar{t}$ event. The isolated track veto is particularly useful for rejecting
    prompt $e/\mu$, leptonically-decaying $\tau$'s and 1 prong $\tau$'s. This is complemented
    by the $\tau$ veto which covers the $\geq$ 3 prong $\tau$ case, as well as additional
    coverage of the prompt $e/\mu$ and 1 prong $\tau$.

    Overall, more than 60\% of the dilepton background is rejected when applying both
    vetoes, while only 11\% of events with no second lepton are lost. An estimation of the
    impact of this veto on the final analysis sensitivity can be computed considering the
    significance $S/\sqrt{B}$. Assuming that the dilepton $t\bar{t}$ represents $f = 75\%$ of
    the background before applying the veto, the gain on the significance is about 25\%.

    \begin{table}
    \hspace*{-1.2cm}
    \begin{tabular}{|c|c|ccccc|c|}
        \hline
        \textbf{selection}                  & no 2$^\text{nd}$ $\ell$ & not in accept. & $e/\mu$ & $\tau \rightarrow e/\mu$ & 1 prong $\tau $ & $\geq$ 3 prong $\tau$ & all 2$\ell$ \\
        \hline
        \textbf{iso. track veto}            & 0.91                    & 0.91  & 0.31  & 0.24  & 0.24  & 0.91  & 0.44  \\
        \textbf{$\tau$ veto}                & 0.97                    & 0.98  & 0.65  & 0.80  & 0.57  & 0.68  & 0.71  \\
        \hline
        \textbf{iso. track + $\tau$ veto}   & 0.89                    & 0.90  & 0.26  & 0.22  & 0.21  & 0.62  & 0.38 \\
        \hline
    \end{tabular}
        \caption{Selection effiencies when applying the two veto, estimated on the different
        category of second leptons on a $t\bar{t}$ sample. The first column presents to
        the case with no generated second lepton (i.e. semileptonic $t\bar{t}$) and is
        related to fake second leptons. The next columns detail the efficiency on each
        category of second lepton. The last columns corresponds to the global efficiency
        on the case with two generated second lepton (i.e. dileptonic $t\bar{t}$).}
        \label{tab:secondLeptonVetoPerformances}
    \end{table}

        \subsection{Jets and missing transverse energy}
        %==============================================

       Jets are reconstructed using the anti-$k_t$ clustering algorithm with a size
       parameter $R = 0.5$ on the particle-flow candidates. Three types of corrections
       are applied sequentially : an energy offset, a $\eta$ and $\pT$ dependent correction,
       and a residual correction accounting for remaining discrepancies between data and simulation.

       Selected jets are required to have $\pT > 30 \GeV$ and $\abseta < 2.4$. They also
       have to be separated from lepton candidates in $\Delta R > 0.4$ and to pass a jet
       identification criteria with loose requirements on the neutral and
       charged fractions, charged multiplicity and number of constituents \refNeeded.
       Furthermore, a pile-up identification algorithm is used which combine the vertex
       compatibility, the topology of the jet shape and the jet object multiplicity into
       a discriminant that helps to better rejects jets from pile-up \refNeeded.

       $b$-tagged jets are defined by, on top of the previous requirements, using the
       medium working point of the Combined Secondary Vertex (CSV) tagging algorithm \refNeeded.
       The efficiency of this tagger is typically around 60\% for $b$-jets while the fake
       rate of light jets is around 1\%. The value of the CSV discriminant is corrected in
       the simulation via a reshaping technique, function of the $\pT$, $\eta$ and flavor
       of the jet to account for known data/MC discrepancies \refNeeded.

       The missing transverse energy, $\vec{\MET}$, is computed by considering the
       negative vector sum of all particle-flow candidates in the event and corrected by
       propagating the previous corrections applied on the jets. Because this quantity is
       crucial in this analysis, a particular attention is given to it. Especially, not only
       the energy resolution is important, but also the $\phi$ resolution. This is why
       a correction is applied on the direction of $\vec{\MET}$ to remove a modulation seen in
       both data and simulations, function of the pile-up. The direction of this
       particle-flow-based $\MET$ is also checked to be consistent with a calorimeter-based
       approach and we veto events were the difference is higher than $1.5~\text{rad}$.
       Finally, we filters events that suffer from high noise, anomalous subdetector operation
       or known misreconstruction issues that lead to unphysical $\MET$ \cite{METperf}.

        \subsection{Events selection}
        %==============================================================

        The baseline selection, or preselection,
        is defined by asking exactly one electron or muon, at least
        four jets among which at least one is $b$-tagged, and a missing transverse energy higher than $80 \GeV$.
        A veto is applied on events containing a second lepton as defined in \refsection{sec:vetoLeptons}.
        For data, we require that events with an electron fired the single electron
        trigger, while for events with a muon, the cross-trigger has to be fired
        if the muon $\pT$ is between $20$ and $26\GeV$ otherwise the single muon trigger
        is used.
        % Justify cuts : 1 b-tag + 4 jets to remove W+jets
        % 2-btag would reduce the efficiency too much
        % MET allows to reject QCD, (W+jets and tt1l) while not being too tight for some signal
        % veto to reject dilepton events...

        We group the backgrounds in fiyr categories : $\oneLeptonTop$, $\Wjets$, $\diLeptonTop$
        and rare. The $\oneLeptonTop$ category consists of semi-leptonic $t\bar{t}$
        production and single top production ($s$ and $t$ modes). The $\Wjets$
        category is the production of a $W$ decaying leptonically, associated with the
        production of several jets from initial or final state radiation. The $\diLeptonTop$
        category corresponds to di-leptonic $t\bar{t}$ production. Finally, the rare
        category regroups several different processes from Drell-Yan, diboson, triboson and
        $t\bar{t}$+boson. The QCD background is not considered as its contribution has been
        found to be negligible.

        \reftab{tab:cutflowPreselection} presents a breakdown of the yields of the backgrounds
        at different steps of the preselection. \reffig{fig:selectionEfficiency} shows
        the selection efficiency of the signal accross the $(\mass{\lstop}, \mass{\lneutralino})$
        plane for the four signal scenarios which are studied.

        % Discuss efficiency of cuts

        \begin{table}[h!]
            \hspace*{-0.3cm}
            \begin{tabular}{|c|cccc|}
                \hline
                                             & =1$\ell$, $\geq 4$ jets   & +$\geq 1b$-tag     & +$\MET > 80 \GeV$ &  +2$^\text{nd}\, \ell$ veto \\
                                             &                           &                    &                   & (preselection) \\
                \hline
                $\oneLeptonTop$              & 253909 $\pm$ 211          & 212568 $\pm$ 193   &  61066 $\pm$ 100  & 54036 $\pm$ 94     \\
                $\diLeptonTop$               &  26240 $\pm$ 67           &  22193 $\pm$ 61    &  11235 $\pm$ 43   &  4169 $\pm$ 26     \\
                $\Wjets$                     & 128327 $\pm$ 239          &  18224 $\pm$ 94    &   4791 $\pm$ 44   &  4460 $\pm$ 43     \\
                rare                         &  41243 $\pm$ 102          &  16630 $\pm$ 79    &   4925 $\pm$ 45   &  3835 $\pm$ 40     \\
                \hline
                total SM                     & 449720 $\pm$ 342          & 269616 $\pm$ 237   &  82019 $\pm$ 127  & 66502 $\pm$ 115    \\
                \hline
$\lstop \rightarrow t \lneutralino$   (450/50) & 341 $\pm$ 7               & 289 $\pm$ 7        & 263 $\pm$ 6       & 224 $\pm$ 6        \\
$\lstop \rightarrow b \lchargino$ (0.5/450/50) & 398 $\pm$ 21              & 356 $\pm$ 20       & 306 $\pm$ 19      & 248 $\pm$ 16       \\
                \hline
            \end{tabular}
            \caption{Breakdown of the yields for each background categories, at different stages of the selection. Uncertainties are statistical only.}
            \label{tab:cutflowPreselection}
        \end{table}

        \insertFourFigures{selectionEfficiency}
                          {selectionEfficiency/T2tt}
                          {selectionEfficiency/T2bw-075}
                          {selectionEfficiency/T2bw-050}
                          {selectionEfficiency/T2bw-025}
                          {0.45}
                          {Preselection efficiency of the signal accross the $(\mass{\lstop},
                          \mass{\lneutralino})$ space for the $\lstop \rightarrow t \lneutralino$
                          decay mode (top left) and the $\lstop \rightarrow b \lchargino$
                          decay mode for $x = 0.75$ (top right), 0.50 (bottom left) and 0.25
                          (bottom right). The efficiency is computed with respect to the
                          inclusive process (i.e. not only to the $1$-lepton channel which
                          correponds to a branching ratio of about 44\%).}


    \section{Signal region design and optimisation \label{sec:analysis_optimization}}
    %==============================================================

    At the preselection level, the dominant background category is the $\oneLeptonTop$.
    For the $\oneLeptonTop$ and $\Wjets$ backgrounds, the only source of genuine missing
    energy is a neutrino $\nu_\ell$ coming from the leptonic decay of a $W$ boson ($W
    \rightarrow \ell \nu_{\ell}$). In comparison, the signal has three main sources of
    genuine missing energy which are a neutrino and the two neutralinos.
    One can exploit this difference by introducing the transverse mass of the ($\ell$,
    $\vec{\MET}$) system, called $\MT$ and defined as

    \eq{MT_definition}
    {
        \MT
        \definedAs
        m_T(\vec{p}(\ell),\vec{\MET})
        =
        \sqrt{2 \MET \cdot \pT(\ell) \cdot ( 1 - cos( \Delta \phi  ) ) }
    }

    where $\Delta \phi$ is the azimuthal angle between the lepton and the $\vec{\MET}$
    directions. This variable has a kinematic end at $\mass{W} \sim 80\GeV$ for $\oneLeptonTop$
    and $\Wjets$ background, while the same variable can get to higher values for the signal
    because of the additional missing energy coming from the two neutralinos.

    \reffig{fig:MTatPreselection} shows how $\MT$ distributes for the different
    backgrounds and two signal benchmarks at preselection level. Despite the sharp drop after
    $\MT \sim 80\GeV$, the $\oneLeptonTop$, $\Wjets$ and rare components still contribute
    in the tail because of $\vec{\MET}$ resolution effects and off-shell $W$ contributions.
    The $\diLeptonTop$ background however has a no kinematic end because of its second
    neutrino contributing to $\vec{\MET}$.

    \insertTwoFigures{MTatPreselection}
                     {variables/MTsuperimposed}{variables/MTstack}{0.45}
                     {Distribution of $\MT$ for the different backgrounds,
                     superimposed after normalization to one (on the left) or
                     stacked after normalization to the luminosity (on the right).
                     Two signal examples are shown, with their cross-sections multiplied
                     by 100 on the right.}

    Given the discriminating power of this variable, cutting on it is the
    starting point of all the signal region definitions of the analysis.
    Two approaches are used to define the signal regions. The first approach is a
    cut-based approach which consists in applying a sequential list of cuts on variables,
    followed by a counting experiment using the expected yields of background
    and signal. The second approach uses a boosted decision tree (BDT) that combines multiple inputs
    into a single discriminating variable, the BDT output, on which a cut is applied to
    also perform a counting experiment. While the BDT approach exploits by design more
    phase space than the cut-based approach given the same information, and therefore is
    expected to have better performances, the cut-based approach is often seen as a more
    transparent tool as it offers the possibility to control one by one the effect of each
    cuts.

    For the cut-based approach, we use $\MT > 120 \GeV$ as a starting point
    whereas for the BDT approach we use $\MT > 100 \GeV$ to allow more phase space for the
    training and higher initial selection efficiency for the signal. \reftab{tab:MTcutImpact}
    shows the breakdown of the yields after the $\MT$ cut for the different backgrounds and two signal benchmarks.
    In particular for $\MT > 120\GeV$, one can notice how 96\% of the $\oneLeptonTop$ and
    $\Wjets$ contributions is rejected while about 65\% of the signal is conserved. The
    $\diLeptonTop$ category is the less impacted background and representabout
    42\% of the total background after the cut.

    \begin{table}[h!]
        \centering
        \begin{tabular}{|c|ccc|}
            \hline
                          & Preselection       & +$\MT > 100 \GeV$   & +$\MT > 120 \GeV$     \\
            \hline
        $\oneLeptonTop$   & 54036 $\pm$ 94     &  5970 $\pm$ 31      &  1663 $\pm$ 16       \\
        $\diLeptonTop$    &  4169 $\pm$ 26     &  2117 $\pm$ 18      &  1529 $\pm$ 16       \\
        $\Wjets$          &  4460 $\pm$ 43     &   477 $\pm$ 13      &   170 $\pm$ 8        \\
        rare              &  3835 $\pm$ 40     &   490 $\pm$ 13      &   233 $\pm$ 9        \\
            \hline
        total SM          & 66502 $\pm$ 115    &  9055 $\pm$ 41      &  3596 $\pm$ 26       \\
            \hline
$\lstop \rightarrow t \lneutralino$   (450/50) & 224 $\pm$ 6         & 160 $\pm$ 5   & 146 $\pm$ 5   \\
$\lstop \rightarrow b \lchargino$ (0.5/450/50) & 248 $\pm$ 16        & 167 $\pm$ 14  & 146 $\pm$ 13  \\
            \hline
        \end{tabular}
        \caption{Breakdown of the yields for each background categoriesi and two signal
        benchmarks, after cutting on $\MT > 100 \GeV$ and $\MT > 120 \GeV$. Uncertainties
        are coming only from the statistics of the Monte-Carlo samples.}
        \label{tab:MTcutImpact}
    \end{table}

    In the following, we first present the discriminating variables that are used to design
    the signal regions, classified according to their nature, then we detail the
    sensitivitiy estimation and optimization procedure.

    \subsection{Discriminating variables \label{sec:analysis_variables}}
        %==============================================================

        This section describes the variables which are later used to define the
        signal regions. \reffig{fig:variables} intend to represent the discriminating
        power of the variables by comparing the background to two signal examples. However,
        it should be kept in mind that some of them show their usefulness only after
        cutting on other variables or in specific regions of the $(\mass{\lstop},\mass{\lneutralino})$ space.

           \subsubsection{$\MET$ and $\MET$ significance}

        While the preselection already includes a cut on the missing transverse energy,
        cutting further on it can significantly increase the signal-to-noise ratio especially
        since the mean $\MET$ is expected to grow as a function of $\deltam$ as shown
        in \reffig{fig:phenoMET}. However, signal
        at low $\deltam$ is more challenging because of the lower $\MET$. To address this
        particular region of the ($\mass{\lstop}$,$\mass{\lneutralino}$) space, one
        can turn to the significance of $\MET$ \cite{METperf, METsignificanceMirman}. The
        uncertainty on $\MET$ is, in the gaussian approximation, $\sqrt{\sum_\text{particles} E_T}$,
        which can be approximated by $\sqrt{H_T}$ with $H_T \definedAs \sum_\text{jets} p_T$.
        The variable $\MET / \sqrt{H_T}$ can therefore be used as an approximation of the
        real $\MET$ significance. Overall, it effectively provides a better discriminating
        power the regular $\MET$ at low $\deltam$. This can be qualitatively understandood
        by considering that some background events can be selected due to large $\MET$ coming
        from mismeasurement due to high hadronic activity, and will therefore be have
        a low $\MET/\sqrt{H_T}$ value. On the contrary, signal with relatively low $\MET$
        can be recovered provided that the event contains a comparably low level of hadronic
        activity, making it more likely that the $\MET$ is essentially genuine.
        \reffig{fig:METvsSqrtHT} illustrates this fact by showing the distribution of
        $\MET$ versus $\sqrt{H_T}$ for the $\oneLeptonTop$ background and a benchmark, and
        the line corresponding to a cut at $\MET / \sqrt{H_T} < 10$.

        \insertTwoFigures{METvsSqrtHT}
                         {variables/METvsSqrtHT_1ltop}{variables/METvsSqrtHT_T2tt}{0.45}
                         {Distribution of $\MET$ vs $\sqrt{H_T}$ for the the
                         $\oneLeptonTop$ (on the left) and the benchmark $\lstop \rightarrow
                         t \lneutralino$ (on the right).
                         The line illustrates a cut $\MET / \sqrt{H_T} < 10$.}

        \subsubsection{$M_{T2}^{W}$}

        $\MT$-like variables have been developed to target cases with two sources of genuine
        missing energy in the event. \todo{ref:hep-ph/1203.4813} These extensions are often called $M_{T2}$. Here, we make use of
        $M_{T2}^W$ which is specifically designed for the topology of the $\diLeptonTop$ background, with
        one lepton missing. $M_{T2}^W$ is designed by decomposing $\vec{\MET}$ into two
        components : one corresponding to the lost leptonic $W_1$ and the other one being
        a neutrino $\nu_2$, such that $\vec{p}_{W_1} + \vec{p}_{\nu_2} = \vec{\MET}$.
        The momentums must fit int (on the right)o the constrains that each $W$-like systems, $W_1$ and
        ($\ell + \nu_2$), should have a mass $m_W \sim 80 \GeV$. Finally, it is imposed that
        $m(b_1 + W_1) = m(b_2 + \ell_2 + \nu_2) \definedAs \mass{Y}$. The computation is
        summarized in \refequation{eq:MT2W_definition}
        The resulting value for $M_{T2}^W$ is the minimum value of $\mass{Y}$ found accross
        all possible decomposition of $\vec{\MET}$. For the $\diLeptonTop$ background,
        this variable has an end point at $m_t \sim 172\GeV$ while the signal can get to
        larger values because it contains three sources of $\MET$.

        \insertFigure{MT2W}{0.3}{Sketch of the object naming in the context of
                                 $M_{T2}^{W}$ and assuming that the event follows the topology
                                 of a dileptonic $t\bar{t}$ event with one lepton not
                                 reconstructed.}

        \eq{MT2W_definition}
        {
            M_{T2}^{W}
            =
            \text{min}
            \left\{
            \text{$M_y$ consistent with:}
                \left[
                    \begin{array}{r}
                    \vec{p}_1^T + \vec{p}_2^T =  \vec{E}_T^{mis}, p_1^2 = 0, \left(p_1 + p_l \right)^2 = p_2^2 = M_W^2, \\
                    \left(p_1 + p_l +  {p_{b}}_{1}  \right)^2 = \left(p_2 + {p_{b}}_{2} \right)^2 =M_y^2
                    \end{array}
                \right]
            \right\}
        }


        \subsubsection{Hadronic top $\chi^{2}$}

        An other way to reject the $\diLeptonTop$ background, complementary to the approach
        offered by $M_{T2}^{W}$, is to try to reconstruct an hadronic top from the jets
        in the event. It can be done by finding the best couple of jets, among selected jets,
        that fits into the constrain of the mass of the $W$ and then to the mass of the top
        by adding a third jet. The value of the hadronic top $\chi^2$ is defined as
        \eq{hadronicChi2definition}
        {
            \chi^2_\text{hadronic top}
            =
            \frac{(\mass{j_1 j_2 j_3} - \mass{t})^2}{\sigma^2_{j_1 j_2 j_3}}
            +
            \frac{(\mass{j_1 j_2} - \mass{W})^2}{\sigma^2_{j_1 j_2}}
        }
        where the $\sigma$ are the uncertainties on the mass of the jet systems by
        propagating the jet energy resolution. This value is in particular relevant to
        discriminate signal events, expected to contain an hadronic top, from $\diLeptonTop$
        and some processes of the rare category, expected to not contain an hadronic top.

        \subsubsection{$\pT(\text{leading } b)$}

        The $\pT$ of the selected objects are a natural source of discriminating power as
        the signal is expected to have a larger momentum than the background on average.
        However for the $\lstop \rightarrow b \lchargino$ decay mode, the $\pT$ of the
        $b$-jet is expected to be high (compared to the typical $\pT$ of $b$-jets in the
        background) as it originates directly from the decay of the stop. It is therefore
        interesting for the low $x$ cases, i.e. where the gap between $\lstop$ and
        $\lchargino$ is set to be high.

        \subsubsection{ISR-tagged jets}

        As described in \refsection{sec:analysis_contextAndPheno}, the low $\deltam$,
        off-shell and stealthy regions are challenging because looking quite like the SM
        $t\bar{t}$ production, if not having lower $\pT$ decay products. One approach to
        work around this issue is inspired by direct dark matter production searches and
        other searches for new physics involving soft objects. It consists in noticing that
        the production of heavy particles, in our case stops, requires more energetic
        partons compared to lighter objects, in our case tops. The rate and energy of
        initial-state radiation (ISR) grows according to the energy of the incoming partons.
        For instance, the probability of transition from a quark to a quark+gluon, as
        function of the fraction $x$ of energy carried by the quark compared to the proton,
        is

        \eq{ISRprobability}
        {
            p_{q \rightarrow qg}(x) = \frac{4}{3} \frac{1+x^2}{1-x}
        }

        which becomes closer to 1 as $x$ increases. One can conclude that the presence and
        the energy of an ISR in the event can allow to discriminate between the $t\bar{t}$
        background and a low $\deltam$ signal. Moreover, the production of an ISR can enhance
        the magnitude of the $\MET$ in the event as the system is recoiling in the opposite
        direction of the ISR, as sketched on \reffig{fig:ISRtagging/ISRsketch} for a
        SUSY signal. This is further illustrated on \reffig{fig:ISRpheno} showing, at
        $\deltam = 125\GeV$, the evolution of the mean stop pair recoil as function
        of $\mass{\lstop}$, as well as the evolution of the mean generated $\MET$ as function
        of the stop pair recoil.

        \insertFigure{ISRtagging/ISRsketch}{0.6}{Topology of a SUSY signal without (A, on the
        left) and with (B, on the right) an ISR jet, boosting the initial pair of SUSY
        particles in the opposite direction. The case with ISR jet leads to increased
        $\MET$ coming from the decay of the SUSY particles. \todo{ref to 0803.0019}}

        \insertTwoFigures{ISRpheno}
                         {ISRtagging/mStop_vs_meanStopPairRecoil}
                         {ISRtagging/stopPairRecoil_vs_meanGenMET}
                         {0.49}
                         {On the left, mean recoil of the stop pair system as function
                         of the stop mass.
                         On the right, mean $\MET$ as function of the stop pair system recoil.
                         The signal benchmarks use here have $\deltam = 125\GeV$ and a
                         preselection requiring at least 1 high-$\pT$ electron or muon,
                         and at least three jets has been applied.}

        To take advantage of this phenomenological aspect, one can design criterias to
        select jets that are likely to originate from ISR \todo{Ref to 0803.0019, 1101.0810}.
        In the present analysis, we define the ISR-tagging criteria as asking that
        the event contains at least five jets, among which one of them with $\pT > 200 \GeV$
        and not $b$-tagged.

        \subsubsection{$\Delta \phi( j_{1,2}, \vec{\MET} )$ }

        A topological difference between the signal and the background is the correlation
        between the direction of the $\MET$ and the two leading jets, $j_1$ and $j_2$.
        $\Delta \phi( j_{1,2}, \vec{\MET} )$ is defined as the minimum of $\Delta \phi$
        between the $\MET$ direction and the leading jet, and $\Delta \phi$ between the
        $\MET$ direction and the next-to-leading jets :

        \eq{deltaPhij1j2METDefinition}
        {
            \Delta \phi( j_{1,2}, \vec{\MET} )
            \definedAs
            \text{min}
            (
                \Delta \phi( j_1, \vec{\MET} )
                ,
                \Delta \phi( j_2, \vec{\MET} )
            )
        }

        The
        $t\bar{t}$ background tends to get low values for this variable because the
        $\nu_{\ell}$ direction is linked to the $b$ quark from the top parent. In signal
        events, $\MET$ is less correlated to the $b$-jet and can $\Delta \phi( j_{1,2},
        \vec{\MET} )$ easily get to higher values.

        \subsubsection{Other variables used in the BDT approach}

        Other variables are used for the BDT approach and are more briefly discribed in
        this section :

    \begin{itemize}
        \item $\pT(\ell)$, $\pT(\text{lead. jet})$, $\pT(\text{lead. } b)$ ; The $\pT$ of the
            selected objects are a natural source of discriminating power as the signal is expected
            to have a larger momentum than the backgrounds. For both the cut-based and BDT approach,
            $\pT(\text{lead. } b)$ is especially a good variable for the $\lstop \rightarrow b \lchargino$
            decay mode as the $b$ quarks are expected to have large momentum because they are direct
            decay product of the stop. For the BDT approach, $\pT(\ell)$ and $\pT(\text{lead. jet})$
            are considered because of the different of correlation they have with other variables
            compared to the background.
        \item $H_T$ ; In a similar fashion, one can look at $H_T \definedAs \sum_{\text{jets}} \pT$.
            This variable is expected to get a larger average value for the signal compared to standard
            $t\bar{t}$ production because of the contribution of the stop rest mass to the momentum of
            the decay products.
        \item $\Delta R( \ell, \text{lead. } b)$ ; In the $\lstop \rightarrow t \lneutralino$ decay mode,
            as $\deltam$ grows, the top quarks become more and more boosted and their decay products are
            more colimated in one single direction. This variable aims to exploit this fact, expecting more
            signal events at lower $\Delta R$. It however offers less discriminating
            power for the $\lstop \rightarrow b \lchargino$ as
        \item $H_{T}^\text{ratio}$ ; Another way of using the $\MET$ direction is to compute the ratio of the
            hadronic activity in the same hemisphere as the $\MET$ compared to the total hadronic activity of the event,
            $H_T$. Because the visible energy recoils on the LSP in signal events, this variable tends to have
            low values for signal while being around 0.5 for background.
        \item $M_{\ell b}$, $M'_{\ell b}$ ; We consider also the invariant mass of the $\ell$+leading $b$-jet system. This variable
            is a simple attempt to reconstruct the mass of the leptonic top system despite missing
            the information of the neutrino. This variable is useful in the search for $\lstop \rightarrow
            b \lchargino$ as there is no end point at $m_t$ for this variable because there is no
            top in the decay chain and larger values can be obtained for signal events. An extension
            of this variable to the case where there is no $b$ in the event, $M'_{\ell b}$, consist
            to use the jet with the highest $b$-tagging discriminator value.
        \item $M_{3b}$ ; In a similar way, one can attempt to reconstruct the mass of the hadronic top system by considering the
            three jets most back-to-back to the selected lepton. Here again, values larger than $m_t$ can be expected in
        \item Jet multiplicity ; The jet multiplicity also tends to provide discriminating power.
            This is related to the previous ISR discussion : in a more general case, signal
            at higher $\deltam$ will contain more jets because of ISR. The BDT is likely to
            be able to exploit the correlation between this variable and the others to
            increase its discriminating power.
    \end{itemize}

    \begin{figure}[h!]
        \centering
        \includegraphics[width=0.32\textwidth]{variables/leadingBPt}
        \includegraphics[width=0.32\textwidth]{variables/HT}
        \includegraphics[width=0.32\textwidth]{variables/deltaRLeptonB}\\
        \includegraphics[width=0.32\textwidth]{variables/MET}
        \includegraphics[width=0.32\textwidth]{variables/METoverSqrtHT}
        \includegraphics[width=0.32\textwidth]{variables/deltaPhiMETJets}\\
        \includegraphics[width=0.32\textwidth]{variables/HTratio}
        \includegraphics[width=0.32\textwidth]{variables/MT2W}
        \includegraphics[width=0.32\textwidth]{variables/HadronicChi2}\\
        \includegraphics[width=0.32\textwidth]{variables/Mlb_hemi}
        \includegraphics[width=0.32\textwidth]{variables/M3b}
        \includegraphics[width=0.32\textwidth]{variables/leadingNonBPtN5}
        \caption{Stacked plots representing the different discriminating variables used to design the signal region, for background and two signal examples after preselection and cut on $M_T > 100 \GeV$.
        \todo{update plot with missing variables and/or use same benchmark as previously, ie 450/50}}
        \label{fig:variables}
    \end{figure}

    \subsection{Figure of merit and sensitivity estimation \label{sec:FoMdiscussion}}
    %==============================================================

    Before going into the details of the signal region optimization, we first need to
    introduce the metric used to do so. The problem of defining cuts to be applied on a
    variable can be summarized as knowing how to compromise between the quantity of
    selected signal, $S(\text{cut})$, versus selected background, $B(\text{cut})$, such
    that the sensitivity of the analysis is maximized. Ideally, one would run the full
    statistical interpretation which, in the context of the LHC experiments, is based
    on the $CL_S$ approach \refNeeded. However, such a procedure is CPU intensive as it is
    requires toy-data generation, and not suitable for a highly iterative process such
    as scanning possible cuts. In the case of a single-bin counting experiment, a more
    flexible way is to use analytical formula, called figure of merit (FoM) that gives an
    immediate estimation of the senstivity of the counting experiment.

    Let's consider the background only hypothesis $H_0$, also called null hypothesis,
    and the signal hypothesis $H_1$. These hypothesis are modeled by a probability density
    function (pdf), describing the probability to observe $N$ events in the data, as
    sketched on \reffig{fig:interpretation}.

    \insertFigure{interpretation}{0.7}
                 {Illustration of the modelization of the hypothesis $H_0$ and $H_1$ with
                 Poisson distributions of means and standard deviations being respectively
                 $(B,\sqrt{B})$ and $(S+B,\sqrt{S+B})$. \todo{fix plot}}

    From the point of view of excluding the signal hypothesis, a statistical hypothesis
    test consists in computing the probability $p$ to observe less than $D$ events in the
    data under $H_1$, $p(N < D|H_1)$. If this probability is lower than a threshold
    $\beta$, one may exclude the signal hypothesis with a confidence level $1-\beta$. A
    common practice is to use a confidence level of 95\%, or $\beta = 5\%$. The exclusion
    potential of a counting experiment can be defined via the probability $p(N < E[H_0]|H_1)$,
    i.e. the probability to observe a background-like realization in the data, if the
    signal exist. It is sometimes more pratical to express this potential in term of a
    significance $\mathcal{S}$, that is to say by expressing the distance $E[H_0] -
    E[H_1]$ in terms of standard deviations $\sigma[H_1]$ :

    $$ \mathcal{S}_\text{exclu.} = \frac{E[H_0] - E[H_1]}{\sigma[H_1]}$$

    If $H_1$ follows a gaussian distribution, then one can reinterpret $\mathcal{S}$ in terms
    of gaussian probabilities, for instance $\mathcal{S} = 2$ as a probability $p\approx5\%$
    using the so-called 68-95-99.7 rule of thumb in statistics. Considering that $H_0$ and
    $H_1$ have means and standard deviations being respectively $(B,\sqrt{B})$ and
    $(S+B,\sqrt{S+B})$ as in \reffig{fig:interpretation}, one ends up with

    $$ \mathcal{S}_\text{exclu.} = \frac{S}{\sqrt{S+B}}$$

    The same reasonning can be applied to the point of view of a discovery claim, where
    one is interested in $p(N > E[H_1]|H_0)$, i.e. the probability to observe a
    signal-like realization in the data, if the signal doesn't exist. Assuming the situation
    sketched in \reffig{fig:interpretation}, one obtains a significance

    $$ \mathcal{S}_\text{disc.} = \frac{S}{\sqrt{B}}$$

    This significance definition can be easily used as a figure of merit when optimizing
    cuts. The picture may be completed by incoporating systematic uncertainties on the
    background to $\sigma[H_0]$ and $\sigma[H_1]$, leading to :

    \begin{equation}
       \mathcal{S}_\text{disc.} = \frac{S}{\sqrt{B+f^2 \cdot B^2}}
       \hspace*{2cm}
       \mathcal{S}_\text{exclu.} = \frac{S}{\sqrt{S+B+f^2 \cdot B^2}}
       \label{FoM}
   \end{equation}

    where $f$ represents an estimate of the relative systematic uncertainty on the background.

    While this figure of merit sometimes gives a reasonnable estimate of the true
    sensitivity of a counting experiment, one should remane conscious of its caveats
    \cite{Punzi} :

    \begin{enumerate}
        \item The intepretation of it is not straightfoward as it does not tell the
              physicist what can be expected to be excluded or discovered
              for a given counting experiment. It remains a 'number of sigmas', not an
              expected excluded cross-section.
        \item The formula \refequation{FoM} does not contain any constrain regarding the fact that
              the observed number of events will be an integer. It will favorize a situation
              with $0.1$ signal events and expected background of $10^{-5}$ over a
              situation with 10 signal events and 1 background event, despite the fact that
              the later will definetely bring more information.
        \item It ultimately relies on a gaussian approximation which can cause significant
              discrepancies with respect to an accurate computation of $p(N|H)$. For instance,
              The Poisson and Gaussian tail integrals are significantly different at low
              means. More specifically, $\mathcal{S}_\text{disc.}$ is known to overestimate
              the true significance while $\mathcal{S}_\text{exclu.}$ underestimates it.
              \reffig{fig:} shows a comparison of $S/\sqrt{B}$ versus an exact
              computation.
    \end{enumerate}

    Caveat \#1 can be adressed by reinterpreting $\mathcal{S}$ in terms of excludable
    (or discoverable) signal strength $\mu$ or cross-section $\sigma$. Replacing
    $S$ with $\epsilon \times \sigma \times \mathcal{L}$, and introducing the discovery
    and exclusion threshold $a$ and $b$ (typically set to 5 and 2) yields :

    \begin{equation}
        \mathcal{\sigma}_\text{disc.} = a \cdot \frac{\sqrt{B+f^2B^2}}{\epsilon \cdot \mathcal{L}}
       \hspace*{2cm}
       \mathcal{\sigma}_\text{exclu.} = \frac{b}{2} \times \frac{b + \sqrt{b^2 + 4 (B + f^2B^2)}}{\epsilon \cdot \mathcal{L}}
       \label{FoM2}
    \end{equation}

    where $\epsilon$ and $\mathcal{L}$ are the selection efficiency and luminosity.

    Caveat \#2 can be worked around by imposing a minimum number of background and signal
    events when computing the figure of merit, for replacing $B$ with $\text{max}(B,1)$
    and ignoring cases with a too low expected signal yields.

    Finally, to address \#3, Ref. \cite{Punzi}, proposes an empirical fit to adjust the
    shape of the FoM with the accurate computation, as function of the parameters
    of the FoM. Alternative significances based on other approaches of the problem also
    exist and are discussed and compared in \todo{arXiv:physics/0702156v4 and arXiv:physics/0312059}.
    \reffig{fig:FOMcomparison} shows one of them, sometimes reffered to as Asimov Z and based on a
    likelihood approach, compared to the exact computation.

    \insertFigure{FOMcomparison}
                 {0.5}
                 {(From \todo{arXiv:1007:1727}, comparison of $S/\sqrt{B}$,
                 Asimov Z significance (denoted $\sqrt{q_{0,A}}$) and the exact significance computation
                 as function of $B$ and for $S = 2$, $5$ and $10$.}

        \subsection{Cut-based signal regions}
        %==============================================================

            \subsubsection{Optimization procedure}
            %==============================================================

    To design the cut-based signal regions on the analysis, the variables are first
    classified by their individual discriminating power, estimated by taking the
    maximum FoM achievable on a few signal benchmarks when scanning the possible
    cuts. The most discriminating variables are found to be $\MT$, $\MET$,
    $\MET/\sqrt{H_T}$ and $M_{T2}^{W}$. The variables $\Delta \phi(j_{1,2},\vec{\MET})$,
    hadronic top $\chi^2$, $\pT(\text{lead. } b)$ and the 5th, ISR jet requirement are
    also found to be helpful in particular cases or after cutting on the more discriminant
    variables. However we found that variables such as $H_T^\text{ratio}$, $M_{3b}$ and
    $M_{\ell b}$ offer lower potential.

    On a few signal benchmarks, we then proceed to a $n$-dimensionnal optimization of the
    cuts on these variables. We however impose some constrain in the use of the variables.
    First, either $\MET$ or $\MET/\sqrt{H_T}$ should be used, but not both at the same time.
    We allow tighter cuts on $\MT$ compared to the starting point $> 120\GeV$, but not
    tighter than $> 140\GeV$ especially to keep enough statistics for the background
    estimation related aspects.

    The optimization of the cuts is done with respect to the exclusion-oriented figure of
    merit defined in formula \refequation{FoM}. To work around cases with very low background or signal leading,
    i.e. caveat \#2, we use $\tilde{B} \definedAs \text{max}(B,1)$ and set the FoM to 0
    if $S < 3$. The relative systematic uncertainty on the background is set to vary between
    15 and 30\% depending on the tightness of the cuts. We also incorporate some feedback
    of the background estimation that will be described later, by rescaling the $\oneLeptonTop$
    and $\Wjets$ contributions with a factor 1.3. This has for effect to favorize the
    rejection of these backgrounds over the $\diLeptonTop$ and rare components.

    After optimizing on a limited bunch of benchmarks, we inject all the sets of cuts found
    and map across the whole $(\mass{\lstop},\mass{\lneutralino)}$ space which is the one
    most performing set for each benchmark. As at this stage the number of set of cuts is
    large, and for the sake of keeping things manageable, we aim to reduce this number by
    manually clustering similar sets while keeping sure to not significantly loose in
    terms of performances.

        \subsubsection{Results and performances \label{sec:cutAndCountPerformances}}
            %==============================================================

    The resulting signal regions definitions are presented on \reftab{tab:cutAndCountCuts}.
    As announced in \refsection{sec:analysis_variables}, $\MET/\sqrt{H_T}$ tends to be
    preferred at low $\deltam$ compared to $\MET$. In the off-shell and stealthy regime,
    the ISR jet requirement plays an important role to gain sensitivity, despite the
    fact that this leaves little room for other cuts given the already low preselection
    efficiency and the tightness of requiring a 5th jet at high $\pT$.
    In the medium and high $\deltam$ regime, cutting on $M_{T2}^W$ provides a good gain
    in sensitivity because the large $\MET$ of the signal is less likely to decomposable
    in such a way that it fits the constrains while style yielding a light $m_Y$ mass.
    At low and medium $\deltam$ regimes for the $\lstop \rightarrow t \lneutralino$, the
    hadronic top $\chi^2$ provides a good alternative or complementary approach to $M_{T2}$
    for rejecting the $\diLeptonTop$ background. For $\lstop \rightarrow b \lchargino$,
    at low and medium $x$, the $\pT$ of the leading $b$ is an important feature, related
    to the high $\lstop-\lchargino$ gap. Finally, in almost all signal regions,
    $\Delta\phi(j_{1,2},\vec{\MET})$ proves to be useful by providing a way to reject the
    $\oneLeptonTop$ component where $\vec{\MET}$ is expected to be close to a ($b$-)jet.

    \reffig{fig:cutAndCountPerformances} shows the best performing set accross the
    $(\mass{\lstop},\mass{\lneutralino})$ plane as well as the corresponding FoM. While
    these results are purely based on the FoM, the final choice of the signal region to
    be used for each benchmark later done by choosing the minimum expected cross-section
    upper limit from the $CL_s$ computation.

\begin{table}[!ht]
{\footnotesize
\begin{center}
\hspace*{-0.8cm}
    \begin{tabular}{|l|ccccccc|}
    \hline
    $\lstop \rightarrow t\lneutralino$ & $\MT$   & $\MET$    & $\MET/\sqrt{H_T}$  & $M_{T2}^W$ & Hadronic top $\chi^2$ & $\Delta\phi(j_{1,2},\vec{\MET})$      &   5th, ISR jet \\
    \hline
    1) off-shell (loose)       & $>$ 125 & -       &   $>$ 8            &     -     & -             &          - &    yes        \\
    2) off-shell (tight)       & $>$ 130 & $>$ 300 &   -                &     -     & -        	    &          - &    yes        \\
    3) low    $\mass{\lstop}$  & $>$ 140 & -       &   $>$ 8            &     -     &  $<$ 5        &  $>$ 0.8   &    -          \\
    4) medium $\deltam$        & $>$ 140 & $>$ 200 &   -                &  $>$ 180  &  $<$ 3        &  $>$ 0.8   &    -          \\
    5) high   $\deltam$        & $>$ 130 & $>$ 350 &   -                &  $>$ 190  & -             &          - &    -          \\
        \hline
    \end{tabular}
    \hspace*{-0.5cm}
    \begin{tabular}{|l|ccccccc|}
    \hline
    $\lstop \rightarrow b\lchargino, x=0.25$   & $\MT$     & $\MET$    & $\MET/\sqrt{H_T}$ & $M_{T2}^W$ & $\pT(\text{lead. }b)$ & $\Delta\phi(j_{1,2},\vec{\MET})$ & 5th, ISR jet  \\
    \hline
    1) off-shell        & $>$ 120   &  -       &    $>$  9       &     -      &   -                   &  $>$ 0.2      & yes           \\
    2) low    $\deltam$ & $>$ 120   &  -       &    $>$  6       &  $>$ 200   & $>$ 180               &  $>$ 0.8      & -             \\
    3) high   $\deltam$ & $>$ 120   & $>$ 300  &     -           &  $>$ 200   & $>$ 180               &  $>$ 0.8      & -             \\
    \hline
    $\lstop \rightarrow b\lchargino, x=0.50$     & $\MT$     & $\MET$    & $\MET/\sqrt{H_T}$ & $M_{T2}^W$ & $\pT(\text{lead. }b)$ & $\Delta\phi(j_{1,2},\vec{\MET})$ & 5th, ISR jet  \\
    \hline
    1) off-shell        &  $>$ 120  &   -      &  $>$  9         &    -       & -                     &  $>$ 0.2      & yes           \\
    2) low masses       &  $>$ 135  &   -      &  $>$  6         & $>$ 180    & -                     &  $>$ 0.8      & -             \\
    3) medium $\deltam$ &  $>$ 140  &   -      &  $>$  7         & $>$ 190    & $>$ 100               &  $>$ 0.8      & -             \\
    4) high   $\deltam$ &  $>$ 120  & $>$ 300  &   -             & $>$ 200    & $>$ 100               &  $>$ 0.8      & -             \\
    \hline
    $\lstop \rightarrow b\lchargino, x=0.75$   & $\MT$     & $\MET$    & $\MET/\sqrt{H_T}$ & $M_{T2}^W$ & $\pT(\text{lead. }b)$ & $\Delta\phi(j_{1,2},\vec{\MET})$ & 5th, ISR jet  \\
    \hline
    1) low    $\deltam$ &  $>$ 120  &   -      &  $>$  12        &     -      &      -                &  $>$ 0.8      & yes           \\
    2) medium $\deltam$ &  $>$ 130  &   -      &  $>$  10        &  $>$ 180   &      -                &  $>$ 0.8      & -             \\
    3) high   $\deltam$ &  $>$ 140  & $>$ 300  &    -            &  $>$ 200   &      -                &  $>$ 0.8      & -             \\
    \hline
    \end{tabular}
\caption{Description of the signal regions defined and optimized for the cut-based approach. \label{tab:cutAndCountCuts}}
\end{center}}
\end{table}


            \begin{figure}[h!]
                \centering
                \includegraphics[width=0.37\textwidth]{cutAndCountPerformances/bestSet_T2tt}
                \includegraphics[width=0.37\textwidth]{cutAndCountPerformances/bestFOM_T2tt}\\
                \includegraphics[width=0.37\textwidth]{cutAndCountPerformances/bestSet_T2bw075}
                \includegraphics[width=0.37\textwidth]{cutAndCountPerformances/bestFOM_T2bw075}\\
                \includegraphics[width=0.37\textwidth]{cutAndCountPerformances/bestSet_T2bw050}
                \includegraphics[width=0.37\textwidth]{cutAndCountPerformances/bestFOM_T2bw050}\\
                \includegraphics[width=0.37\textwidth]{cutAndCountPerformances/bestSet_T2bw025}
                \includegraphics[width=0.37\textwidth]{cutAndCountPerformances/bestFOM_T2bw025}
                \caption{Best set of cuts (on the left) and performances in term of
                FoM$_\text{exclusion}$ (on the right) for the $\lstop \rightarrow t
                \lneutralino$ decay mode (first row) and $\lstop \rightarrow b \lchargino$
                decay mode with x = 0.75 (second row), 0.50 (third row) and 0.25 (last row).
                \todo{Maybe redo performance plot using the excluded signal strength formula+ contour at mu = 1, or if not show contour at FoM=2}}
                \label{fig:cutAndCountPerformances}
            \end{figure}

        \subsection{BDT-based signal regions}
        %==============================================================

        For the multivariate approach, several BDT are trained on slices of $\deltam$ in the $(\mass{\lstop},\mass{\lneutralino})$
        space against the $t\bar{t}$ background only. The choice of the variable is driven by an iterative method where variables
        are added to BDT and kept if the performances are overall significantly improved across different slices of $\deltam$. The
        performances of the BDTs are quantified by optimizing the cut on the BDT output with respect to a discovery-oriented
        FoM, considering all the backgrounds and assuming a relative systematic uncertainty of 15\%.

        The set of variables used are presented on \reftab{tab:BDTVariableUsage} as function of the decay mode. The definition
        of the training region in the $(\mass{\lstop},\mass{\lneutralino})$ is then being looked at, noticing that some of the
        $\deltam$ slices can be merged together as the performances of the training are similar, essentially because the kinematic
        is not strongly different when moving from one slice to the other. The optimization of the cut on the BDT output is then
        performed by iteratively looking at the excluded cross-section after the full procedure explained in the following sections.
        The cuts are tuned manually to optimize the sensitivity accordingly. Here, because of the cross-section regimes leading to
        different amount of signal statistics, it is noticed that there is sometimes a significant gain in loosening or tightening
        the cut inside a same training region.

        The final definition of the training regions is presented on \reffig{fig:BDTTrainingRegions}. Each number represents a
        different BDT training. The dashed lines represent the cases when one training region leads to several cuts applied to define
        signal regions.

        \begin{table}[h!]
            \begin{center}
                \begin{tabular}{|c|cc|}

                    \hline
                    Variable                            & T2tt      & T2tt      \\
                                                        & off-shell & on-shell  \\
                    \hline
                    $\MET$                              & $\times$  & $\times$  \\
                    $H_{T}^\text{ratio}$                & $\times$  & $\times$  \\
                    $\pT(\text{lead. } \ell)$           & $\times$  & $\times$  \\
                    $\Delta\phi(j_{1,2},\vec{\MET})$    & $\times$  & $\times$  \\
                    $N_\text{jets}$                     & $\times$  & $\times$  \\
                    \hline
                    $\pT(\text{lead. jet})$             &           & $\times$  \\
                    $\Delta R( \ell, \text{lead. } b)$  &           & $\times$  \\
                    hadronic top $\chi^2$               &           & $\times$  \\
                    $M_{T2}^W$                          &           & $\times$  \\
                    $M_{\ell b}$                        & $\times$  &           \\
                    $\pT(\text{lead. } b)$              & $\times$  &           \\
                    \hline
                \end{tabular}
                \begin{tabular}{|c|ccc|}

                    \hline
                    Variable                            & T2bw      & T2bw      & T2bw      \\
                                                        & $x=0.25$  & $x=0.50$  & $x=0.75$  \\
                    \hline
                    $\MET$                              & $\times$  & $\times$  & $\times$  \\
                    $M_{T2}^W$                          & $\times$  & $\times$  & $\times$  \\
                    $M_{\ell b}$                        & $\times$  & $\times$  & $\times$  \\
                    $M_{3 b}$                           & $\times$  & $\times$  & $\times$  \\
                    $\pT(\text{lead. } \ell)$           & $\times$  & $\times$  & $\times$  \\
                    $\Delta\phi(j_{1,2},\vec{\MET})$    & $\times$  & $\times$  & $\times$  \\
                    $N_\text{jets}$                     & $\times$  & $\times$  & $\times$  \\
                    \hline
                    $\pT(\text{lead. } b)$              & $\times$  & $\times$  &           \\
                    $\Delta R( \ell, \text{lead. } b)$  &           & $\times$  &           \\
                    $H_{T}$                             &           &           & $\times$  \\
                    $\pT(\text{lead. jet})$             &           &           & $\times$  \\
                    \hline
                \end{tabular}
                \caption{List of variables considered for the training of the BDT
                as function of the decay mode. A $\times$ mark indicate that the variable
                is used in the final trainings.}
                \label{tab:BDTVariableUsage}
            \end{center}
        \end{table}

            \insertFourFigures{BDTTrainingRegions}
                              {BDT/training_T2tt}
                              {BDT/training_T2bw075}
                              {BDT/training_T2bw050}
                              {BDT/training_T2bw025}
                              {0.49}
                              {Slicing of the $(\mass{\lstop},\mass{\lneutralino})$ space
                              to define the training region of the BDTs. Some training
                              regions are subdivised into subregions where different cuts
                              are applied on the BDT output in order to adapt the sensitivity
                              to the local cross-section. \todo{Fix printability of plots}}

    \section{Background estimation \label{sec:analysis_backgroundEstimation}}
    %==============================================================

        \subsection{Overview}
        %==============================================================

    In this section, we focus on the estimation of the different background contributions.
    Four kinds of control regions are defined by inverting some of the requirements of the
    preselection and signal regions, as illustrated on \reffig{fig:backgroundEstimationOverview}.
    Each of these aim to provide signal-free sectors in which to check how good is the
    modeling of the backgrounds by the Monte-Carlo and perform data-driven estimations.

        \insertFigure{backgroundEstimationOverview}{0.7}
                     {Overview of the control regions used in the background estimation method. \todo{Fix printability of this sketch?}}

    The $\MT$-peak control region is defined by looking at events satisfying $50 < \MT <
    80\GeV$ instead of the signal region $\MT$ requirement. This control region is enriched
    in $\oneLeptonTop$ and is used as a well-controlled region in which to normalize the
    $\oneLeptonTop$, $\Wjets$ and $\diLeptonTop$ as documented in \refsection{sec:MTpeakNormalization}.

    The $0\, b\text{-tag}$ control region is defined by requiring no $b$-tagged jet in the
    event. This region is enriched in $\Wjets$ and $\oneLeptonTop$ and is used to control
    and correct the tail of $\MT$ for these two components as described in \refsection{sec:MTtailCorrection}.

    The $2\ell$ control region is defined by requiring exactly two selected leptons instead
    of one, at least one jet, and lowering the $\MET$ cut to $50 \GeV$. Additionnally, to limit the
    contribution from Drell-Yan, we veto events where the invariant mass of the dilepton
    system, $\mass{\ell\ell}$, is such that $\left|\mass{\ell\ell} - m_{Z}\right| < 15 \GeV$.
    Finally, the reversed veto control region is defined by requiring exactly one selected
    lepton and reversing the second lepton veto, effectively asking for an isolated track
    or $\tau$ as defined in \refsection{sec:vetoLeptons}. This region  is intended to control
    the modeling of the second lepton veto. Both the $2\ell$ and reversed veto are being
    looked at on the whole $\MT$ range and are not used to derive scale factors but only
    systematic uncertainties on the background as detailed in \refsection{sec:background_systematics}.

    \reftab{tab:cutflowControlRegions} shows a breakdown of the background contributions
    in the different control regions at the preselection level.

\begin{table}[h!]
    \centering
\begin{tabular}{|c|cccc|}
    \hline
                     & $\MT$-peak       & $0\, b\text{-tag}$ ($M_T$ tail) & reversed veto    & 2 leptons             \\
    \hline
     $\oneLeptonTop$ & 18523 $\pm$ 55   &  1213 $\pm$ 14        &  7030 $\pm$ 34   &   41 $\pm$ 2.7  \\
     $\diLeptonTop$  &   656 $\pm$ 10   &   382 $\pm$ 8         &  7066 $\pm$ 34   & 9211 $\pm$ 39   \\
     $\Wjets$        &  1470 $\pm$ 24   &  2669 $\pm$ 33        &   331 $\pm$ 11   &  2.1 $\pm$ 0.9  \\
     rare            &  1209 $\pm$ 23   &   198 $\pm$ 7         &  1093 $\pm$ 20   &  626 $\pm$ 15   \\
    \hline
     total SM        & 21859 $\pm$ 66   &  4462 $\pm$ 38        & 15521 $\pm$ 53   & 9882 $\pm$ 42   \\
    \hline
\end{tabular}
    \caption{Breakdown of the yield of the different background categories in the four
    control regions without applying any signal region cuts. Uncertainties are statistical only.}
    \label{tab:cutflowControlRegions}
\end{table}

        \subsection{Background normalization in $\MT$-peak \label{sec:MTpeakNormalization}}
        %==============================================================

    The $\MT$-peak control region, defined as $50 < \MT < 80 \GeV$ is the first step of the
    background estimation method. It is used to normalize the $\oneLeptonTop$, $\Wjets$ and
    $\diLeptonTop$ components while the rare component is taken directly from Monte-Carlo.
    It is important to note that this normalization is done for each signal region individually,
    effectively allowing to absorb disagreements caused by cuts on not-so-well modeled
    variables and uncertainties on the jet energy corrections, trigger efficiency,
    lepton identification efficiency and luminosity.

    To separate the effect of the second lepton veto, the  normalization is done in two
    steps : first, a scale factor $\SFpre$ is computed before the application of the
    second lepton veto, substracting the rare component :

    \begin{equation}
        \SFpre \definedAs \left( \frac{N(\text{data}) - N(\text{rare})}{N(\oneLeptonTop) + N(\Wjets) + N(\diLeptonTop)} \right)
        \label{eq:SFpreDefinition}
    \end{equation}

    $\SFpre$ is used to normalize only the $\diLeptonTop$ component. Another scale factor,
    $\SFpost$ is used after application of the second lepton veto, substracting the rare
    and the corrected $\diLeptonTop$ component :

    \begin{equation}
        \SFpost \definedAs \left( \frac{N(\text{data}) - N(\text{rare}) - \SFpre \times N(\diLeptonTop)}{N(\oneLeptonTop) + N(\Wjets)} \right)
        \label{eq:SFpostDefinition}
    \end{equation}

     When applying no signal region cuts, $\SFpre$ and $\SFpost$ are equal to
     $(1.06 \pm 0.01)$ and $(1.05 \pm 0.01)$ respectively. This values, while not
     being compatible with 1, can be interpreted as a relatively small disagreement
     which is attributed to the misknowledge of the previously listed effects. Across the
     different cut-based signal regions, these scale factors range from $0.8$ to $1.4$,
     sometimes only compatible with unity at 3 standard deviations. The magnitude of this
     effect is attributed to bad modelization of the far tail of some variables in the
     Monte-Carlo.

        \subsection{$\MT$-tail correction in the $0\, b\text{-tag}$ region \label{sec:MTtailCorrection}}
        %==============================================================

    The $0\, b\text{-tag}$ control region allows to control the tail of $\MT$ for the $\Wjets$ and
    $\oneLeptonTop$ components. Before looking at the tail, however, we start by normalizing
    the background in the $\MT$ peak of this control region, in a similar fashion to what
    is done in \refsection{sec:MTpeakNormalization}. This is done by introducing $\SFnobtag$,
    used to normalize the $\Wjets$ and $\oneLeptonTop$ contributions :

    \begin{equation}
        \SFnobtag \definedAs \left( \frac{N(\text{data}) - N(\text{rare}) - N(\diLeptonTop)}{N(\oneLeptonTop) + N(\Wjets)} \right)
        \label{eq:SF0btagDefinition}
    \end{equation}

    Without applying any signal cuts, $\SFnobtag$ is found to be $(0.99 \pm 0.01)$.
    After normalization to the peak, a clear disagreement in the tail of $\MT$ is observed
    for $\MT > 100 \GeV$ between the data and Monte-Carlo, as shown on
    \reffig{fig:templateFit/MT_notCorrected}. This is an important point of the analysis as
    it means that the Monte-Carlo needs to be corrected to have a reliable prediction of
    the $\oneLeptonTop$ and $\Wjets$ in the $\MT$ tail. \todo{Is it really a 'known'
    disagreement ... ? SMP-14-020, AN-2014/279 and AN-2014/114  see a disagreement
    for Wjets, but B2G-14-004 doesn't see anything for ttbar...}\todo{MTW sensitivity
    to MET also discussed in AN-15-013}.

    \insertFigure{templateFit/MT_notCorrected}{0.55}
                 {Data/MC comparison on the full $M_T$ distribution in the $0\, b\text{-tag}$ control
                 region at preselection level, after propagation of $\SFnobtag$. A clear
                 discrepancy is visible for $\MT > 100 \GeV$.}

    One can investigate the cause of the disagreement by first getting a better idea from
    the Monte-Carlo simulation of what is causing $\oneLeptonTop$ and $\Wjets$ events
    to each the tail of $\MT$. The resolution of $\MT$ is strongly related to both the
    resolution in energy of $\vec{\MET}$, and its $\Delta\phi$ with the lepton. The relative
    resolution of $\MET$ and $\Delta \phi$ is investigated, defined as the ratio of the
    reconstructed quantity versus the generated quantity, constructed from the generated
    prompt neutrinos in the event. \reffig{fig:MTtailResolution} shows the distribution of the relative
    resolution of $\MET$ and $\Delta \phi$ as function of $\MT$ for the $\oneLeptonTop$
    background and their mean values for each background.

    \insertFourFigures{MTtailResolution}
                      {MTtailResolution/MTvsMETReso}
                      {MTtailResolution/MTvsDeltaPhiReso}
                      {MTtailResolution/MTvsMeanMETReso}
                      {MTtailResolution/MTvsMeanDeltaPhiReso}
                      {0.45}
                      {$\MET$ and $\Delta \phi$ relative resolutions as function of $\MT$,
                      using the generated $MET$ from neutrinos coming from a $W$, on top
                      showing the full 2 dimensional distribution and on bottom showing
                      the evolution of the mean resolution for each background. Error
                      bars correspond to the uncertainties on the mean estimation.}

    The results of this small investigation tends to point out that the $\MT$ tail for
    the $\oneLeptonTop$, $\Wjets$ and rare backgrounds originates more from the
    misreconstruction of the magnitude of the $\MET$ rather than its direction. Small
    features are however observed in the $\Delta \phi$ spectra in the tail of $\MT$, which
    may originate from other sources of $\MET$ in the event such as neutrinos inside $b$-jets.

    %So far, the understanding of this disagreement comes from two source : first, the
    %off-shell contributions of $\Wjets$ is underestimated by the simulation, and second,
    %detector effects leads to a larger tail for the $\oneLeptonTop$ events.
    So far, the method used to correct the discrepancy is to compute ad-hoc scale factors
    using a template fit that estimates separately the contribution of $\oneLeptonTop$ and
    $\Wjets$ from the data. To do this, we use $\MlbPrime$ which was found to have a good
    discriminating power between the two process categories and being well discribed in
    the peak of $\MT$, as shown on \reffig{fig:MlbPrimeForTemplateFit}.

        \insertFourFigures{MlbPrimeForTemplateFit}
                          {templateFit/Mlb_0b_peak}
                          {templateFit/Mlb_0b_tail}
                          {templateFit/preselection_Mlb_noFit}
                          {templateFit/preselection_Mlb_withFit}
                          {0.45}
                          {Distribution of $\MlbPrime$ in the $0\, b\text{-tag}$ control region, data/MC comparison in the $M_T$ peak (top left), superimposed and normalized $\oneLeptonTop$ and $\Wjets$ components in the $M_T$ tail (top right), data/MC comparison in the $M_T$ tail before correction of the Monte-Carlo (bottom left) and after correction (bottom right). On the bottom right, the uncertainties on the scale factors are propagated in the ratio.}

    The method is implemented using the \textsc{RooFit} toolbox \refNeeded with the Minuit2
    implementation of the \textsc{Migrad} minimizer algorithm \refNeeded. The normalization
    of the $\oneLeptonTop$ and $\Wjets$ components are free parameters translated in term
    of scale factors, $SF_{\oneLeptonTop}$ and $SF_{\Wjets}$, while the normalization of
    the $\diLeptonTop$ and rare components are taken from the Monte-Carlo and constrained
    with a 20\% uncertainty during the fit process. To validate the method, a closure test
    is performed by generating toy data from the Monte-Carlo where arbitrary scale factors
    whre injected. The estimated scale factors are then compared to the input scale factors.
    A very good linearity is found for scale factors varying from 0.10 to 3. The fit is
    performed both in the $\MT$ peak and tail regions and we extract $SFR = SF^{\text{tail}}
    / SF^{\text{peak}}$ for both processes, representing the discrepancy in the tail
    independently from the peak normalization. Different sources of systematic uncertainties
    are considered : the jet energy scale, the normalization of the rare background, the
    Monte-Carlo statistics, the choice of the minimizer algorithm, the choice of the
    initial fit conditions, the generator setup for $t\bar{t}$ (using \textsc{Powheg} or
    \textsc{MadGraph}, varying the matching parameters, RGE scale, top mass, applying or
    not the top $\pT$ reweighting). The most important sources are the Monte-Carlo statistics
    (leading to 11\% of relative uncertainty on SFR), the generator scale (9\%) and the jet
    energy scale. A conservative 20\% is used as relative systematic uncertainty.

    Without applying any signal region cuts, $\SFRoneLeptonTop$ and $\SFRWjets$ are found
    to be $(1.04 \pm 0.16 (\text{stat.}) \pm 0.21 (\text{syst.}))$ and $(1.33 \pm 0.10
    (\text{stat.}) \pm 0.27 (\text{syst.}) )$ respectively. The plot on bottom right of
    \reffig{fig:MlbPrimeForTemplateFit}
    illustrates the impact on the $\MlbPrime$ data/MC comparison after propagating the scale
    factors. For the cut-based signal regions, we use the scale factors derived from a
    single $M_T$ cut associated to the signal region as shown on \reffig{fig:templateFit/CnC_MTcuts}.
    To cover possible correlation between the $SFR$ and $\MET$, the uncertainty from the
    single $\MET$ or $\MET/\sqrt{H_T}$ cut associated to the signal region is also
    quadratically added to the scale factor. These scale factors are shown on \reffig{fig:templateFitCnCResultsMET}.

    For the BDT signal regions, to be as close as possible to the kinematic of the BDT tails,
    we choose cuts on each BDT outputs such as at least 25\% of the background is still
    selected and apply the template fit method in that region. Two common scale factors
    to be applied to each BDT are then computed by averaging across all the trainings.
    The values found are $\SFRoneLeptonTop = (1.38 \pm 0.61)$ and $\SFRWjets = (1.21 \pm 0.36)$.

        \insertFigure{templateFit/CnC_MTcuts}
                     {0.6}
                     {Template fit results for individual cuts on $\MT$ after
                     preselection. The uncertainties shown are statistical only.}

        \insertTwoFigures{templateFitCnCResultsMET}
                         {templateFit/CnC_METcuts}
                         {templateFit/CnC_METoverSqrtHTcuts}
                         {0.45}
                         {Template fit results for invidiviual cuts on $\MET$ (left)
                         and $\MET/\sqrt{H_T}$ (right) after preselection + $\MT > 100\GeV$.
                         The uncertainties shown are statistical only.}

        \subsection{Control of $\diLeptonTop$ component and second lepton veto \label{sec:analysis_controlDileptonTop}}
        %==============================================================

        The 2-lepton and reversed veto control regions allow to check for the good modeling
        of the $\diLeptonTop$ background and the veto lepton definition. For the
        2 lepton control region, $\MT$ is defined using the leading lepton and ignoring
        the second one. The $\MT$-peak of the reversed veto control region is dominated by
        $\oneLeptonTop$ where a fake second lepton was reconstructed while the $\MT$-tail
        is dominated by $\diLeptonTop$ with a true second lepton. As for the $\MT$-peak
        normalization, we introduce scale factors to normalize the backgrounds in the peak
        of $\MT$ and quantify the agreement in the tail.

        In the reversed veto region, we define $\SFveto$ in a similar fashion as $\SFpost$,
        to control the fake-dominated region, and $\SFvetoTail$ to control the true second
        lepton dominated region after propagation of all the relevant scale factors including
        $\SFRoneLeptonTop$.

        Without applying any signal region cuts, we find $\SFveto = (1.18 \pm 0.03)$ and
        $\SFvetoTail = (1.07 \pm 0.02)$. The relatively large value of $\SFveto$ is
        interpreted as a mismodeling of the fake rate of the veto lepton by the Monte-Carlo
        and, despite being not compatible with unity, is not used to compute the prediction
        in the signal region because its effect is already included in $\SFpost$. The
        value of $\SFvetoTail$ also manifest a discrepancy in the veto lepton selection
        efficiency. Despite the fact that this scale factor is not propagated to the
        signal region, a systematic is later introduced to cover this effect.

        In the 2 leptons control region, we define $\SFtwoLep$ and $\SFtwoLepTail$ to
        control respectivelly the whole $\MT$ distribution and the tail of it.
        Without applying any signal region cuts, we find $\SFtwoLep = (0.96 \pm 0.01)$
        and $\SFtwoLepTail = (1.01 \pm 0.02)$ showing therefore a good modeling of $\MT$
        for this background.

        Another important check in the 2 leptons control region is the jet multiplicity
        modeling of $\diLeptonTop$ as the four jets requirements corresponds to two
        additional jets coming from radiations or pile-up for this background.
        \reffig{fig:} presents the distribution of the number of selected jets after
        application of $\MT>100$, which is found in good agreement.

        \reffig{fig:preselMT2leptonAndLepPlusVeto} presents the full $\MT$
        distributions for the reversed veto and two leptons control region.

        \begin{figure}[h!]
            \centering
            \includegraphics[width=0.45\textwidth]{controlPlots/reversedVeto_noMTCut/MT}
            \includegraphics[width=0.45\textwidth]{controlPlots/2leptons_noMTCut/MT}
            \caption{Full $\MT$ distributions for the reversed veto control region (on the left) and two leptons control region (on the right). On the left, $\SFpre$, $\SFveto$, $\SFRoneLeptonTop$ and $\SFRWjets$ are propagated. On the right, no scale factors is applied.}
                    \label{fig:preselMT2leptonAndLepPlusVeto}
        \end{figure}

        \insertFigure{controlPlots/2leptons/nJets}
                     {0.5}
                     {Distribution of the number of selected jets in the two leptons control region after applying $\MT > 100$.}

        \subsection{Control plots at preselection level}
        %==============================================================

        \reffig{fig:preselControlPlots} shows control plots for $\MT$, $\MET$ and $M_{T2}^W$ in the different control regions at the preselection level. \todo{Add comments here but I don't know what to say :/}

            \begin{figure}[h!]
                \centering
                \includegraphics[width=0.325\textwidth]{controlPlots/MTpeak/MT}
                \includegraphics[width=0.325\textwidth]{controlPlots/MTpeak/MET}
                \includegraphics[width=0.325\textwidth]{controlPlots/MTpeak/MT2W}\\
                \includegraphics[width=0.325\textwidth]{controlPlots/0btag/MT}
                \includegraphics[width=0.325\textwidth]{controlPlots/0btag/MET}
                \includegraphics[width=0.325\textwidth]{controlPlots/0btag/MT2W}\\
                \includegraphics[width=0.325\textwidth]{controlPlots/reversedVeto_noMTCut/MT}
                \includegraphics[width=0.325\textwidth]{controlPlots/reversedVeto_noMTCut/MET}
                \includegraphics[width=0.325\textwidth]{controlPlots/reversedVeto_noMTCut/MT2W}\\
                \includegraphics[width=0.325\textwidth]{controlPlots/2leptons_noMTCut/MT}
                \includegraphics[width=0.325\textwidth]{controlPlots/2leptons_noMTCut/MET}
                \includegraphics[width=0.325\textwidth]{controlPlots/2leptons_noMTCut/MT2W}\\
                \caption{A few control plots, showing the data/MC comparison for $\MT$ (left column),
                        $\MET$ (middle column) and $M_{T2}^W$ (right column) in the different control
                        regions : $\MT$-peak (first row), $0\, b\text{-tag}$ (second row), reversed veto (third
                        row) and 2 leptons (fourth row). The $\MT$-peak normalization and $\MT$-tail
                        correction scale factors are propagated where relevant.}
                        \label{fig:preselControlPlots}
            \end{figure}

        \subsection{Background prediction}
        %==============================================================

        The background prediction is obtained by taking the Monte-Carlo yield in the
        $\MT$-tail and propagating $\SFpre$ to the $\diLeptonTop$ component and $\SFpost$
        to the $\oneLeptonTop$ and $\Wjets$ component. The $\oneLeptonTop$ and $\Wjets$
        are also corrected with $\SFRoneLeptonTop$ and $\SFRWjets$ respectively. The
        prediction for the rare component is directly the Monte-Carlo yield in $\MT$-tail.
        The \refequation{eq:prediction1ltop} to \refequation{eq:predictionrare} below summarize
        the computation of the prediction. The procedure is repeated for each signal
        region as all the scale factors involved are signal-region dependent. As an
        illustration, \reftab{tab:predictionPreselection}
        shows the comparison between the raw Monte-Carlo and the prediction obtained at
        preselection level, while \reftab{tab:report_yield_CnC_T2tt} presents in particular
        the predictions for the cut \& count approach for $\lstop \rightarrow t \lneutralino$.


        \begin{eqnarray}
            N^\text{pred}_\text{tail}(\oneLeptonTop) & = & N^\text{MC}_\text{tail}(\oneLeptonTop)  \times \SFpost \times \SFRoneLeptonTop \label{eq:prediction1ltop}  \\
            N^\text{pred}_\text{tail}(\Wjets)        & = & N^\text{MC}_\text{tail}(\Wjets)         \times \SFpost  \times \SFRWjets                             \\
            N^\text{pred}_\text{tail}(\diLeptonTop)  & = & N^\text{MC}_\text{tail}(\diLeptonTop)   \times \SFpre                                                \\
            N^\text{pred}_\text{tail}(\text{rare})   & = & N^\text{MC}_\text{tail}(\text{rare})                                           \label{eq:predictionrare}
        \end{eqnarray}

        \begin{table}[!ht]
            \begin{center}
                \begin{tabular}{|l|c|c|}
                    \hline
                                             &  \textbf{Raw MC}    & \textbf{Prediction}       \\
                    \hline
                    \textbf{$\oneLeptonTop$} &  5970 $\pm$ 31      & 6526 $\pm$ 1632     \\
                    \textbf{$\diLeptonTop$}  &  2117 $\pm$ 18      & 2253 $\pm$  229     \\
                    \textbf{$\Wjets$}        &   477 $\pm$ 13      &  669 $\pm$  364     \\
                    \textbf{rare}            &   490 $\pm$ 13      &  490 $\pm$  245     \\
                    \hline
                    \textbf{Total SM}        &  9055 $\pm$ 41      & 9940 $\pm$ 1666     \\
                    \hline
                \end{tabular}
                \caption{Background prediction at the preselection + $\MT > 100\GeV$ level.
                The raw MC uncertainties are only coming from the Monte-Carlo sample statistics
                while the uncertainties on the prediction include all the effects discussed
                in \refsection{sec:background_systematics}.}
                \label{tab:predictionPreselection}
            \end{center}
        \end{table}

        \begin{table}[!ht]
            \begin{center}
                { \footnotesize
                \begin{tabular}{|l|ccccc|}
                    \hline
                    &
                    \textbf{Off-shell loose}    &
                    \textbf{Off-shell tight}    &
                    \textbf{Low $\Delta m$}     &
                    \textbf{Medium $\Delta m$}  &
                    \textbf{High $\Delta m$}    \\
                    \hline
                    \textbf{1$\ell$ top}     & 4.32 $\pm$ 1.69   & 0.15 $\pm$ 0.20   & 28.18 $\pm$ 13.71     & 2.26 $\pm$ 1.28   & 0.00 $\pm$ 0.00   \\
                    \textbf{$t\bar{t} \rightarrow \ell \ell$}    & 29.07 $\pm$ 7.58      & 8.65 $\pm$ 4.16   & 130.01 $\pm$ 11.29    & 4.86 $\pm$ 1.86   & 2.20 $\pm$ 1.16   \\
                    \textbf{$W$+jets}    & 0.87 $\pm$ 0.82   & 0.75 $\pm$ 0.79   & 6.73 $\pm$ 4.19   & 0.85 $\pm$ 0.83   & 0.00 $\pm$ 0.00   \\
                    \textbf{rare}    & 4.26 $\pm$ 2.35   & 1.86 $\pm$ 1.15   & 13.97 $\pm$ 7.21      & 2.69 $\pm$ 1.43   & 1.22 $\pm$ 0.78   \\
                    \textbf{total SM}    & 38.53 $\pm$ 8.38      & 11.40 $\pm$ 4.40      & 178.90 $\pm$ 21.82    & 10.66 $\pm$ 2.60      & 3.42 $\pm$ 1.40   \\
                    \hline
                \end{tabular}
                }
                \caption{Background predictions for the signal regions of the cut \& count approach targetting the $\lstop \rightarrow t \lneutralino$ decay mode.}
                \label{tab:report_yield_CnC_T2tt}
            \end{center}
        \end{table}


    \section{Systematic uncertainties \label{sec:analysis_systematics}}
    %==============================================================

        This section describes the sources of systematic uncertainties that are considered for the background and the signal.

        \subsection{Systematic uncertainties on the background \label{sec:background_systematics}}
        %==============================================================

            Several sources of systematic uncertainties are considered for the background,
            the most important ones being from the $\MT$-peak normalization, the $\MT$-tail
            correction and the $\diLeptonTop$ modeling of the $\MT$ tail.

            \subsubsection{Modeling of $\diLeptonTop$ in $\MT$ tail}

            As discussed in \refsection{sec:analysis_controlDileptonTop}, the modeling by
            the Monte-Carlo of the $\MT$ tail of the $\diLeptonTop$ background is found
            to be good in the 2 leptons and reversed veto control regions. As it is a
            major background of the analysis, a systematic uncertainty is nevertheless
            asserted to quantify the trust in the Monte-Carlo on a per-signal-region basis.

            To do this, one wants to probe the 2 leptons and reversed veto control regions
            as close as possible of the signal region. However, as the signal region cuts
            are sometimes quite tight, the remaining statistics in these control regions is
            too low and doesn't allow a reasonable check of the distributions. To work
            around this problem while still probing the tail of $\MT$ near the signal
            region, we define loosened cuts to check these scale factors with more
            statistics. These cuts are designed by requiring to have at least 30 events
            remaining in the tail of $\MT$ for the 2 lepton control region.

            For each of the relaxed control regions, we compute the value of $\SFtwoLepTail$
            and $\SFvetoTail$ to quantify the agreement between data and simulations in
            the tail of $\MT$. An envelope is then computed for each signal
            region to account for the spread of the scale factors for each of the
            associated control regions. For the cut-based signal regions, this leads to a
            relative uncertainty on the total background yield varying from 1.5 to 35\%. For the
            BDT signal regions, this relative uncertainty is between 7 and 40\%.

            \subsubsection{Second lepton veto efficiency}
            %============================================

            The uncertainty on the efficiency of the second lepton veto is propagated to the
            fraction of $\diLeptonTop$ events that have a second lepton in the acceptance. For
            the isolated track veto lepton, this is defined as having a second generated
            $e/\mu$ or a one prong $\tau \rightarrow h$ with $\pT > 5 \text{ or } 10 \GeV$,
            respectively, with $\abseta < 2.4$. This fraction is between 50-70\% for all
            signal regions. The uncertainty for these events is 6\% and is obtained from
            tag-and-probe studies \refNeeded. Regarding the $\tau$ veto leptons, the
            events considered are those with a hadronic $\tau$ in the acceptance, with
            true visible transverse energy $> 20\GeV$ in $\abseta < 2.4$. This fraction is
            about $10 \sim 20 \%$ of the total. The uncertainty on the efficiency of the
            $\tau$-ID algorithm is 7\%, taken from $\tau$ group studies \refNeeded.

            \subsubsection{Uncertainty on $\SFRoneLeptonTop$ and $\SFRWjets$}
            %================================================================

            As described in \refsection{sec:MTtailCorrection}, the $\MT$-tail correction
            scale factors for $\oneLeptonTop$ and $\Wjets$ are computed with an uncertainty
            coming from statistics in the $0\, b\text{-tag}$ control region and systematic effects
            from the template fit method itself, as well as . This uncertainty is propagated
            to the total background yield uncertainty and is one of the major contribution
            for the signal regions with a large remaining fraction of $\oneLeptonTop$.
            For the cut-based signal regions, this corresponds to a relative uncertainty
            on the total background ranging from 0 to 15\%, and up to 17\% for the BDT
            signal regions.

            \subsubsection{Statistic uncertainty in $\MT$ peak}
            %==============================================================

            The $\MT$-peak normalization scale factors are an important part of the background
            estimation procedure, but is nevertheless limited by the statistics available
            in the peak region. Therefore, the $\SFpre$ and $\SFpost$ scale factors come
            associated to an uncertainty, dominated by the event count of data. This
            uncertainty is propagated to the prediction in the tail. This lead to a
            relative uncertainty on the total background ranging from 2 to 15\% for the cut-based
            signal regions and between 3 and 40\% for the BDT signal regions.

            \subsubsection{Other sources of systematic uncertainties}
            %========================================================

            Other sources of uncertainties are taken into account though being small compared
            to the ones described in the previous subsections :
            \begin{itemize}
                \item To cover the modeling of ISR and FSR jets, the $N_\text{jets}$
                      distribution is studied in the 2 leptons control region. An
                      uncertainty of 2\% is asserted on the $\diLeptonTop$ background
                      from this check.
                \item To account for possible mismodeling of the relative proportions of
                      the backgrounds, the $\oneLeptonTop$ component cross-section
                      is varied by 10\% while the $\Wjets$ cross-section is varied by 50\%
                      during the background estimation procedure.
                \item As it is difficult to design a control region for the rare category,
                      in part due to the variety of processes it contains, its contribution
                      is taken directly from MC. We however put a conservative 50\% uncertainty
                      in the rate of this category.
                \item The Monte-Carlo statistics available in the $\MT$ tail being limited,
                      it also contributes to the systematic uncertainty on the final prediction.
            \end{itemize}

            \subsubsection{Summary of background uncertainties at preselection}
            %==============================================================

            \reftab{tab:systematicsSummary} shows a breakdown of the different systematic
            uncertainties that are considered, at preselection level and the range of them
            for the two kinds of signal regions. The relative importance of the individual
            systematics varies depending on the signal regions as the composition of the
            backgrounds itself varies : at preselection level, the importance of the
            $\SFRoneLeptonTop$ uncertainty is high as the $\oneLeptonTop$ component is still large.
            However for some signal regions, the $\diLeptonTop$ is the dominant contribution and
            uncertainty from the $\MT$-tail modeling becomes the leading systematic source.

            % Add range of uncertainties in the signal regions

            \begin{table}[!ht]
                \begin{center}
                    \begin{tabular}{|l|c|cc|}
                        \hline
                                                                       & Preselection    & Cut-based      & BDT             \\
                                                                       & + $\MT>100\GeV$ & signal regions & signal regions  \\
                        \hline
                        \textbf{$\diLeptonTop$ ($\MT$-tail modeling)}  & 1.6                      & 2-35         & 7-40    \\
                        \textbf{$\diLeptonTop$ (jets modeling)}        & 1.1                      & 1-4          & 0.5-4   \\
                        \textbf{$\diLeptonTop$ (2nd lepton veto)}      & 1.2                      & 0-4          & 1-4     \\
                        \textbf{$\SFRWjets$ uncertainty}               & 1.4                      & 0-6          & 0-5     \\
                        \textbf{$\SFRoneLeptonTop$ uncertainty}        & 16.4                     & 0-15         & 0-17    \\
                        \textbf{$\MT$-peak SF uncertainties}           & 0.7                      & 2-15         & 3-40    \\
                        \textbf{Cross-sections and MC stat}            & 1.9                      & 7-48         & 7-47    \\
                        \hline
                        \textbf{total}                                 & 16.8                     & 12-50        & 25-60   \\
                        \hline
                    \end{tabular}
                    \caption{Summary of the relative uncertainties (in \%) at preselection+$\MT>100\GeV$
                    and range of relative uncertainties with respect to the total predicted
                    background yield for cut-based signal regions and BDT signal regions.
                    \label{tab:systematicsSummary}}
                \end{center}
            \end{table}

        \subsection{Systematic uncertainties on the signal}
            %==============================================================

        While the background prediction is dominated by data-driven systematic
        uncertainties, the signal uncertainty sources are more related to the
        confidence in the different element of the construction of the Monte-Carlo
        samples and algorithm used.

        The limited available statistics of the signal sample leads to a maximal
        2\% uncertainty. The integrated luminosity is known with a precision of 2.2\%
        and is propagated to the uncertainty of the yield. The trigger efficiency
        used in the very first steps of the selection is known with a precision of 3\%.
        The lepton identification and isolation efficiency are observed to be consistent
        between data and Monte-Carlo within an envelope of 5\%.

        The jet energy scale uncertainty is studied by varying the jet energy corrections
        within their $\pm1\sigma$ uncertainty before the jet selection. The variation is
        properly propagated into the $\MET$ value. During the process, we also assume a
        10\% uncertainty on the unclustered energy defined as $(\vec{\MET} + \sum_\text{jets}
        \vec{p} + \sum_\text{leptons} \vec{p})$ where jets and leptons are selected with looser
        $\pT$ and $\abseta$ requirements. This effect leads to a maximum 10\% uncertainty on
        the signal yields.

        The uncertainty on the reshaping of the $b$-tagging discriminator is also considered
        by varying the technique within $\pm1\sigma$ uncertainy before the application of
        $b$-tagging requirements. This leads to a 3\% uncertainty on the signal yields.

        The uncertainty on the ISR jets reweighting applied on signal is taken from data/MC
        scale factors derived from the analysis of events with high $\diLeptonTop$ purity. The
        scale factors, function of the $\pT$ recoil of the system, are varied within their
        uncertainties and lead to a maximum variation of 8 and 10\% on the signal yield, depending
        of the decay mode.

        Finally, the uncertainty on PDF are calculated, following the PDF4LHC prescription,
        using the CT10, NNPDF 2.1, and MSTW2009 PDF sets. \refNeeded The impact on the
        signal efficiency is about 5\%.

    \section{Signal contamination handling}
    %==============================================================

        Signal contamination occurs when a significant fraction of signal events is present in
        the control regions. While it doesn't affect the predicted yield for the background-only hypothesis
        ($H_0$), a significant contamination can bias the data-driven aspects of the background
        estimation when predicting the expected yield under the signal hypothesis ($H_1$). As a
        consequence, it leads to an overestimation of the expected background under the signal hypothesis,
        therefore increasing the probability to incorrectly reject the signal hypothesis (type II error).

        The signal contamination level is studied across the $(\mass{\lstop},\mass{\lneutralino})$
        plane by computing the $C \definedAs S/B$ in the $\MT$-peak control region and 0 $b$-tag
        control region and comparing it to the signal purity, $P \definedAs S/B$, in the signal region :

        \begin{equation}
            R \definedAs \frac{C}{P} = \frac{(S/B)_\text{control region}}{(S/B)_\text{signal region}}
            \label{eq:contaminationRatio}
        \end{equation}

        This ratio $R$ is found to be sometimes higher than an arbitrary threshold value of 15$\sim$20\%.
        This is especially true when considering the low $\deltam$ region of the $(\mass{\lstop},
        \mass{\lneutralino})$ plane, as the signal is likely to get smaller values of $\MT$ and the $b$-jets
        are less likely to be selected or correctly $b$-tagged as their momentum decrease. This is illustrated
        on \reffig{fig:signalContaminationIllustration} which shows the differences in shape of $\MT$ and
        $b$-tagged jet multiplicity for two benchmarks of the $\lstop \rightarrow t \lneutralino$ decay mode.

        \insertTwoFigures{signalContaminationIllustration}
                         {signalContamination/MT}{signalContamination/nBtag}{0.4}
                         {Illustration of the signal contamination evolution using two signal examples
                         T2tt (250/100) and T2tt(650/50). The low $\deltam$ benchmark, T2tt (250/100),
                         has 30\% of events with 0$b$-tag jets and a large fraction of events at low $\MT$.}

        It is concluded that the signal contamination can not be neglected. One needs to correct the modeling
        of the $H_1$ hypothesis by peforming a different background estimation $\tilde{B}$ compared to the
        $H_0$ hypothesis.

        To do this, the data-driven aspects are corrected by including the signal when computing the scale factors for the
        $\MT$-peak normalization and $\MT$-tail correction. In the case of $\SFpre$ and $\SFpost$, the scale factors
        are corrected by substracting alo the signal component to the data when normalizing the $\oneLeptonTop$, $\Wjets$
        and $\diLeptonTop$ components :

        \begin{equation}
            \SFpreTilde \definedAs \left( \frac{N(\text{data}) - N(\text{rare}) - N(\text{signal}))}{N(\oneLeptonTop) + N(\Wjets) + N(\diLeptonTop)} \right)
        \end{equation}
        \begin{equation}
            \SFpostTilde \definedAs \left( \frac{N(\text{data}) - N(\text{rare}) - N(\text{signal}) - \SFpreTilde \times N(\diLeptonTop)}{N(\oneLeptonTop) + N(\Wjets)} \right)
        \end{equation}

        In the case of the $\MT$-tail correction scale factors, they are corrected by including the signal contribution
        to the rare category before fitting the $\oneLeptonTop$ and $\Wjets$ components to the data using the template
        fit method. We however constrain, a posteriori to the fit, $\SFRoneLeptonTopTilde$ to be $\geq 1$.

        The corrected background $\tilde{B}$ is computed the same way as described in \refequation{eq:prediction1ltop} to \refequation{eq:predictionrare}
        using the corrected scale factors. As the correction depends on the signal, it has to be performed on a per-benchmark
        basis. However, as it is a CPU intensive task, it is done only with a step of $50 \GeV$ instead of the $25 \GeV$
        of the signal samples. The background prediction for other benchmarks is corrected using an interpolation of
        the ration $\tilde{B}/B$ across the $(\mass{\lstop},\mass{\lneutralino})$ plane. \reffig{fig:signalContaminationSFmap} shows
        the obtained ratio $\tilde{B}/B$ for each signal type, showing an effect up to 25\% at low masses.

        \insertFourFigures{signalContaminationSFmap}
                          {signalContamination/globalSFmap_T2tt}
                          {signalContamination/globalSFmap_T2bw-075}
                          {signalContamination/globalSFmap_T2bw-050}
                          {signalContamination/globalSFmap_T2bw-025}
                          {0.4}
                          {Map of the ratio $\tilde{B}/B$, i.e. signal-contamination corrected background prediction versus uncorrected prediction, using the BDT signal regions and for the $\lstop \rightarrow t \lneutralino$ decay mode (top left) and $\lstop \rightarrow b \lchargino$ decay mode with $x=0.75$ (top right), $x=0.50$ (bottom left) and $x=0.25$ (bottom right).}

    \section{Results and interpretation \label{sec:analysis_results}}
    %==============================================================

    On the left of figures \reffig{fig:resultsCnC} and \reffig{fig:resultsBDT}, comparisons
    of the yield between data and background prediction under the null hypothesis
    for each signal regions of the cut-based approach and BDT approach are presented.
    A good compatibility is observed with the background-only expectation. The results
    are therefore interpreted in terms of upper limit on $\sigma(\lstop\lstop) \times BR$
    with a 95\% confidence level (CL).  Comparing the upper limit with the theoretical
    expectation for a branching ratio of 1, one can derive limits in terms of excluded
    region of the $(\mass{\lstop}, \mass{\lneutralino})$ space, as reported on the right
    of \reffig{fig:resultsCnC} and \reffig{fig:resultsBDT}. During the interpretation,
    the signal hypothesis modeling is corrected to account for signal contamination.

    \begin{figure}[h!]
        \centering
        \includegraphics[width=0.33\textwidth]{results/CnC_T2tt/signalRegion_MTtail_yield}
        \includegraphics[width=0.31\textwidth]{limits/T2tt_CC}\\
        \includegraphics[width=0.33\textwidth]{results/CnC_T2bw075/signalRegion_MTtail_yield}
        \includegraphics[width=0.31\textwidth]{limits/T2bw075_CC}\\
        \includegraphics[width=0.33\textwidth]{results/CnC_T2bw050/signalRegion_MTtail_yield}
        \includegraphics[width=0.31\textwidth]{limits/T2bw050_CC}\\
        \includegraphics[width=0.33\textwidth]{results/CnC_T2bw025/signalRegion_MTtail_yield}
        \includegraphics[width=0.31\textwidth]{limits/T2bw025_CC}\\
        \caption{On the left : comparison of the yield of the different cut-based signal
        regions between data and the background prediction under the null hypothesis. The
        grey hatching represents the systematic uncertainty, propagated on the ratio plot.
        On the right : upper limit at 95\% confidence level and exclusion in terms of
        $(\mass{\lstop},\mass{\lneutralino})$ after comparison to the theory, assuming
        $BR = 1$. On the first row, for $\lstop \rightarrow t \lneutralino$ decay mode on
        the second, third and last row for $\lstop \rightarrow b \lchargino$ decay mode
        with $x=0.75$, $0.50$ and $0.25$ respectively.}
        \label{fig:resultsCnC}
    \end{figure}

    \begin{figure}[h!]
        \centering
        \includegraphics[width=0.33\textwidth]{results/BDT_T2tt/signalRegion_MTtail_yield}
        \includegraphics[width=0.31\textwidth]{limits/T2tt_BDT}\\
        \includegraphics[width=0.33\textwidth]{results/BDT_T2bw075/signalRegion_MTtail_yield}
        \includegraphics[width=0.31\textwidth]{limits/T2bw075_BDT}\\
        \includegraphics[width=0.33\textwidth]{results/BDT_T2bw050/signalRegion_MTtail_yield}
        \includegraphics[width=0.31\textwidth]{limits/T2bw050_BDT}\\
        \includegraphics[width=0.33\textwidth]{results/BDT_T2bw025/signalRegion_MTtail_yield}
        \includegraphics[width=0.31\textwidth]{limits/T2bw025_BDT}\\
        \caption{On the left : comparison of the yield of the different BDT-based signal
        regions between data and the background prediction under the null hypothesis. The
        grey hatching represents the systematic uncertainty, propagated on the ratio plot.
        On the right : upper limit at 95\% confidence level and exclusion in terms of
        $(\mass{\lstop},\mass{\lneutralino})$ after comparison to the theory, assuming
        $BR = 1$. On the first row, for $\lstop \rightarrow t \lneutralino$ decay mode on
        the second, third and last row for $\lstop \rightarrow b \lchargino$ decay mode
        with $x=0.75$, $0.50$ and $0.25$ respectively.}
        \label{fig:resultsBDT}
    \end{figure}

    The BDT approach leads to limits which are usually about $50 \GeV$ futher compared to
    the cut-based approach. For the $\lstop \rightarrow t \lneutralino$ decay mode, the
    observed exclusion using the BDT approach goes up to $\mass{\lstop} \sim 700 \GeV$
    and $\mass{\lneutralino} \sim 250 \GeV$ in the on-shell region and up to
    $\mass{\lneutralino} \sim 125 \GeV$ in the off-shell region. It is however difficult
    to exclude the region $\deltam \sim \mass{t}$ as the kinematic here is very close to
    standard model $t\bar{t}$.

    \todo{To be added when available (for the CnC especially) : polarized limits, branching
    ratio variation, maybe replace BDT results by a comparison BDT <-> CnC. \\Combination
    with 2$\ell$}

    Alternative exclusion in terms of $(\mass{\lstop}, \mass{\lneutralino})$ are proposed
    for different branching ratio assumptions, as represented on \reffig{fig:}.
    Different polarization assumptions are also being looked at on \reffig{fig:} for both
    decay types and can have a significant impact

    \reffig{fig:event62838873} shows one of the most signal-like event in the
    high-$\deltam$ signal region of the cut-based approach for $\lstop \rightarrow t \lneutralino$.

    \insertFigure{event62838873}{0.9}
    {One of the most signal-like event for the $\lstop \lstop^{*} \rightarrow t t \lneutralino \lneutralino$ decay-mode in the high-$\deltam$ cut-based approach. The event has one muon with $\pT = 114\GeV$, 4 jets among which 2 $b$-tag, $\MET = 392\GeV$ and $\MT = 300 \GeV$. Only tracks coming from the primary vertex are shown.}

    \newpage

    \section{Perspectives \label{sec:analysis_perspective}}
    %==============================================================

    \subsection{$W$-tagging in the high $\Delta m$ regime}

            \subsubsection{Motivation}

             As one considers higher $\Delta m$ values for the signal, the mean momentum of
             decay products increases. In particular, if we consider the hadronically
             decaying $W$, an increase of the $\pT$ translates into more collimated objects,
             in that case the pair of quarks that will hadronize. This is illustrated on
             the \reffig{fig:wTagging/ptW_vs_genDeltaRqq_fromttbar} showing the
             distribution of the $\Delta R$ between the
             quarks coming from the decay of a $W$ boson against the $\pT$ of the generated
             $W$. In the situation were the $\Delta R$ between the quarks approaches the
             size parameter used by the standard clustering algorithm (i.e. $\Delta R
             \sim 0.5$), only one big jet gets reconstructed instead of two smaller ones.
             This topology is referred to as boosted hadronic $W$.

             \insertFigure{wTagging/ptW_vs_genDeltaRqq_fromttbar}{0.5}{Distribution of
             the $\Delta R$ between the quarks coming from the decay of an hadronic $W$,
             as function of the generated $\pT$ of the $W$. The mean $\Delta R$ approaches
             0.5, the standard size parameter used at 8 TeV, at $\pT \sim 200\GeV$,
             meaning that jets coming from the quark will be merged by the clustering
             algorithm.}

             Driven by the fact that some new physics signatures are expected to contain
             such boosted hadronic $W$ \refNeeded, techniques have been developped to
             address this topology by providing variables to tag jets originating from
             boosted $W$ decays. The strategy consists in using a wider radius parameter
             when clustering the jets, clean and correct the jets from pile-up contamination,
             and analyze the substructure of the jets to derive variables that discriminate
             between boosted $W$ decays and fakes.

             \reffig{fig:genWPtForSignal} illustrates the interest that these techniques might have to
             select the signal : on the left plot, the mean $\pT$ of the generated
             $W$ bosons for the signal accross the $(\mass{\lstop}, \mass{\lneutralino})$
             space grows as function of $\deltam$. For $\deltam > 650\GeV$, the
             mean $\pT$ is about $200\GeV$ and we can expect a large fraction of boosted $W$.
             \reffig{fig:genWPtForSignal}, on the right, compares the distribution of the $\pT$ of the
             hadronic $W$ for one particular signal benchmark at high-$\deltam$ against
             the different backgrounds and shows that the presence a boosted $W$ tends to
             be discriminating.

             \insertTwoFigures{genWPtForSignal}
                              {wTagging/T2tt_meanGenWPt}{wTagging/genWPt_backgroundVsSignal}{0.4}
                              {On the left : mean $\pT$ of the generated $W$ for the
                              signal across the $(\mass{\lstop}, \mass{\lneutralino})$ space.
                              On the right : comparison of the $\pT$ spectra of the generated
                              hadronic $W$ for the $\oneLeptonTop$ and rare backgrounds, and
                              the signal benchmark $(\mass{\lstop}, \mass{\lneutralino}) = (700,25) \GeV$.
                              The $\diLeptonTop$ and $\Wjets$ backgrounds are not represented
                              as they do not contain a generated hadronic $W$ by definition.}

            \subsubsection{Selection and performances}

            As an alternative to the standard anti-$k_T$ clustering algorithm with a
            size parameter $R = 0.5$ (AK5), an other jet collection is built using the
            Cambridge-Aachen clustering algorithm with a size parameter $R = 0.8$ (CA8).
            \todo{Add a reference to the detector chapter where AK, CA and kt are
            presented/compared ?}\todo{Comparison between CA and AK performances for jet
            substructure done in http://arxiv.org/pdf/0903.5081v4.pdf justifies the
            choice of CA8}

            To clean the jet from pile-up contributions and improve rejection of
            quark/gluon jets, a grooming technique called pruning is applied. The grooming
            consists in reclustering the jet into smaller subjets. During the process,
            conditions are applied to forbid the combinations of softer components (e.g.
            with energy smaller than 10\% of the protojet) or large angle combinations,
            as illustrated on \reffig{fig:wTagging/jetGrooming}. \todo{Ref needed : JME-13-006, arXiv:0903.5081v4}.

            \insertFigure{wTagging/jetGrooming}{0.7}{Illustration of the three available
            jet grooming techniques. The flitering technique consist in re-clustering
            components of the jets with a smaller jet size parameter (e.g. 0.3) and
            keeping only a given number (e.g three) of the subjets. The trimming techniques
            also reclusters the components with a smaller jet size parameter, but keeps
            all subjects with a significant $\pT$ fraction of the fat jet $\pT$.
            Finally, the pruning techniques veto soft or large angle combinations between
            the jet components, likely to come from pile-up.}

            The substructure of the jet is analyzed via the $N$-subjetiness variables,
            which are designed to quantify how likely a jet is to be composed of $N$
            sub-jets. \todo{Ref to arXiv:1011.2268v3} These variables are denoted $\tau_1$,
            $\tau_2$, ... $\tau_N$. A value close to 0 for $\tau_N$ tends to indicate
            a good compatibility with the $N$-subjets hypothesis. In the context of
            $W$-tagging, it is common to focus on the use of the ratio $\tau_2/\tau_1$
            which provides good discriminability between real $W$ and quark/gluons jets.

            To define selection criteria, we study the distribution of a few variables
            on a $t\bar{t}$ Monte-Carlo sample after applying the preselection defined in
            \refsection{sec:analysis_objectAndEventSelection}. We however allow events with at
            least three regular (i.e. based on anti$k_T$ with a size parameter 0.5) jets
            instead of four. $W$ candidates are matched to generated
            hadronically-decaying $W$ : if the candidate is within $\Delta R < 0.4$, it
            is considered as matched, whereas candidates which are in $\Delta R > 2$ are
            considered to be fakes originating from quark or gluons. We study the
            prunned mass of the jet, the $N$-subjetiness ratio $\tau_2 / \tau_1$ and the
            distance to the selected lepton $\Delta R (\ell,\text{jet})$.

            \reffig{fig:wTaggingVariables} shows the distribution of the prunned mass of
            the jet and the $N$-subjetiness ratio $\tau_2 / \tau_1$ for candidates with
            $\pT > 150\GeV$ and with $\Delta R(\ell,\text{jet}) > 1.5$. A good working point
            is found to be $\text{mass}(\text{jet}) > 70\GeV)$ and $\tau_2 / \tau_1 < 0.5$.
            The resulting tagging efficiency is estimated as function of the $\pT$ of the
            candidate as presented on \reffig{fig:wTagging/taggingEfficiency_fromttbar}.
            The efficiency for candidates matched
            to true $W$ is about 30\% at $200\GeV$ and reaches a plateau to 70\% at $270\GeV$. It
            however starts decreasing around $350\GeV$ as it gets more difficult to resolve
            the two subjets. The fake rate is about 5\% for candidates of $200\GeV$ and
            grows linearly with the $\pT$ as momentum tends to create unphysical large
            mass for the jets.

            \insertTwoFigures{wTaggingVariables}
                             {wTagging/prunnedMassAfterBasicSelection_fromttbar}
                             {wTagging/tau2OverTau1AfterBasicSelection_fromttbar}
                             {0.4}
                             {Distribution of the prunned mass (on the left) and $\tau_2
                             / \tau_1$ for CA8 jets with $\pT > 150\GeV$ and
                             $\Delta R(\ell,\text{jet}) > 1.5$ in $t\bar{t}$ events
                             with preselection applied.}

            \insertFigure{wTagging/taggingEfficiency_fromttbar}{0.5}
                         {Tagging efficiency for the true $W$ jets (in red) and fakes from quark/gluon
                         jets (in blue), as function of the $\pT$ of the jet.}

            \subsubsection{Impact on the analysis sensitivity}

            In this section, we investigate the potential benefit of the use of $W$-tagging
            in the context of the analysis and in term of sensitivity. \reffig{fig:wTagging/analysisSelectionEfficiency}
            shows the fraction of background and signals containing or not a $W$-tagged jet
            with $\pT > 250\GeV$ at preselection level with $\MT > 100\GeV$. The
            rare category has around 5\% of events containing such a $W$-tagged jet compared
            to less than 3\% for the other categories. The fraction of signal with
            a $W$-tagged jet increases from 20 to 30\% between the two benchmarks $(600,0)$
            and $(800,0)$ for the $\lstop \rightarrow t \lneutralino$ decay mode.

            \insertFigure{wTagging/analysisSelectionEfficiency}{0.5}
                         {Fraction of background and signal events containing a
                         $W$-tagged jet with $\pT > 250\GeV$, as preselection level with
                         $\MT > 100\GeV$. The two signal benchmarks considered are the $\lstop
                         \rightarrow t \lneutralino$ decay mode with
                         $(\mass{\lstop},\mass{\neutralino}) = (600,0)\GeV$ and $(800,0)\GeV$.}

            At this stage, one can already estimate that the significance gain, $\epsilon_S / \sqrt{\epsilon_B}$,
            of requiring at least one $W$-tag in the event, is around $1.34$ for the
            signal benchmark $(800,0)$. However, while this alone shows that $W$-tagging
            is an interesting technique, the question we want to answer is whether or not
            it is possible to increase the performances of the analysis compared to a set
            of optimized cuts on $\MT$, $\MET$ and $M_{T2}^{W}$ as in
            \refsection{sec:cutAndCountPerformances} for the high $\deltam$ region.

            To answer this question, we consider the analysis (0) that make no use of $W$-tagging
            and rely only on optimized cuts on $\MT$, $\MET$ and $M_{T2}^{W}$. In parallel,
            we consider two populations of events : the first one, (A), passing the preselection,
            but also allowing event with three regular jets to pass, and requiring at least
            one CA8 $W$-tagged jet ; the second one (B) passing the preselection (with at least
            four jets) and requiring that no CA8 jet is $W$-tagged.

            On each of these selection, we train cuts on $\MT$, $\MET$ and $M_{T2}^{W}$
            to maximize an exclusion-oriented figure of merit in the high $\deltam$ ($> 600\GeV$) region,
            as described in \refsection{sec:FoMdiscussion}. The relative systematic uncertainty
            on the background is set to 30\% and to avoid extreme cut values,
            the figure of merit is set to 0 if the signal yield is lower than 0.5, and the
            background yield $B$ is replaced by $\text{max}($B$,1)$. The cuts are expected
            to be looser in the preselection with at least one $W$-tagged jet as this
            requirement is already tight and only few event remains. The resulting cuts are
            presented on \reftab{tab:wTaggingAnalysisCuts}, as well as a breakdown of the
            yield of the background and several signal benchmarks with increasing $\deltam$.

        \begin{table}
            \centering
            \begin{tabular}{c|c|cc}
                                       & (0)               & (A)                          & (B)                    \\
                                       & Preselection      & Preselection                 & Preselection           \\
                                       & No $W$-tagging    & with $\geq$3 AK5 jets        & ($\geq$4 AK5 jets)     \\
                                       & usage             & + $\geq$1 $W$-tagged CA8 jet & + 0 $W$-tagged CA8 jet \\
                \hline
                \textbf{$\MET$}        & >350              & >325 & >350 \\
                \textbf{$\MT$}         & >150              & >130 & >150 \\
                \textbf{$M_{T2}^{W}$}  & >220              & >190 & >220 \\
                \hline
                \hline
                \textbf{Total SM}      & 1.91 $\pm$ 0.57   & 0.97 $\pm$ 0.26     &  1.66 $\pm$ 0.56    \\
                \hline
                \textbf{T2tt (600/0)}   & 6.61 $\pm$ 0.28  & 3.02 $\pm$ 0.18     &  5.02 $\pm$ 0.25    \\
                \textbf{T2tt (650/0)}   & 4.50 $\pm$ 0.17  & 2.30 $\pm$ 0.12     &  3.30 $\pm$ 0.15    \\
                \textbf{T2tt (700/0)}   & 2.87 $\pm$ 0.10  & 1.49 $\pm$ 0.07     &  2.08 $\pm$ 0.08    \\
                \textbf{T2tt (750/0)}   & 2.01 $\pm$ 0.07  & 0.99 $\pm$ 0.05     &  1.43 $\pm$ 0.06    \\
                \textbf{T2tt (800/0)}   & 1.33 $\pm$ 0.04  & 0.73 $\pm$ 0.03     &  0.89 $\pm$ 0.04
            \end{tabular}
            \caption{Optimized cuts on $\MET$, $\MT$ and $\MT2W$ in the high $\deltam$
                     ($> 600\GeV$) region.}
                     \label{tab:wTaggingAnalysisCuts}
        \end{table}

        We are now interested in combining the information given by (A) and (B), and compare
        it to the case with no $W$-tagging usage (0). A straightforward approach consists
        in summing the two categories (A) and (B) together, compute the global
        exclusion-oriented significance of this sum (A+B), and compare it to the case with
        no $W$-tagging usage. The result of this comparison is presented on
        \reftab{tab:wTaggingSignificanceGain}.

        \begin{table}
            \centering
            \begin{tabular}{c|c|cc|c|c}
                                       & (0)               & (A)                          & (B)                    & (A+B)  & Gain (0) $\rightarrow$ (A+B)\\
                \hline
                \textbf{T2tt (600/0)}  & 2.22              & 1.50                         & 1.90                   & 2.39   & 1.08 \\
                \textbf{T2tt (650/0)}  & 1.73              & 1.26                         & 1.44                   & 1.88   & 1.09 \\
                \textbf{T2tt (700/0)}  & 1.27              & 0.93                         & 1.04                   & 1.37   & 1.08 \\
                \textbf{T2tt (750/0)}  & 0.97              & 0.69                         & 0.79                   & 1.02   & 1.05 \\
                \textbf{T2tt (800/0)}  & 0.70              & 0.55                         & 0.53                   & 0.73   & 1.05
            \end{tabular}
            \caption{Exclusion-oriented significance for signal benchmarks with increasing $\deltam$ in the different scenarios (0), (A), (B) and (A+B),
            as well as the gain of significance between (0) and (A+B).}
            \label{tab:wTaggingSignificanceGain}
        \end{table}

        The conclusion is therefore that the use of $W$-tagging may provide a mild gain of
        5$\sim$10\% on the total significance of the analysis in the high-$\deltam$ region.
        Some other studies have been dedicated to try to improve this gain.
        First, one
        may look for quantities related to the $W$-tagged jet that may help to further
        increase the performances. In example, it can be expected to find a $b$-tagged
        jet (or a jet with high $b$-taggign discriminator) in the proximity of the $W$-tagged
        jet to reconstruct the hadronic top. Nevertheless, this attempt did not yield
        any discriminating variable likely to improve the performance.
        Secondly, the statistical usage of the $W$-tagging category can be done in different
        ways. This variable may be used in input of the boosted decision tree to let it
        exploit the correlation between this topology and the rest of the event variables.
        One may also statistically combine the two categories (A) and (B) with multi-bins
        techniques, instead of merging them together into a single bin.

        \subsection{Sensitivity and perspectives for the Run II}
        % =====================================================

            \subsubsection{Extrapolation of the sensitivity from Run I analysis}
        % =====================================================

        One can get a basic estimation of the integrated luminosity $\mathcal{L}$ required
        at 13 TeV to obtain equivalent sensitivity compared to 8 TeV, starting from the
        following equation :

        $$ \left( \frac{S}{\sqrt{B}} \right)_{8\TeV} = \left( \frac{S}{\sqrt{B}} \right)_{13\TeV}  $$

        Using $N = \mathcal{L} \times \sigma \times \epsilon$ and assuming that the selection
        efficiencies $\epsilon$ remain the same between $8$ and $13\TeV$, one finds that

        $$ \mathcal{L}^\text{equiv.}_{13\TeV} = \mathcal{L}_{8\TeV} \times \frac{\kappa_B}{\kappa^2_S} $$

        where $\kappa \definedAs \sigma_{13\TeV} / \sigma_{8\TeV}$. For $\mass{\lstop} \sim 800\GeV$,
        $\kappa_S \sim 10$. From the experience at $8\TeV$ using the high-$\deltam$ selection,
        the dominant backgrounds are $t\bar{t}$ and $t\bar{t}+Z$. For these processes, one
        gets $\kappa_B \sim 3.3$. One ends up with

        $$ \mathcal{L}^\text{equiv.}_{13\TeV} \sim 0.7\invfb$$

        The sensitivity has been further studied on two Monte-Carlo benchmarks for the
        $\lstop \rightarrow t \lneutralino$ signal type with $(\mass{\lstop},
        \mass{\lneutralino}) = (650,325)$ and $(850,100) \GeV$. An object selection
        strongly inspired from \refsection{sec:analysis_objectAndEventSelection}, though simplified
        for this study, was used. A similar preselection is applied, requiring one electron
        or muon with $\pT > 30\GeV$, at least four jets among which one $b$-tagged, at least
        50 \GeV for $\MET$ and vetoing on a second lepton with $\pT > 5\GeV$. \reftab{tab:phys14Preselection}
        shows the Monte-Carlo yields for $\mathcal{L} = 1\invfb$ obtained at preselection
        and after cutting on $\MT > 120\GeV$.

        \begin{table}
            \centering
            \begin{tabular}{|l|cc|}
\hline
&
\textbf{preselection}     &
\textbf{preselection + $M_{T}$ > 120} \\
\hline
\textbf{$t\bar{t}$}      & 36583.82 $\pm$ 345.61     & 4666.13 $\pm$ 123.44  \\
\textbf{$W$+jets}        & 2842.44 $\pm$ 857.03      & 516.81 $\pm$ 365.44   \\
\textbf{$t\bar{t}V$}     & 145.34 $\pm$ 0.81         & 34.16 $\pm$ 0.40      \\
\hline
\textbf{total SM}        & 39571.60 $\pm$ 924.09     & 5217.10 $\pm$ 385.72  \\
\hline
\textbf{T2tt (850/100)}  & 2.69 $\pm$ 0.02           & 2.25 $\pm$ 0.02       \\
\textbf{T2tt (650/325)}  & 13.45 $\pm$ 0.12          & 9.39 $\pm$ 0.10       \\
\hline
\end{tabular}

            \caption{\todo{To be updated with higher stat} \label{tab:phys14Preselection}}
        \end{table}

        We define three signal regions SR 1, 2 and 3 inspired from the high-$\deltam$
        selection at $8\TeV$ using only increasing cuts on $\MET$ and $\MTTwoW$ as defined
        on \reftab{tab:phys14Cuts}. \reftab{tab:phys14SignalRegions} shows the yield
        obtained from Monte-Carlo assuming $\mathcal{L} = 1\invfb$.

        \begin{table}
            \centering
            \begin{tabular}{c|cc}
                \textbf{Signal Region} & $E_T^\text{miss}$ & $M_{T2}^W$ \\
                \hline
                \textbf{SR1}           & >250 & >180 \\
                \textbf{SR2}           & >300 & >190 \\
                \textbf{SR3}           & >350 & >200 \\
            \end{tabular}
            \caption{\todo{Write caption} \label{tab:phys14Cuts}}
        \end{table}

        \begin{table}
            \centering
            \begin{tabular}{|l|ccc|}
\hline
&
\textbf{SR1}   &
\textbf{SR2}   &
\textbf{SR3}   \\
\hline
\textbf{$\oneLeptonTop$}          & 2.10 $\pm$ 0.26   & 0.98 $\pm$ 0.18   & 0.59 $\pm$ 0.14 \\
\textbf{$\diLeptonTop$}           & 8.47 $\pm$ 0.53   & 3.17 $\pm$ 0.32   & 1.57 $\pm$ 0.23 \\
\textbf{$W$+jets}                 & 0.00 $\pm$ 0.00   & 0.00 $\pm$ 0.00   & 0.00 $\pm$ 0.00 \\
\textbf{rare}                     & 2.56 $\pm$ 0.46   & 1.38 $\pm$ 0.29   & 0.81 $\pm$ 0.21 \\
\hline
\textbf{total SM}                 & 13.12 $\pm$ 0.75  & 5.54 $\pm$ 0.47   & 2.97 $\pm$ 0.34 \\
\hline
\textbf{\textsc{T2tt} (850/100)}  & 1.34 $\pm$ 0.02   & 1.19 $\pm$ 0.01   & 1.03 $\pm$ 0.01 \\
\textbf{\textsc{T2tt} (650/325)}  & 2.61 $\pm$ 0.05   & 1.82 $\pm$ 0.04   & 1.15 $\pm$ 0.03 \\
\hline
\end{tabular}




            \caption{\todo{To be updated with higher stat} \label{tab:phys14SignalRegions}}
        \end{table}

       Finally, the sensitivity is estimated from the results in the SR2 region, as function
       of the integrated luminosity using $S / \sqrt{B + f^2 B^2}$ with a relative systematic
       uncertainty of $f = 50\%$ on the background. The results are presented on \reffig{fig:}
       and suggest that a signal with $(\mass{\lstop}, \mass{\lneutralino}) = (850,100)$
       could be excluded with $\mathcal{L} = 4\sim5\invfb$ or put in evidence with
       $\mathcal{L} = 10\sim15\invfb$.

            \insertFigure{phys14/sensitivityAsFunctionOfLumi}
                         {0.9}
                         {\todo{To be updated with higher stat (and maybe put discoverable/excluded signal strength instead ?)}}

%==============================================================
\chapternonum{Conclusion}

% There is no failure except in no longer trying. Elbert Hubbard

%==============================================================
        \loremipsum




\begin{thebibliography}{2}
   
\addReference{EllisDarkMatter}
             {J. Ellis, K. A. Olive}
             {Supersymmetric Dark Matter Candidates}
             {arXiv:1001.3651}

%====================================
%Naturalness, stop motivation related
%====================================

\addReference{LEPparadox}
             {R. Barbieri, A. Strumia}
             {The `LEP paradox'}
             {arXiv:hep-ph/0007265v2}

\addReference{LectureStandardModelHiggsBoson}
             {I. van Vulpen, A. Castelli}
             {Lecture on Particle Physics II, 2011-2012, The Standard Model Higgs Boson}
             {http://master.particles.nl/LectureNotes/2011-PPII-Higgs.pdf}

%\addReference{Naturalness}
%             {S. Dimopoulos, G.F. Giudice}
%             {Naturalness Constraints in Supersymmetric Theories with Non-Universal Soft Terms.}
%             {arXiv:hep-ph/9507282}

%\addReference{}
%Baryogenesis (to read)
%http://www.slac.stanford.edu/econf/C0508141/proc/pres/ALCPG0333_TALK.PDF
%https://kicp-workshops.uchicago.edu/DM-LHC2013/depot/talk-carena-marcela.pdf
%arXiv: 1009.3969, arXiv:1110.4378,
%http://arxiv.org/pdf/1207.6330v2.pdf
%http://arxiv.org/pdf/hep-ph/0404184v1.pdf

%=================
%Simplified models
%=================

\addReference{LiemSMS}
             {S. Liem}
             {Constraining Supersymmetry using Simplified Models}
             {urn:nbn:se:su:diva-91365}

\addReference{SmodelS}
             {S. Kraml et al.}
             {SModelS: a tool for interpreting simplified-model results from the LHC and its application to supersymmetry}
             {arXiv:1312.4175v3}

%=============
%Polarization
%=============

\addReference{polarization1}
             {M. Perelstein, A. Weiler}
             {Polarized Tops from Stop Decays at the LHC}
             {arXiv:0811.1024v2}

\addReference{polarization2}
             {Ian Low}
             {Polarized Charginos (and Tops) in Stop Decays}
             {arXiv:1304.0491v2}

%=======================
%FOM-related, statistics
%=======================

\addReference{PunziParameter}
             {G. Punzi}
             {Sensitivity of searches for new signals and its optimization}
             {arXiv:physics/0308063v2}

\addReference{TMVA}
             {A. Hoecker et al.}
             {TMVA - Toolkit for Multivariate Data Analysis}
             {arXiv:physics/0703039}

%========
%Analysis
%========

\addReference{SUS-12-023-PAS}
             {The CMS Collaboration}
             {Search for direct top squark pair production in events with a single isolated lepton, jets and missing transverse energy at sqrt(s) = 8 TeV}
             {CMS Physics Analysis Summary SUS-12-023}

\addReference{SUS-13-011-PUB}
             {The CMS Collaboration}
             {Search for top-squark pair production in the single-lepton final state in pp collisions at sqrt(s) = 8 TeV}
             {Eur. Phys. J. C 73 (2013) 2677. arXiv:1308.1586}

\addReference{SUS-14-015-PAS}
             {The CMS Collaboration}
             {Search for direct stop pair production in the single lepton channel at sqrt(s)=8 TeV}
             {CMS Physics Analysis Summary SUS-14-015}

%==========
%MET/sqrt(HT), MET significance
%==========

\addReference{METsignificanceMirman}
             {N. Mirman, Y. Wang, J. Alexander}
             {Missing transverse energy significance at CMS}
             {arXiv:1409.3028v1}

%========
%ISR-related
%========

\addReference{ISRtagging}
             {D. Krohn, L. Randall, L. Wang}
             {On the Feasibility and Utility of ISR Tagging}
             {arXiv:1101.0810}

\addReference{ISRGluinoTevatron}
             {J. Alwall et al.}
             {Searching for Directly Decaying Gluinos at the Tevatron}
             {arXiv:0803.0019}

\addReference{ISRSignalCorrection}
             {The CMS Collaboration}
             {Hadronic Recoil Studies of Heavy Boosted Systems}
             {CMS Analysis Note 2013/059}

%===========
%Misc / tmp
%===========

\addReference{ClosingStopGap}
             {Michal Czakon et al.}
             {Closing the stop gap}
             {arXiv:1407.1043}

\end{thebibliography}




